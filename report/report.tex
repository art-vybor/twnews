% utf-8 ru, unix eolns
\documentclass[12pt,a4paper,oneside]{extarticle}
    \righthyphenmin=2 %минимально переносится 2 символа %%%
    \sloppy

% Рукопись оформлена в соответствии с правилами оформления 
% электронной версии авторского оригинала, 
% принятыми в Издательстве МГТУ им. Н.Э. Баумана.

\usepackage{geometry} % А4, примерно 28-31 строк(а) на странице 
    \geometry{paper=a4paper}
    \geometry{includehead=false} % Нет верх. колонтитула
    \geometry{includefoot=true}  % Есть номер страницы
    \geometry{bindingoffset=0mm} % Переплет    : 0  мм
    \geometry{top=20mm}          % Поле верхнее: 20 мм
    \geometry{bottom=25mm}       % Поле нижнее : 25 мм 
    \geometry{left=25mm}         % Поле левое  : 25 мм
    \geometry{right=25mm}        % Поле правое : 25 мм
    \geometry{headsep=10mm}  % От края до верх. колонтитула: 10 мм
    \geometry{footskip=20mm} % От края до нижн. колонтитула: 20 мм 

\usepackage{cmap}
\usepackage[T2A]{fontenc} 
\usepackage[utf8x]{inputenc}
\usepackage[english,russian]{babel}
\usepackage{misccorr}

\usepackage{amsmath}
\usepackage{amsfonts}
\usepackage{amssymb}

%\usepackage{cm-super} %человеческий рендер русских шрифтов

\setlength{\parindent}{1.25cm}  % Абзацный отступ: 1,25 см
\usepackage{indentfirst}        % 1-й абзац имеет отступ

\usepackage{setspace}   

\onehalfspacing % Полуторный интервал между строками

\makeatletter
\renewcommand{\@oddfoot }{\hfil\thepage\hfil} % Номер стр.
\renewcommand{\@evenfoot}{\hfil\thepage\hfil} % Номер стр.
\renewcommand{\@oddhead }{} % Нет верх. колонтитула
\renewcommand{\@evenhead}{} % Нет верх. колонтитула
\makeatother

\usepackage{fancyvrb}

\usepackage[nounderscore]{syntax} %для поддержки рбнф
%\setlength{\grammarindent}{12em} %устанавливает нужный отступ перед ::=
\setlength{\grammarparsep}{6pt plus 1pt minus 1pt}  %сокращает расстояние между правилами


\usepackage[pdftex]{graphicx}  % поддержка картинок для пдф
\graphicspath{ {./pictures/} }
\usepackage{rotating}
\usepackage{graphicx}
%\DeclareGraphicsExtensions{.jpg,.png}

\renewcommand{\labelenumi}{\theenumi.} %меняет вид нумерованного списка

\usepackage{perpage} %нумерация сносок 
\MakePerPage{footnote}

\usepackage[all]{xy} %поддержка графов

\usepackage{listings} %листинги
\renewcommand{\lstlistingname}{Листинг}
\lstset{
  basicstyle=\small,
  breaklines=true
  }

\usepackage{url}


\usepackage{tikz} %для рисования графиков
\usepackage{pgfplots}

\usepackage{rotating}

\usepackage{ccaption}%изменяет подпись к рисунку
\makeatletter 
\renewcommand{\fnum@figure}[1]{Рисунок~\thefigure~---~\sffamily}
\makeatother





\begin{document}
\pgfplotsset{compat=1.8}

\thispagestyle{empty}
\newpage
{
\centering


\textbf{
МОСКОВСКИЙ ГОСУДАРСТВЕННЫЙ ТЕХНИЧЕСКИЙ УНИВЕРСИТЕТ ИМЕНИ Н. Э. БАУМАНА \\
Факультет информатики и систем управления \\
Кафедра теоретической информатики и компьютерных технологий}
\bigskip
\bigskip
\bigskip
\bigskip
\bigskip
\bigskip
\bigskip

\vfill

Курсовой проект \\
по курсу <<Конструирование компиляторов>>

\bigskip

{\large <<Препроцессор синаксического сахара для языка Scheme>>}
\bigskip

\vfill



\hfill\parbox{4cm} {
Выполнил:\\
студент ИУ9-101 \hfill \\
Выборнов А. И.\hfill \medskip\\
Руководитель:\\
Дубанов А. В.\hfill
}


\vspace{\fill}

Москва \number\year
\clearpage
}


\tableofcontents

\clearpage

\section*{Введение}
\addcontentsline{toc}{section}{Введение}

\section{Обзор литературы}

    \subsection{Linking Tweets to News: A Framework to Enrich Short Text Data in Social Media}
    \subsection{Bridging Vocabularies to Link Tweets and News}
    \subsection{TwitterStand: News in Tweets}
    \subsection{Gibberish, Assistant, or Master? Using Tweets Linking to News for Extractive Single-Document Summarization}
    \subsection{Detecting Event-Related Links and Sentiments from Social Media Texts}

\section{Полезные ссылки}

    \begin{itemize}
        \item Твиттер NYT: https://twitter.com/nytimes (20млн подписчиков, 200тыс твитов)
        \item Диплом на тему <<Определение тематической направленности текстового содержимого микроблогов>>: http://seminar.at.ispras.ru/wp-content/uploads/2012/07/Gomzin-thesis1.pdf
        \item Диплом на тему <<Разработка сервиса извлечения мнений>>: http://goo.gl/B7IKOH \
            есть инфа по работе с твиттером
        \item Крайне отстойная статья на тему: http://cyberleninka.ru/article/n/issledovanie-otklika-polzovateley-twitter-na-novosti-iz-smi
        \item Идея для формирования train: http://techcrunch.com/2013/08/19/twitter-related-headlines/

    \end{itemize}


\clearpage

\begin{thebibliography}{0}
\addcontentsline{toc}{section}{Список литературы}
    \bibitem{racket} Matthew Flatt, Robert Bruce Findler. The Racket Guide. Racket Documentation: URL: http://docs.racket-lang.org/guide/index.html.
    \bibitem{r5rs} R. Kelsey, W. Clinger, J. Rees. Revised$^5$ Report on the Algorithmic Language Scheme. Higher-Order and Symbolic Computation, Vol. 11, No. 1, August, 1998
    \bibitem{haskell} Simon Marlow. Haskell 2010 Language Report, 2010.
    \bibitem{antlr} Terence Parr. The Definitive ANTLR 4 Reference. 2013.
    \bibitem{dragon} Ахо, Альфред В., Лам, Моника С., Сети, Рави, Ульман, Джеффри Д. Компиляторы: принципы, технологии и инструментарий, 2-е изд.: Пер. с англ.~--~М.: ООО <<И.Д.Вильямс>>, 2008~--~1184 с.
    \bibitem{mcconnell} Макконнелл С., Совершенный код. Мастер-класс: Пер. с англ.~---~ М.: Издательство <<Русская редакция>>, 2014~---~896 стр.        
\end{thebibliography}

\end{document}

