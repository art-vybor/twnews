% utf-8 ru, unix eolns
\documentclass[12pt,a4paper,oneside]{extarticle}
    \righthyphenmin=2 %минимально переносится 2 символа %%%
    \sloppy

% Рукопись оформлена в соответствии с правилами оформления 
% электронной версии авторского оригинала, 
% принятыми в Издательстве МГТУ им. Н.Э. Баумана.

\usepackage{geometry} % А4, примерно 28-31 строк(а) на странице 
    \geometry{paper=a4paper}
    \geometry{includehead=false} % Нет верх. колонтитула
    \geometry{includefoot=true}  % Есть номер страницы
    \geometry{bindingoffset=0mm} % Переплет    : 0  мм
    \geometry{top=20mm}          % Поле верхнее: 20 мм
    \geometry{bottom=25mm}       % Поле нижнее : 25 мм 
    \geometry{left=25mm}         % Поле левое  : 25 мм
    \geometry{right=25mm}        % Поле правое : 25 мм
    \geometry{headsep=10mm}  % От края до верх. колонтитула: 10 мм
    \geometry{footskip=20mm} % От края до нижн. колонтитула: 20 мм 

\usepackage{cmap}
\usepackage[T2A]{fontenc} 
\usepackage[utf8x]{inputenc}
\usepackage[english,russian]{babel}
\usepackage{misccorr}

\usepackage{amsmath}
\usepackage{amsfonts}
\usepackage{amssymb}

%\usepackage{cm-super} %человеческий рендер русских шрифтов

\setlength{\parindent}{1.25cm}  % Абзацный отступ: 1,25 см
\usepackage{indentfirst}        % 1-й абзац имеет отступ

\usepackage{setspace}   

\onehalfspacing % Полуторный интервал между строками

\makeatletter
\renewcommand{\@oddfoot }{\hfil\thepage\hfil} % Номер стр.
\renewcommand{\@evenfoot}{\hfil\thepage\hfil} % Номер стр.
\renewcommand{\@oddhead }{} % Нет верх. колонтитула
\renewcommand{\@evenhead}{} % Нет верх. колонтитула
\makeatother

\usepackage{fancyvrb}

\usepackage[nounderscore]{syntax} %для поддержки рбнф
%\setlength{\grammarindent}{12em} %устанавливает нужный отступ перед ::=
\setlength{\grammarparsep}{6pt plus 1pt minus 1pt}  %сокращает расстояние между правилами


\usepackage[pdftex]{graphicx}  % поддержка картинок для пдф
\graphicspath{ {./pictures/} }
\usepackage{rotating}
\usepackage{graphicx}
%\DeclareGraphicsExtensions{.jpg,.png}

\renewcommand{\labelenumi}{\theenumi.} %меняет вид нумерованного списка

\usepackage{perpage} %нумерация сносок 
\MakePerPage{footnote}

\usepackage[all]{xy} %поддержка графов

\usepackage{listings} %листинги
\renewcommand{\lstlistingname}{Листинг}
\lstset{
  basicstyle=\small,
  breaklines=true
  }

\usepackage{url}


\usepackage{tikz} %для рисования графиков
\usepackage{pgfplots}

\usepackage{rotating}

\usepackage{ccaption}%изменяет подпись к рисунку
\makeatletter 
\renewcommand{\fnum@figure}[1]{Рисунок~\thefigure~---~\sffamily}
\makeatother





\begin{document}
\pgfplotsset{compat=1.8}

\thispagestyle{empty}
\newpage
{
\centering


\textbf{
МОСКОВСКИЙ ГОСУДАРСТВЕННЫЙ ТЕХНИЧЕСКИЙ УНИВЕРСИТЕТ ИМЕНИ Н. Э. БАУМАНА \\
Факультет информатики и систем управления \\
Кафедра теоретической информатики и компьютерных технологий}
\bigskip
\bigskip
\bigskip
\bigskip
\bigskip
\bigskip
\bigskip

\vfill

Курсовой проект \\
по курсу <<Конструирование компиляторов>>

\bigskip

{\large <<Препроцессор синаксического сахара для языка Scheme>>}
\bigskip

\vfill



\hfill\parbox{4cm} {
Выполнил:\\
студент ИУ9-101 \hfill \\
Выборнов А. И.\hfill \medskip\\
Руководитель:\\
Дубанов А. В.\hfill
}


\vspace{\fill}

Москва \number\year
\clearpage
}


\tableofcontents

\clearpage

\section*{Введение}
\addcontentsline{toc}{section}{Введение}

\section{Обзор литературы}

    \subsection{Linking Tweets to News: A Framework to Enrich Short Text Data in Social Media}
        \subsubsection{Перевод аннотации}
        Многие современные методы обработки естественного языка~(NLP\footnote{Natural Language Processing}) хорошо работают, используя большой массив текста в качестве входных данных.
        Однако они становятся неэффективными, когда применяются на коротких текстах, таких как твиты.
        Чтобы преодолеть эту проблему, мы хотим найти соответствующий данному твиту новостной документ, для большей эффективности NLP задач.
        Это требует хорошего моделирования и понимания семантики коротких текстовых твитов.

        Вклад статьи двойной:
        1. мы предcтавим задачу связывания твитов с новостями, а также набор пар твит-новости, из этого могут извлечь выгоду многие NLP задачи;
        2. в отличие от предыдущих исследований, которые фокусируются на лексических особенностях в коротких текстов~(информация о связи текст-слово), мы предлагаем граф, основанный на модели скрытой переменной, которая моделирует корреляцию между короткими текстами~(информация о связи текст-текст).
        Это обосновано наблюдением: твит обычно покрывает только один аспект события.
        Мы покажем, что c помощью особенных признаков твита~(хэштегов) и особых признаков новостей~(именнованные сущности\footnote{Какой-то кривой перевод, найдо найти получше. In data mining, a named entity is a phrase that clearly identifies one item from a set of other items that have similar attributes.}) а также временн\'{ы}х ограничений, мы можем получить взаимосвязь текст-текст, и, таким образом, дополнить семантическую картину короткого текста.
        Наши эксперименты показывают значительное преимущество нашей новой модели над baseline\footnote{Как перевести?} для трёх методов оценки.

        \subsubsection{Краткое изложение статьи}
        Современные NLP подходы на коротких текстах плохо работают или не работают вообще.
        Чтобы это преодолеть предлагается к твитам привязывать соответствующие новости.

        Для train выкачали твиты за 18 дней, которые имели ссылки на CNN или NYT, опубликованные в этот период.

        Модели со скрытой переменной хорошо подходят для отображения коротких сообщений в малоразмерный вектор.

        Построили модель со скрытой переменной, чтобы применять её и к твитам и к новостям. Протестили модель на наборе данных из небольших сообщений: с большим запасом превзошла и LSA и LDA.


        \subsubsection{Используемые технологии}

        \subsubsection{Основная идея}
        

        \subsubsection{Полученные результаты}

    \subsection{Linking Online News and Social Media}
        \subsubsection{Перевод аннотации}
            Многое из того, что обсуждается в социальных медиа вдохновлено событиями, описанными в новостях и, наоборот, социальные медиа предоставляют механизм, позволяющий влиять на новостные события.
            Мы обращаемся к следующей связывающей задаче: по новости, найти в социальных сетях высказывания, которые неявно на неё ссылаются.
            Мы используем трехступенчатый подход: сначала получаем несколько моделей запросов по исходной статье, модели используются для получения высказываний из индекса целевого социального медиа, в результате получаем несколько ранжированных списков, которые объединяются с использованием техники слияния данных.
            Модели запроса создаются на основе структуры исходной статьи и явно связанных высказываний из социальных медиа, в которых обсуждается исходная статья.
            Для борьбы с дрейфом запроса\footnote{Порождение менее подходящего запроса.} в результате большого объема текста, либо в самой исходной новости, либо в явно связанных высказываниях в социальных медиа, мы предлагаем основанный на графике\footnote{Или всё же графах?} метод для выбора отличительных условий.

            Для нашей экспериментальной оценки, мы используем данные из Twitter, Digg, Delicious\footnote{Веб-сайт, бесплатно дающий зарегистрированным пользователям услугу хранения и публикации закладок на страницы Всемирной сети.}, the New York Times Community, Wikipedia и блогосферы, для порождения моделей запросов.
            Мы покажем, что другие модели запросов, основанные на различных источниках данных, обеспечивают дополнительную информацию и влияют на получение различных высказываний из социальные медиа по нашему целевому индексу.
            Как следствие, методы слияния данных приводят к значительному повышению производительности в сравнении с индивидуальными подходами.
            Показано, что основанный на графике метод выделения условий помог улучшить как эффективность, так и продуктивность.
        
        \subsubsection{Используемые технологии}

        \subsubsection{Основная идея}

        \subsubsection{Полученные результаты}

        Ещё одна статья от Лукашевич
    \subsection{Bridging Vocabularies to Link Tweets and News}
        Ещё один подход
    \subsection{TwitterStand: News in Tweets}

\section{Дополнительная литература}
    \subsection{Gibberish, Assistant, or Master? Using Tweets Linking to News for Extractive Single-Document Summarization}
    \subsection{Detecting Event-Related Links and Sentiments from Social Media Texts}
    \subsection{Определение тематической направленности текстового содержимого микроблогов}
    \subsection{Разработка сервиса извлечения мнений}
    

\section{Полезные ссылки}
    \begin{itemize}
        \item Твиттер NYT: https://twitter.com/nytimes (20млн подписчиков, 200тыс твитов)
        \item Крайне отстойная статья на тему: http://cyberleninka.ru/article/n/issledovanie-otklika-polzovateley-twitter-na-novosti-iz-smi
        \item Идея для формирования train: http://techcrunch.com/2013/08/19/twitter-related-headlines/
    \end{itemize}
\clearpage

\section{Формирование датасетов}
    Для формирования датасетов нужно сделать:
    \begin{enumerate}
        \item Выбрать источники данных.
        \item Выбрать формат хранения.
        \item Выбрать БД, для записи результата.
        \item Написать отказоустойчивый консьюмер.
        \item В течение длительного времени собрать данные.
    \end{enumerate}

    Предлагаю выбор формата хранения и БД отложить до тех пор, пока не будут получены все необходимые данные. А пока коллекционировать все возможные данные в текстовых файлах.

    В качестве источников данных 

    \subsection{Примерно что делал}
        Лог работы в raw формате :) будет много копипасты с используемых мануалов
 \begin{enumerate}
    \item Step 1: Getting Twitter API keys

        In order to access Twitter Streaming API, we need to get 4 pieces of information from Twitter: API key, API secret, Access token and Access token secret. Follow the steps below to get all 4 elements:

        Create a twitter account if you do not already have one.
        Go to https://apps.twitter.com/ and log in with your twitter credentials.
        Click "Create New App"
        Fill out the form, agree to the terms, and click "Create your Twitter application"
        In the next page, click on "API keys" tab, and copy your "API key" and "API secret".
        Scroll down and click "Create my access token", and copy your "Access token" and "Access token secret".
    \item 
    \end{enumerate}

\section{Используемое ПО}
    \subsection{tweepy}
        https://github.com/tweepy/tweepy


\begin{thebibliography}{0}
\addcontentsline{toc}{section}{Список литературы}
    \bibitem{linking_base} W. Guo, H. Li, H. Ji, and M. T. Diab. Linking tweets to news: A framework to enrich short text data in social media. - ACL, pages 239–249, 2013.
    \bibitem{bridging} T. Hoang-Vu, A. Bessa, L. Barbosa and J. Freire. Bridging Vocabularies to Link Tweets and News. - International Workshop on the Web and Databases (WebDB 2014), Snowbird, Utah, US, 2014.
    \bibitem{twitterstand} J. Sankaranarayanan, H. Samet, B. Teitler, M. Lieberman, J. Sperling. TwitterStand: news in tweets. - 17th ACM SIGSPATIAL International Conference on Advances in Geographic Information Systems, 2009, Seattle, Washington.

    \hrulefill

    \bibitem{racket} Matthew Flatt, Robert Bruce Findler. The Racket Guide. Racket Documentation: URL: http://docs.racket-lang.org/guide/index.html.
    \bibitem{r5rs} R. Kelsey, W. Clinger, J. Rees. Revised$^5$ Report on the Algorithmic Language Scheme. Higher-Order and Symbolic Computation, Vol. 11, No. 1, August, 1998
    \bibitem{haskell} Simon Marlow. Haskell 2010 Language Report, 2010.
    \bibitem{antlr} Terence Parr. The Definitive ANTLR 4 Reference. 2013.
    \bibitem{dragon} Ахо, Альфред В., Лам, Моника С., Сети, Рави, Ульман, Джеффри Д. Компиляторы: принципы, технологии и инструментарий, 2-е изд.: Пер. с англ.~--~М.: ООО <<И.Д.Вильямс>>, 2008~--~1184 с.
    \bibitem{mcconnell} Макконнелл С., Совершенный код. Мастер-класс: Пер. с англ.~---~ М.: Издательство <<Русская редакция>>, 2014~---~896 стр.        
\end{thebibliography}

\end{document}

