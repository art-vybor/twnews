\subsection{Выбор подхода для решения задачи}
    В качестве основного подхода, на основе которого строится решение задачи по установлению связей между твитами и новостями, был выбран WTMF-G.
    Основной причиной подобного выбора является то, что большинство подходов учитывают только статистические зависимости вида текст-слово;
    метод WTMF-G, напротив, не ограничивается зависимостями текст-слово, а позволяет учесть взаимосвязь текст-текст,
    что, как ожидается, даст прирост качества в решении задачи установления связей.

    Также, в рамках работы задача по установления связей между твитами и новостями решена классическим подходом для установления связей между текстами~---~
    определение схожести текстов на основе частотности употребления слов. Этот подход даёт хорошие результаты на больших текстах.
    Результаты этого метода помогут оценить влияние связей вида текст-текст в методе WTMF-G на качество полученного решения.


