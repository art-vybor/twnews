\subsection{!Терминология}
    \textcolor{red}{Глава будет пополняться терминами в течение написании записки}

    \textit{Твиттер}~---~cоциальная сеть для публичного обмена короткими~(до 140 символов) сообщениями при помощи веб-интерфейса, SMS, средств мгновенного обмена сообщениями или сторонних программ-клиентов.

    \textit{Твит}~---~термин сервиса микроблоггинга Твиттер, обозначающий сообщение, публикуемое пользователем в его твиттере.
    Особенностью твита является его длина, которая не может быть больше 140 знаков.

    \textit{Новость}~---~оперативное информационное сообщение, которое представляет политический, социальный или экономический интерес для аудитории в своей свежести, то есть сообщение о событиях произошедших недавно или происходящих в данный момент.

    \textit{Обработка естественного языка}~---~

    \textit{Tf-idf}~---~

    \textit{WTMF}~---~

    \textit{Именованная сущность}~---~

    \textit{Тематическое моделирование}~---~

    \textit{LDA}~---~

    \textit{Информационный поиск}~---~

    \textit{URL}~---~

%    \textit{Классификация}~---~Это осмысленный порядок вещей, явлений, разделение их на разновидности согласно каким-либо важным признакам.
%
%    \textit{Тематическая модель}~---~модель, которая по коллекции текстовых документов, определяет, к каким темам относится каждый документ и какие слова~(термины) образуют каждую тему.
%
%    \textit{Тональность текста}~---~эмоциональная оценка автора по отношению к объектам, речь о которых идет в тексте.
%
%    \textit{Аннотирование текста}~---~краткое представление содержания текста в виде аннотации~(обзорного реферата).
%
%    \textit{Точность}~---~доля релевантных документов выборки по отношению ко всем документам в выборке.
%
%    \textit{Полнота}~---~доля релевантных документов в выборке по отношению ко всем релевантным документам коллекции.


