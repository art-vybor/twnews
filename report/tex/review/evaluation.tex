\subsection{Оценка качества}
    формулирование того как переходим к рекомендациям и что вообще в них меряем

    \subsubsection{RR}
        описание метрики
    \subsubsection{TOP10}
        описание метрики
    \subsubsection{ATOP}
        описание метрики и обоснование почему она не очень

        Используем метрику $ATOP$ (метрика подробно описана в \cite{steck_recommender}).
        Рассмотрим что означает эта метрика в применении к нашей задаче (я немного модифицировал метрику, для более простого описания, полученная метрика полностью совпадает с описанной метрикой).
        Пусть $T$ - это множество твитов, $N \in \mathbb{N}$ - размер рассматриваемого топа новостей для твита (могут быть все новости вообще), $k < N \in \mathbb{N}$.
        $TOPK_t(k) = 1$, если твит $t \in T$ соответствует хотя бы одной новости в top-k результатов, иначе $TOPK_t(k) = 0$
        $$TOPK(k) = \dfrac {\sum_{t \in T} TOPK_t(k)} {|T|},$$
        $$ATOP = \dfrac{\sum_{k=1}^N TOPK_t(k)}{N} = \dfrac{1}{|T| * N} \sum_{k=\overline{1,N}, ~t \in T} TOPK_t(k).$$

        Значения метрики $ATOP$ лежат на отрезке $[0,1]$. Чем ближе $ATOP$ к $1$ тем лучше.