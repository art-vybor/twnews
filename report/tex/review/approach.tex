\subsection{Существующие подходы к решению задачи}
    %мб и не нужна следующая фраза:
    %В рамках предварительного исследования были разобраны несколько статей . Ниже приводится краткое изложение основных идей, описанных в выбранных статьях.
    Задача автоматического установления связей между твитами и новостными статьями пока не имеет устоявшегося решения.
    В рамках предварительного исследования были отобраны наиболее перспективные подходы к решению задачи, а именно:
    \begin{itemize}
        \item Метод WTMT-G, представляющий собой доработку метода WTMF, которая позволила учитывать информацию о связях между текстами. Предложен в статье \cite{linking_base}.
        \item Обобщённый метод, позволяющий по новости находить относящиеся к ней высказывания из социальных медиа.
        \item Связывание твитов с новостями на основе bridging словарей
        \textcolor{red}{(нужен перевод слова bridging, в конктексте это примерно значит: связующие словарии или перекрывающие)}.
    \end{itemize}
    Также рассматривается классическое решение задачи поиска похожих текстов на основе частотности употребления слов.
    Ниже представлен краткий обзор выбранных методов.

    \subsubsection{Поиск похожих текстов на основе частотности употребления слов}
        Наиболее простым и очевидным подходом к решению задачи связывания твитов с новостными статьями, является связывание на основе сходства по частотности употребления слов.

        Для получения информации о частотности употребления слов используется \textit{tf-idf матрица}~---~матрица, строки которой соответствуют словам из корпуса, а столбцы текстам.
        Значение ячейки матрицы $(i,j)$ равно значению метрики tf-idf для слова, соответствующего строчке $i$, и текста, соответствующего столбцу $j$.

        Решение задачи связывания твитов с новостными статьями на основе частотности употребления слов можно представить в виде небольшого алгоритма:
        \begin{enumerate}
            \item объединить тексты всех твитов и тексты всех новостей (для новости текст это конкатенация заголовка и краткого изложения);
            \item в качестве корпуса использовать объединение нормальных форм всех слов, используемых в текстах, за вычетом списка стоп-слов;
            \item по множеству текстов и построенному корпусу построить tf-idf матрицу;
            \item каждому тексту сопоставить столбец tf-idf матрицы, соответствующий тексту (вектор для сравнения);
            \item рассматривая вектор для сравнения в качестве координат в метрическом пространстве, для каждого твита найти список наиболее похожих на него новостей.
        \end{enumerate}

        В качестве меры близости в метрическом пространстве в работе используется косинусная мера близости~---~мера численно равная косинусу угла между векторами.

    \subsubsection{WTMF-G}
        по статье Linking Tweets to News: A Framework to Enrich Short Text Data in Social Media

        В двух-трёх абзацах описание подхода, описание задачи убрать.

        \paragraph{Перевод аннотации}
            Многие современные методы обработки естественного языка~(NLP\footnote{Natural Language Processing}) хорошо работают с большой массив текста в качестве входных данных.
            Однако они очень неэффективными при работе с короткими текстами (к примеру твиты).
            Преодоление этой проблемы мы видим в нахождении соответствующего твиту новостного документа.
            Решение этой задачи требует хорошего моделирования семантики коротких текстов.

            Основной вклад статьи двойной:
            \begin{enumerate}
                \item представлено решение задачи нахождения взаимосвязи между твитами и новостями, из этого могут извлечь выгоду многие NLP задачи;
                \item в отличие от предыдущих исследований, которые фокусируются на лексических особенностях коротких текстов~(информация о связи текст-слово), мы предлагаем взаимосвязь, основанную на модели скрытой переменной, которая моделирует корреляцию между короткими текстами~(информация о связи текст-текст). Необходимость этого обоснована наблюдением: твит обычно покрывает только один аспект события.
            \end{enumerate}

            Мы покажем, что c помощью особенных признаков твита~(хэштегов) и особых признаков новостей~(именнованных сущностей\footnote{Какой-то кривой перевод, найдо найти получше. In data mining, a named entity is a phrase that clearly identifies one item from a set of other items that have similar attributes.}) а также временн\'{ы}х ограничений, мы можем получить взаимосвязь текст-текст, и, таким образом, дополнить семантическую картину короткого текста.
            Наши эксперименты показывают значительное преимущество нашей новой модели над baseline\footnote{Как перевести?}.

        \paragraph{Идея статьи}
            Современные методы обработки естественного языка плохо работают с короткими текстами. Для преоболения этого к твитам привязываются соответствующие новости.

            Для формирования обучающей выборки, были выбраны твиты, которые имели ссылки на новости, опубликованные новостными агенствами (CNN или NYT) в тот же период.

            Как показано в статье~\cite{long_to_short}, добавление к твиту содержимого веб-страницы, ссылка на которую включена в этот твит, повышает {\color{red} purity score} их кластеризации с 0.280 до 0.392.

            Модели со скрытой переменной хорошо подходят для отображения коротких текстов в плотный малоразмерный вектор.
            В рамках решения задачи была применена модель со скрытой переменной, которая называется WTMF~(Weighted Textual Matrix Factorization, подробное описание\cite{wtmf}), к твитам и к новостям. Модель была протестирована на двух схожих наборах данных из небольших сообщений. Как результат - используемая модель с большим запасом превзошла и LSA~(Latent Semantic Analysis) и LDA~(Latent Dirichelet Allocation). Эта модель позволила добавить информацию об отсутствующих словах в твит (модель WTMF добавляет более 1000 фичей к твиту, LDA лишь 14). Недостатком WTMF является то, что порождается только связь текст-слово, без учёта взаимосвязи между короткими текстами.

            Ввиду разреженности исходных данных, возникает ещё одна проблема: твит обычно отражает, только один аспект события.

            Полученный подход не учитывает следующих характеристик, которым обладает исходная выборка:
            \begin{enumerate}
                \item Хэштеги, которые являются прямым указанием на смысл твита.
                \item {\color{red} Named entities} новостей. Из новостей можно с высокой точностью извлекать {\color{red} named entities}, используя инструменты для NER~(Named Entity Recognition). Если несколько текстов содержат схожие {\color{red} named entities} они наверняка описывают одно и тоже событие.
                \item Информация о времени публикации для твитов и новостей. Если несколько текстов опубликованы примерно в одно и то же время, то велик шанс, что они описывают одно и тоже событие
            \end{enumerate}
            В статье описывается решение проблемы поиска взаимосвязи между текстами, с использованием описанных выше характеристик. Два связанных текста, должны иметь схожий скрытый вектор (семантическая модель твита достраивается из схожих твитов).

            Это дополнительная информация была добавлена в модель WTMF. Было также показано различное влияние на связь текст-текст жанра твита и жанра новости. Был получен на порядок более лучший результат чем при использовании исходной WTMF модель.

    \subsubsection{Обобщённый метод, сопоставляющий новостной статье высказывания из социальных медиа}
        В рамках метода решается следующая задача: по новости необходимо найти высказывания в социальных сетях, которые на неё неявно ссылаются.
        Метод был предложен в статье Linking Online News and Social Media~\cite{linking_news_media}.

        Поставленная задача решается в три этапа:
        \begin{enumerate}
            \item по заданной новостной статье формируется несколько моделей запросов, которые создаются как на основе структуры статьи,
                так и на основе явно связанных со статьей высказываний из социальных медиа.
            \item построенные модели используются для получения высказываний из индекса целевого социального медиа, результатом являются несколько ранжированных списков;
            \item полученные списки объединяются с использованием особой техники слияния данных.
        \end{enumerate}

        Авторы также предлагают способ, созданный для борьбы с дрейфом запроса\footnote{Под дрейфом запроса подразумевается порождение менее подходящего запроса.}
        при большого объёме используемого текста.
        Способ основан на выборе дополнительных отличительных условий.

        Для экспериментальной оценки используются данные из различных медиа, таких как Twitter, Digg, Delicious, the New York Times Community, Wikipedia, а так же из блогов.
        %\footnote{Веб-сайт, бесплатно дающий зарегистрированным пользователям услугу хранения и публикации закладок на страницы Всемирной сети.}

        В результате работы показано, что модели запросов, основанные на различных источниках данных, повышают точность выявления высказываний из социальных медиа;
        методы слияния ранжированных списков приводят к значительному повышению производительности в сравнении с другими подходами.

    \subsubsection{Связывание твитов с новостями на основе \textcolor{red}{bridging словарей}}
        Метод связывания твитов с новостными статья, основанный на \textcolor{red}{bridging словарей}, предложен в статье Bridging Vocabularies to Link Tweets and News~\cite{bridging}.
        Авторы предложили способ автоматического установление связи между множеством твитов и множеством новостей определённой темы.
        Темы извлекаются из новостей на основе методов тематического моделирования.

        Под \textcolor{red}{\textit{bridging словарём}} авторы подразумевают множество слов, которые встречаются только в твитах и, соответственно, не встречаются в новстях.

        Значительную сложность при решении проблемы связывания твитов с новостями вызывают малый размер твита и различия в словарях: в твитах используются аббревиатуры,
        неформальный язык, сленг; в новостях, напротив, используется литературный язык.
        В частности, твиты могут вообще не нести смысловой нагрузки.

        Твиттер предлагает хештэги, как механизм для категоризации твитов.
        Но этот подход обладает рядом недостатков, таких как: не все записи содержат хештеги, хeштег не содержит информацию о событии, хeштег сформулирован в слишком общей форме,
        твит содержит несколько хештегов.
        Следовательно использование одних хештегов приведёт к низкому качеству связывания твитов с новостями.

         Для решения задачи и преодоления описанных выше проблем, авторами работы предлагается следующий подход:
         \begin{enumerate}
            \item С помощью метода LDA из множества новостей извлекается набор тем. Тема характеризуется распределением частот слов, характерных для этой темы.
            \item К каждой полученной теме сопоставляется множество наиболее близких к ней твитов.
            \item Из полученных твитов извлекаются слова, которые дополняют характеристику рассматриваемой темы.
            \item Полученные слова образуют \textcolor{red}{Bridging} словарь и служат \textcolor{red}{<<мостом>>} к другим твитам.
         \end{enumerate}

         ...