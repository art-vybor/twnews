\subsection{Существующие подходы к решению задачи}
    Задача автоматического установления связей между твитами и новостными статьями до сих пор не имеет устоявшегося решения.
    В рамках предварительного исследования были отобраны наиболее перспективные подходы к решению задачи, а именно:
    \begin{itemize}
        \item метод WTMT-G, представляющий собой доработку метода WTMF, которая позволила учитывать информацию о связях между текстами;
        \item обобщённый метод, позволяющий по новости находить относящиеся к ней высказывания из социальных медиа;
        \item связывание твитов с новостями на основе \textcolor{red}{bridging словарей};
    \end{itemize}
    Также рассматривается классическое решение задачи определения схожести текстов на основе частотности употребления слов.
    Ниже представлен краткий обзор выбранных методов.

    Стоит также ввести несколько определений употребляемых в дальнейшем:
    под \textit{связью текст-текст} подразумевается определение двух текстов как схожих на основе некоторой дополнительной информации;
    под \textit{связью текст-слово} подразумевается определение двух текстов как схожих только на основе слов, из которых состоит текст.

    \subsubsection{Определение схожести текстов на основе частотности употребления слов}
        Наиболее простым и очевидным подходом к решению задачи связывания твитов с новостными статьями, является связывание наиболее текстов,
        наиболее близких по частотности употребления слов.

        Решение задачи связывания твитов с новостными статьями на основе частотности употребления слов можно представить в виде небольшого алгоритма:
        \begin{enumerate}
            \item объединить тексты всех твитов и тексты всех новостей~---~(для новости текст это конкатенация заголовка и краткого изложения);
            \item в качестве корпуса использовать объединение начальных форм всех слов, используемых в текстах, за вычетом списка стоп-слов~
            (под списком стоп-слов подразумевается набор часто употребимых слов языка, которые вне контекста не несут смысловой нагрузки, к примеру, предлоги);
            \item по множеству текстов и построенному корпусу построить TF-IDF матрицу;
            \item каждому тексту сопоставить столбец TF-IDF матрицы, соответствующий тексту (вектор для сравнения);
            \item рассматривая вектор для сравнения в качестве координат в метрическом пространстве, для каждого твита найти список наиболее похожих на него новостей.
        \end{enumerate}

        В работе в качестве меры близости в метрическом пространстве используется косинусная мера близости~---~мера численно равная косинусу угла между векторами.
        В дальнейшем каждый раз, когда говориться о схожести или близости двух векторов, подразумевается близость согласно косинусной мере.

    \subsubsection{Метод WTMF-G}
        Метод WTMG-G решает задачу установления связей между твитами и новостными статьями, путём построения модели, которая учитывает неявные связи между текстами.
        Метод был предложен в статье Linking Tweets to News: A Framework to Enrich Short Text Data in Social Media~\cite{linking_base}.

        Метод WTMF-G представляет собой доработанный метод WTMF, который позволяет хорошо моделировать семантику коротких текстов,
        но не учитывает некоторые специфичные для твитов и новостей характеристики, которыми обладает исходная выборка и
        которые взаимосвязаны с семантической близостью текстов:
        \begin{enumerate}
            \item хештеги, которые являются прямым указанием на смысл твита;
            \item именованные сущности, которые с высокой точностью можно извлекать из новостей;
            \item информацию о времени публикации твитов и новостей.
        \end{enumerate}
        Метод WTMF-G расширяет возможности метода WTMF, путём учёта взаимосвязи текстов на основе специфичных для твитов и новостных статей характеристик,
        то есть позволяет учесть информацию о взаимосвязи текст-текст.

        В статье показано, что добавление информации о взаимосвязи текст-текст позволяет повысить качество установления связей между твитами и новостными статьями.

    \subsubsection{Обобщённый метод, сопоставляющий новостной статье высказывания из социальных медиа}
        В рамках метода решается следующая задача: по новости необходимо найти высказывания в социальных сетях, которые на неё неявно ссылаются.
        Метод был предложен в статье Linking Online News and Social Media~\cite{linking_news_media}.

        Поставленная задача решается в три этапа:
        \begin{enumerate}
            \item по заданной новостной статье формируется несколько моделей запросов, которые создаются как на основе структуры статьи,
                так и на основе явно связанных со статьей высказываний из социальных медиа.
            \item построенные модели используются для получения высказываний из индекса целевого социального медиа, результатом являются несколько ранжированных списков;
            \item полученные списки объединяются с использованием особой техники слияния данных.
        \end{enumerate}

        Авторы также предлагают способ, созданный для борьбы с дрейфом запроса~(порождение менее подходящего запроса), который возникает при большого объёме используемого текста.
        Способ основан на выборе дополнительных отличительных условий.

        Для экспериментальной оценки используются данные из различных медиа, таких как Twitter, Digg, Delicious, the New York Times Community, Wikipedia, а так же из блогов.
        %\footnote{Веб-сайт, бесплатно дающий зарегистрированным пользователям услугу хранения и публикации закладок на страницы Всемирной сети.}

        В результате работы показано, что модели запросов, основанные на различных источниках данных, повышают точность выявления высказываний из социальных медиа;
        методы слияния ранжированных списков приводят к значительному повышению производительности в сравнении с другими подходами.

    \subsubsection{Связывание твитов с новостями на основе \textcolor{red}{bridging словарей}}
        Метод связывания твитов с новостными статья, основанный на \textcolor{red}{bridging словарях}~(множество слов,
        которые встречаются только в твитах и, соответственно, не встречаются в новостях), предложен в статье Bridging Vocabularies to Link Tweets and News~\cite{bridging}.
        Авторы предложили способ автоматического установление связи между множеством твитов и множеством новостей определённой темы.
        Темы извлекаются из новостей на основе методов тематического моделирования.

        Значительную сложность при решении проблемы связывания твитов с новостями вызывают малый размер твита и различия в словарях: в твитах используются аббревиатуры,
        неформальный язык, сленг; в новостях, напротив, используется литературный язык.
        В частности, твиты могут вообще не нести смысловой нагрузки.

        Твиттер предлагает хештэги, как механизм для категоризации твитов.
        Но этот подход обладает рядом недостатков, таких как: не все записи содержат хештеги, хeштег не содержит информацию о событии, хeштег сформулирован в слишком общей форме,
        твит содержит несколько хештегов.
        Следовательно использование одних хештегов приведёт к низкому качеству связывания твитов с новостями.

         Для решения задачи и преодоления описанных выше проблем, авторами работы предлагается следующий подход:
         \begin{enumerate}
            \item С помощью метода LDA из множества новостей извлекается набор тем. Тема характеризуется распределением частот слов, характерных для этой темы.
            \item К каждой полученной теме сопоставляется множество наиболее близких к ней твитов.
            \item Из полученных твитов извлекаются слова, которые дополняют характеристику рассматриваемой темы.
            \item Полученные слова образуют \textcolor{red}{Bridging} словарь и служат <<мостом>> к другим твитам.
         \end{enumerate}

         В результате работы продемонстрирован способ установления связей между множеством твитов и множеством новостей с использованием \textcolor{red}{bridging словарей}.
