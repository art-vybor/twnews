\subsection{Сформированные наборы данных}
    На основе собранной информации было сформировано несколько базовых эталонных наборов, а именно:
    \begin{enumerate}
        \item auto~---~автоматически размеченный набор данных;
        \item manual~--~вручную размеченный набор данных;
        \item total~---~набор данных состоящий из объединения всех размеченных связей (то есть объединение auto и manual);
        \item cutted~---~набор данных, основанный на наборе total, в котором количество новостей сравнимо с количеством твитов~
        (набор данных создавался с целью изучения влияния соотношения количества новостей и твитов на качество установления связей).
    \end{enumerate}
    Также рассматриваются эталонные наборы данных без тривиальных связей, подобный набор образуется путём удаления из базового эталонного набора твитов, образующих тривиальную связь.
    Обозначим эталонные наборы данных с удалёнными тривиальными связями как auto\_nt, manual\_nt, total\_nt и cutted\_nt,
    полученные путём удаления тривиальных связей из базовых эталонных наборов auto, manual, total и cutted, соответственно.

    Ключевая информация характеризующая построенные эталонные наборы представлена в таблице~\ref{tabular:dataset_info}.
    \begin{table}[ht!]
        %\small
        \caption{Сводная таблица по эталонным наборам данных\bigskip}
        \centering

        \label{tabular:dataset_info}
        \begin{tabular}{|c|c|c|}
            \hline
            \bf{\specialcell{Набор данных}} &
            \bf{\specialcell{Количество твитов}} &
            \bf{\specialcell{Количество новостей}} \\ \hline
            manual & 1600 & 13711 \\ \hline
            auto & 4324 & 13711 \\ \hline
            total & 5798 & 13711 \\ \hline
            cutted & 5798 & 6011 \\ \hline
            manual\_nt & 976 & 13711 \\ \hline
            auto\_nt & 746 & 13711 \\ \hline
            total\_nt & 1709 & 13711 \\ \hline
            cutted\_nt & 1709 & 6011 \\ \hline
        \end{tabular}
    \end{table}

