\subsection{Построение связей текст-текст}
    Построение связей текст-текст предполагает поиск потенциально семантически близких текстов.
    При построении связей текст-текст было использовано три способа:
    \begin{enumerate}
        \item построение связей на основе общих хэштегов,
        \item построение связей на основе общих именованных сущностей,
        \item построение связей на основе близости по времени.
    \end{enumerate}

    Связь твитов с помощью хэштегов выполнена следующим образом:
        из твитов извлекаются все хэштеги;
        затем в хэштеги превращаются все слова во всех твитах, которые совпали с ранее извлечёнными хэштегами;
        для каждого твита и для каждого хэштега извлекается $k$ твитов, которые содержат этот хэштег;
        eсли хэштег появлялся в более чем $k$ твитах, то берём $k$ твитов наиболее близких во времени к исходному;

    Связь твитов с помощью именованных сущностей выполнена следующим образом:
        сначала к краткому изложению новостей применяются методы извлечения именованных сущностей,
        затем для каждого твита, содержащего именованную сущность, извлекается $k$ твитов, которые содержат эту же именованную сущность,
        и наконец если именованная сущность содержалась более чем в $k$ твитах, то берётся $k$ твитов наиболее близких во времени к исходному;

    Связь твитов и новостей на основе близости по времени производилась наиболее просто:
        для каждого твита~(новости) взяты $k$ связей с наиболее схожими твитами~(новостями) в окрестности 24 часов.

    Построение связей текст-текст было реализовано для $k=10$. С целью ограничения числа ошибочных связей использовался набор эвристических ограничений:
    \begin{enumerate}
        \item слишком популярные хэштеги удаляются~(слишком популярным считаем хэштег, который встретился более чем в 10 твитах из обучающей выборки);
        \item твиты считаются связанными, если содержат не менее 2 общих хэштегов или именованных сущностей;
        \item связь не устанавливается если тексты слишком похожи (если косинусная мера близости текстов больше 0.99);
        \item в случае установления связей на основе времени публикации и схожести текстов, слишком не похожие тексты отбрасываются
        (слишком не похожие тексты это тексты с мерой близости меньше чем 0.3).
    \end{enumerate}

