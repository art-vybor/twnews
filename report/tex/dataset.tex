

\section{Формирование датасетов}
    Для формирования датасетов нужно сделать:
    \begin{enumerate}
        \item Выбрать источники данных.
        \item Выбрать формат хранения.
        \item Выбрать БД, для записи результата.
        \item Написать отказоустойчивый консьюмер.
        \item В течение длительного времени собрать данные.
    \end{enumerate}

    Предлагаю выбор формата хранения и БД отложить до тех пор, пока не будут получены все необходимые данные. А пока коллекционировать все возможные данные в текстовых файлах.

    В качестве источников данных 

    \subsection{Консьюмер твиттера}
        Лог работы в raw формате :) будет много копипасты с используемых мануалов

        \begin{enumerate}
            \item Step 1: Getting Twitter API keys

                In order to access Twitter Streaming API, we need to get 4 pieces of information from Twitter: API key, API secret, Access token and Access token secret. Follow the steps below to get all 4 elements:

                Create a twitter account if you do not already have one.
                Go to https://apps.twitter.com/ and log in with your twitter credentials.
                Click "Create New App"
                Fill out the form, agree to the terms, and click "Create your Twitter application"
                In the next page, click on "API keys" tab, and copy your "API key" and "API secret".
                Scroll down and click "Create my access token", and copy your "Access token" and "Access token secret".
            \item 
        \end{enumerate}

    \subsection{Поиск подходящего источника новостей}
        Пока только письмо, потом оформлю

        Провёл небольшое исследование, получил достаточно странные результаты.

        Суть исследования была в выявлении наиболее популярных новостных источников среди русскоязычных твитов.
        Был написан код, который умеет находит финальную страничку среди сервисов сокращателей ссылок (в твиттере все ссылки сокращаются).
        Всего твиттов, в анализируемой выборке: 35704
        Ссылок в анализируемой выборке: 13670
        Уникальных ссылок: 12510

        Топ20 результатов по частоте в формате кортежей: домен, количество ссылок ведущих на домен, процент от общего числа ссылок. В скобочках некоторые предположения о том, что конкретная ссылка означает.
        \begin{verbatim}
        twitter.com 3521 25.76%%                      (внутренний контент твиттера, в основном картинки)
        www.facebook.com 1418 10.37%%        (ссылки на facebook, преимущественно не на посты, а на приложения, видно приложения автоматически порождают оповещалки в твиттер)
        t.co 405 2.96%%                                      (наверняка это погрешности моего метода, ссылки которые мой код не смог автоматически раскрыть, уточню позднее)
        www.youtube.com 315 2.30%%              (видео контент)
        news.yandex.ru 239 1.75%%                  (популярный в россии новостной аггрегатор)
        su.epeak.in 214 1.57%%                         (популярный на украине новостной аггрегатор)
        www.instagram.com 198 1.45%%           (фото контент)
        www.periscope.tv 191 1.40%%               (трансляции)
        l.ask.fm 121 0.89%%                               (сервис анонимных мнений)
        lifenews.ru 109 0.80%%                          (новостной ресурс)
        ria.ru 108 0.79%%                                   (новостной ресурс)
        vk.com 93 0.68%%                                  (преимущественно ссылки на посты в вк)
        news.7crime.com 82 0.60%%                 (украинские новости)
        lenta.ru 74 0.54%%                                 (новостной ресурс)
        russian.rt.com 61 0.45%%                      (новостной ресурс)
        linkis.com 57 0.42%%                             (сервис сокращающий ссылки, тоже можно отнести к погрешности в технологии)
        www.gazeta.ru 53 0.39%%                     (новостной ресурс)
        tass.ru 43 0.31%%                                  (новостной ресурс)
        www.swarmapp.com 42 0.31%%            (какие-то игрушки на телефон)
        pi2.17bullets.com 36 0.26%%                 (возможно, погрешность метода)

        Итого в топ20 результатов от общего количества ссылок:
            Всего ссылок в топ20 : 54%%.
            Ссылок на новостные ресурсы: 3.3%% (при это самый популярный только 0.8%%)
            Ссылок на новостные аггрегаторы: 1.75%%
            Ссылок с ошибками в топ20: 2.5%%
        \end{verbatim}



        Вывод:
        Популярные новостные агенства составляют слишком малую долю в общем количестве ссылок. Даже суммарный объём крайне мал. При этом хотелось бы сосредоточиться на одном-двух источниках информации, но судя по всему не получится.

        Было решено использовать lifenews и ria новости.

        На данных за неделю было полученно:
        \begin{verbatim}
        341863 134945 115940
        TOP20 by 115940 rows
        -----------------  -----  -----
        twitter.com        36807  31.75
        apps.facebook.com   6234   5.38
        www.youtube.com     3659   3.16
        m.vk.com            2400   2.07
        www.periscope.tv    2215   1.91
        news.yandex.ru      2041   1.76
        www.instagram.com   1798   1.55
        su.epeak.in         1624   1.4
        www.facebook.com    1406   1.21
        lifenews.ru          888   0.77
        ria.ru               863   0.74
        l.ask.fm             803   0.69
        vk.com               696   0.6
        lenta.ru             647   0.56
        pi2.17bullets.com    577   0.5
        news.7crime.com      567   0.49
        russian.rt.com       564   0.49
        www.gazeta.ru        523   0.45
        linkis.com           485   0.42
        ask.fm               430   0.37
        -----------------  -----  -----
        total_percent in TOP20: 56.26
        --------------------------------------
        \end{verbatim}

        Мало

        решение
        \begin{verbatim}
        Изменения для консьюмера:
            собирать resolve_url который отдаёт twitter
            собирать больше новостных источников
            собирать английские твиты
            собирать английские новости
            собирать хештеги
            решить проблему с нераспознанным RT
        \end{verbatim}

    \subsection{Консьюмер новостей}
        lifenews - http://lifenews.ru/xml/feed.xml

        ria - http://ria.ru/export/rss2/index.xml


    \subsection{Обработка данных твиттера}
        Известные проблемы: 
        * Ретвиты обрезаются.

        Берём поля:
        * retweet - \lstinline{parsed_tweet['retweeted'] or text.startswith('RT @')}
        * text.
        * created\_at - время по гринвичу.
        * timestamp\_ms - timestamp по местному времени (в нашем случае +0300).
        * lang.

        Ретвиты скипаем, ввиду описанной выше проблемы. Надо разобраться как брать для ретвита исходный твит.

        Формируем tsv по lang:
        timestamp\_ms, text

        Пример строения типичного сообщения твиттера, получаемого с помощью Twitter Stream API.
\begin{lstlisting}
{
    "contributors": null,
    "coordinates": null,
    "created_at": "Wed Mar 09 00:24:55 +0000 2016",
    "entities": {
        "hashtags": [
            {
                "indices": [
                    61,
                    74
                ],
                "text": "OneDirection"
            },
            {
                "indices": [
                    75,
                    94
                ],
                "text": "YouKnowYouLoveThem"
            }
        ],
        "symbols": [],
        "urls": [],
        "user_mentions": [
            {
                "id": 4387486337,
                "id_str": "4387486337",
                "indices": [
                    3,
                    16
                ],
                "name": "HELP 1D",
                "screen_name": "HELPONEDVOTE"
            },
            {
                "id": 77504008,
                "id_str": "77504008",
                "indices": [
                    95,
                    107
                ],
                "name": "RADIO DISNEY",
                "screen_name": "radiodisney"
            }
        ]
    },
    "favorite_count": 0,
    "favorited": false,
    "filter_level": "low",
    "geo": null,
    "id": 707361488508469248,
    "id_str": "707361488508469248",
    "in_reply_to_screen_name": null,
    "in_reply_to_status_id": null,
    "in_reply_to_status_id_str": null,
    "in_reply_to_user_id": null,
    "in_reply_to_user_id_str": null,
    "is_quote_status": false,
    "lang": "pt",
    "place": null,
    "retweet_count": 0,
    "retweeted": false,
    "retweeted_status": {
        "contributors": null,
        "coordinates": null,
        "created_at": "Tue Mar 08 21:38:42 +0000 2016",
        "entities": {
            "hashtags": [
                {
                    "indices": [
                        43,
                        56
                    ],
                    "text": "OneDirection"
                },
                {
                    "indices": [
                        57,
                        76
                    ],
                    "text": "YouKnowYouLoveThem"
                }
            ],
            "symbols": [],
            "urls": [],
            "user_mentions": [
                {
                    "id": 77504008,
                    "id_str": "77504008",
                    "indices": [
                        77,
                        89
                    ],
                    "name": "RADIO DISNEY",
                    "screen_name": "radiodisney"
                }
            ]
        },
        "favorite_count": 13,
        "favorited": false,
        "filter_level": "low",
        "geo": null,
        "id": 707319658517549057,
        "id_str": "707319658517549057",
        "in_reply_to_screen_name": null,
        "in_reply_to_status_id": null,
        "in_reply_to_status_id_str": null,
        "in_reply_to_user_id": null,
        "in_reply_to_user_id_str": null,
        "is_quote_status": false,
        "lang": "pt",
        "place": {
            "attributes": {},
            "bounding_box": {
                "coordinates": [
                    [
                        [
                            -44.062789,
                            -20.059816
                        ],
                        [
                            -44.062789,
                            -19.777568
                        ],
                        [
                            -43.856856,
                            -19.777568
                        ],
                        [
                            -43.856856,
                            -20.059816
                        ]
                    ]
                ],
                "type": "Polygon"
            },
            "country": "Brasil",
            "country_code": "BR",
            "full_name": "Belo Horizonte, Brasil",
            "id": "d9d978b087a92583",
            "name": "Belo Horizonte",
            "place_type": "city",
            "url": "https://api.twitter.com/1.1/geo/id/d9d978b087a92583.json"
        },
        "retweet_count": 82,
        "retweeted": false,
        "source": "<a href=\"http://twitter.com\" rel=\"nofollow\">Twitter Web Client</a>",
        "text": "Eeh a tag n\u00e3o subiu em\nFAMILY ONED\n- Maria\n#OneDirection #YouKnowYouLoveThem @radiodisney",
        "truncated": false,
        "user": {
            "contributors_enabled": false,
            "created_at": "Sat Dec 05 22:28:59 +0000 2015",
            "default_profile": false,
            "default_profile_image": false,
            "description": "Projeto feito na inten\u00e7\u00e3o de ajudar os meninos nas vota\u00e7\u00f5es. Ative as notifica\u00e7\u00f5es e participe de mutir\u00f5es. Adms:Anny, Cah, Maria, Mary, Kaah, Biiah, Mari.",
            "favourites_count": 4013,
            "follow_request_sent": null,
            "followers_count": 5901,
            "following": null,
            "friends_count": 5866,
            "geo_enabled": true,
            "id": 4387486337,
            "id_str": "4387486337",
            "is_translator": false,
            "lang": "pt",
            "listed_count": 3,
            "location": "SNAP : PROJETOHELP",
            "name": "HELP 1D",
            "notifications": null,
            "profile_background_color": "000000",
            "profile_background_image_url": "http://abs.twimg.com/images/themes/theme1/bg.png",
            "profile_background_image_url_https": "https://abs.twimg.com/images/themes/theme1/bg.png",
            "profile_background_tile": false,
            "profile_banner_url": "https://pbs.twimg.com/profile_banners/4387486337/1457296538",
            "profile_image_url": "http://pbs.twimg.com/profile_images/706651250323025923/Csjoq0NA_normal.jpg",
            "profile_image_url_https": "https://pbs.twimg.com/profile_images/706651250323025923/Csjoq0NA_normal.jpg",
            "profile_link_color": "FF691F",
            "profile_sidebar_border_color": "000000",
            "profile_sidebar_fill_color": "000000",
            "profile_text_color": "000000",
            "profile_use_background_image": false,
            "protected": false,
            "screen_name": "HELPONEDVOTE",
            "statuses_count": 8533,
            "time_zone": null,
            "url": null,
            "utc_offset": null,
            "verified": false
        }
    },
    "source": "<a href=\"http://twitter.com/download/android\" rel=\"nofollow\">Twitter for Android</a>",
    "text": "RT @HELPONEDVOTE: Eeh a tag n\u00e3o subiu em\nFAMILY ONED\n- Maria\n#OneDirection #YouKnowYouLoveThem @radiodisney",
    "timestamp_ms": "1457483095658",
    "truncated": false,
    "user": {
        "contributors_enabled": false,
        "created_at": "Tue Feb 02 18:00:32 +0000 2016",
        "default_profile": true,
        "default_profile_image": false,
        "description": "ACESSE NOT\u00cdCIA FOTOS E V\u00cdDEOS SEBRE ONE DIRECTION NO BRASIL",
        "favourites_count": 117,
        "follow_request_sent": null,
        "followers_count": 30,
        "following": null,
        "friends_count": 35,
        "geo_enabled": false,
        "id": 4872198435,
        "id_str": "4872198435",
        "is_translator": false,
        "lang": "pt",
        "listed_count": 0,
        "location": "Brasil",
        "name": "ACESSO 1D",
        "notifications": null,
        "profile_background_color": "F5F8FA",
        "profile_background_image_url": "",
        "profile_background_image_url_https": "",
        "profile_background_tile": false,
        "profile_banner_url": "https://pbs.twimg.com/profile_banners/4872198435/1454436907",
        "profile_image_url": "http://pbs.twimg.com/profile_images/694584374961004545/G-Oh7i6P_normal.jpg",
        "profile_image_url_https": "https://pbs.twimg.com/profile_images/694584374961004545/G-Oh7i6P_normal.jpg",
        "profile_link_color": "2B7BB9",
        "profile_sidebar_border_color": "C0DEED",
        "profile_sidebar_fill_color": "DDEEF6",
        "profile_text_color": "333333",
        "profile_use_background_image": true,
        "protected": false,
        "screen_name": "acesso1DcomBR",
        "statuses_count": 1050,
        "time_zone": null,
        "url": null,
        "utc_offset": null,
        "verified": false
    }
}


\end{lstlisting}