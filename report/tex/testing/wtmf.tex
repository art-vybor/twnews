\subsection{Оптимизация качества WTMF, путём варьирования параметров}
    Оптимизация параметров модели для метода WTMF будет производится на наборе данных cutted, используя метрику МRR.
    Модель WTMF зависит от четырёх параметров: $K$, $I$, $\lambda$, $w_m$.
    Параметры $K$ и $I$ влияют на время построения модели, а параметры $\lambda$ и $w_m$ не влияют на время построения модели.

    В качестве начального приближения берутся значения параметров, которое использовали авторы работы~\cite{linking_base}, а именно:
    $K=30$, $I=3$, $\lambda=20$, $w_m=0.1$.

    Оптимизируются параметры, не влияющие на время работы алгоритма: $\lambda$ и $w_m$.
    Для этого фиксируются остальные параметры: $I=1$, $K=30$.
    Для начала находится оптимальный порядок значений начального приближения. Результаты занесены в таблицу~\ref{tabular:wtmf_test1}.

    \begin{table}[ht!]
        %\small
        \caption{Качество работы алгоритма WMTF для различных значений $\lambda$ и $w_m$ при фиксированных значениях $I=1$, $K=30$. \bigskip}
        \centering

        \label{tabular:wtmf_test1}
        \begin{tabular}{|c|c|c|c|c|c|c|} \hline
            $\lambda/w_m$ & \bf{0.001} & \bf{0.01} & \bf{0.1} & \bf{1} & \bf{10} & \bf{100} \\ \hline
            \bf{0.2} & 0.6855 & 0.6877 & 0.7482 & 0.3651 & 0.1526 & 0.1485 \\ \hline
            \bf{2} & 0.7000 & 0.7015 & 0.7173 & 0.7525 & 0.3707 & 0.1605 \\ \hline
            \bf{20} & 0.6964 & 0.7081 & 0.7149 & 0.7308 & 0.7507 & 0.3784 \\ \hline
            \bf{200} & 0.7075 & 0.6991 & 0.7010 & 0.7016 & 0.7146 & 0.7448 \\ \hline
            \bf{2000} & 0.6970 & 0.7070 & 0.6991 & 0.7114 & 0.6994 & 0.7044 \\ \hline
        \end{tabular}
    \end{table}
    Как видно из таблицы~\ref{tabular:wtmf_test1} в целом получена достаточно однородная картина для всех порядков $\lambda$ и $w_m$.
    Заметное снижение качества происходит при большом порядке $w_m$ и малом порядке $\lambda$.
    Максимальное значение метрики достигнуто при $\lambda=2$ и $w_m=1$.
    Для уточнения значения коэффициентов, производится исследование качества работы алгоритма в окрестностях максимального значения метрики.
    Результаты приведены в таблице~\ref{tabular:wtmf_test2}.

    \begin{table}[ht!]
        %\small
        \caption{Качество работы алгоритма WMTF для различных значений $\lambda$ и $w_m$ при фиксированных значениях $I=1$, $K=30$. \bigskip}
        \centering

        \label{tabular:wtmf_test2}
        \begin{tabular}{|c|c|c|c|c|c|c|} \hline
            $\lambda/w_m$ & \bf{0.9} & \bf{0.95} & \bf{1} & \bf{1.1} & \bf{1.2} \\ \hline
            \bf{1.9} & 0.7442 & 0.7451 & 0.7536 & 0.7542 & 0.7544 \\ \hline
            \bf{1.95} & 0.7447 & 0.7554 & 0.7452 & 0.7439 & 0.7504 \\ \hline
            \bf{2} & 0.7507 & 0.7528 & 0.7504 & 0.7515 & 0.7566 \\ \hline
            \bf{2.05} & 0.7413 & 0.7505 & 0.7424 & 0.7525 & 0.7479 \\ \hline
            \bf{2.1} & 0.7405 & 0.7484 & 0.7485 & 0.7502 & 0.7501 \\ \hline
        \end{tabular}
    \end{table}
    Из таблицы~\ref{tabular:wtmf_test2} получаем оптимальные значения коэффициентов $\lambda=0.95$ и $w_m=1.95$.

    Оптимизируются параметры, влияющие на время работы алгоритма: $K$ и $I$.
    Для этого фиксируются остальные параметры: $\lambda=0.95$, $w_m=1.95$.
    Для начала находится примерное значение коэффициента $K$ и оптимальное значение $I$.
    Результаты занесены в таблицу~\ref{tabular:wtmf_test3}.

    \begin{table}[ht!]
        %\small
        \caption{Качество работы алгоритма WMTF для различных значений $K$ и $I$ при фиксированных значениях $\lambda=0.95$, $w_m=1.95$. \bigskip}
        \centering

        \label{tabular:wtmf_test3}
        \begin{tabular}{|c|c|c|c|} \hline
            $K/I$ & \bf{1} & \bf{2} & \bf{3} \\ \hline
            \bf{5} & 0.1232 & 0.1593 & 0.1838 \\ \hline
            \bf{10} & 0.3521 & 0.4102 & 0.4437 \\ \hline
            \bf{30} & 0.7426 & 0.7422 & 0.7158 \\ \hline
            \bf{60} & 0.8326 & 0.8117 & 0.7620 \\ \hline
        \end{tabular}
    \end{table}
    Как видно из таблицы~\ref{tabular:wtmf_test3} увеличение $K$ приводит к значительному улучшению качества работы алгоритма,
    увеличении $I$ приводит к улучшению качества алгоритма только при малых значениях параметра $K$, при больших значениях $K$ увеличение параметра $I$ приводит к ухудшению качества.
    Максимальное значение метрики достигнуто при $K=60$ и $I=1$.
    Для уточнения значения коэффициента $K$, производится исследование качества работы алгоритма при фиксированном значении коэффициента $I$.
    Результаты приведены в таблице~\ref{tabular:wtmf_test4}.

    \begin{table}[ht!]
        %\small
        \caption{Качество работы алгоритма WMTF для различных значений $K$ при фиксированных значениях $I=1$, $\lambda=0.95$, $w_m=1.95$. \bigskip}
        \centering

        \label{tabular:wtmf_test4}
        \begin{tabular}{|c|c|} \hline
            \bf{K}  & \bf{Значение метрики RR} \\ \hline
            \bf{10} & 0.3595 \\ \hline
            \bf{20} & 0.6460 \\ \hline
            \bf{30} & 0.7496 \\ \hline
            \bf{40} & 0.8003 \\ \hline
            \bf{50} & 0.8220 \\ \hline
            \bf{60} & 0.8424 \\ \hline
            \bf{70} & 0.8472 \\ \hline
            \bf{80} & 0.8535 \\ \hline
            \bf{82} & 0.8549 \\ \hline
            \bf{84} & 0.8597 \\ \hline
            \bf{86} & 0.8592 \\ \hline
            \bf{88} & 0.8572 \\ \hline
            \bf{90} & 0.8675 \\ \hline
            \bf{92} & 0.8580 \\ \hline
            \bf{94} & 0.8604 \\ \hline
            \bf{96} & 0.8612 \\ \hline
            \bf{98} & 0.8644 \\ \hline
            \bf{100} & 0.8655 \\ \hline
            \bf{110} & 0.8627 \\ \hline
        \end{tabular}
    \end{table}
    Из таблицы~\ref{tabular:wtmf_test4} получаем оптимальные значения коэффициента $K=90$

    В итоге оптимизации качества рекомендаций на основе алгоритма WMTF были получены оптимальные параметры:
    $K=90$, $I=1$, $\lambda=0.95$, $w_m=1.95$.

    \clearpage