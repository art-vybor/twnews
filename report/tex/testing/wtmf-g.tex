\subsection{Оптимизация качества WTMF-G, путём варьирования параметров}

    Датасет cutted\_0.0

    Начальное приближение $\delta=0.1, w_m=1.95, K=90, I=1, \lambda=0.95$.

    \begin{table}[ht!]
        %\small
        \caption{Качество работы алгоритма WMTF-G для различных значений $\lambda$ и $\delta$ при фиксированных значениях $I=1$, $K=30$, $w_m=1.95$. \bigskip}
        \centering

        \label{tabular:wtmfg_test1}
        \begin{tabular}{|c|c|c|c|c|c|} \hline
            $\lambda \backslash \delta$ & \bf{0.001} & \bf{0.01} & \bf{0.1} & \bf{1} & \bf{10}  \\ \hline
            \bf{0.01} & 0.3889 & 0.3842 & 0.3924 & 0.3900 & 0.3895 \\ \hline
            \bf{0.1}  & 0.4895 & 0.4875 & 0.4886 & 0.4850 & 0.4847  \\ \hline
            \bf{1}    & 0.8227 & 0.8256 & 0.8242 & 0.8225 & 0.8212 \\ \hline
            \bf{10}   & 0.8477 & 0.8440 & 0.8496 & 0.8454 & 0.8495 \\ \hline
            \bf{100}  & 0.8294 & 0.8318 & 0.8283 & 0.8240  & 0.8243 \\ \hline
        \end{tabular}
    \end{table}

    максимум при $\lambda=10$, $\delta=0.1$  рассмотрим окрестности.

    \begin{table}[ht!]
        %\small
        \caption{Качество работы алгоритма WMTF-G для различных значений $\lambda$ и $\delta$ при фиксированных значениях $I=1$, $K=30$, $w_m=1.95$. \bigskip}
        \centering

        \label{tabular:wtmfg_test1}
        \begin{tabular}{|c|c|c|c|c|c|} \hline
            $\lambda \backslash \delta$ & \bf{0.06} & \bf{0.08} & \bf{0.1} & \bf{0.12} & \bf{0.14}  \\ \hline
            \bf{6}  & 0.8589 & 0.8524 & 0.8511 & 0.8580 & 0.8493 \\ \hline
            \bf{8}  & 0.8483 & 0.8528 & 0.8539 & 0.8439 & 0.8498  \\ \hline
            \bf{10} & 0.8504 & 0.8455 & 0.8416 & 0.8453 & 0.8408  \\ \hline
            \bf{12} & 0.8453 & 0.8398 & 0.8472 & 0.8376 & 0.8415   \\ \hline
            \bf{14} & 0.8462 & 0.8456 & 0.8387 & 0.8398 & 0.8377 \\ \hline
        \end{tabular}
    \end{table}

    рассмотрим максимум при $\lambda=6$, $\delta=0.06$

    вариьируем $w_m$

    0.01 0.8283
    0.05 0.8296
    0.1 0.8285
    0.5 0.8359
    1 0.8442
    5 0.8639
    10 0.8391
    50 0.6094
    100 0.5035

    1.5 0.8474
    2.0 0.8507
    2.5 0.8563
    3.0 0.8592
    3.5 0.8585
    4.0 0.8594
    4.5 0.8603
    5.0 0.8597
    5.5 0.8591
    6.0 0.8586
    6.5 0.8574
    7.5 0.8536

    берём w\_m=5, варьируем K/I

    \begin{table}[ht!]
        %\small
        \caption{Качество работы алгоритма WMTF-G для различных значений $K$ и $I$ при фиксированных значениях $w\_m=5$, $\lambda=6$, $\delta=0.06$. \bigskip}
        \centering

            for delta in [0.06, 0.08, 0.1, 0.12, 0.14]:
                for lmbd in [6, 8, 10, 12, 14]:
        \label{tabular:wtmfg_test1}
        \begin{tabular}{|c|c|c|c|c|c|} \hline
            $K \backslash I$ & 1 & 2 & 3 & 4 & 5  \\ \hline
            30 & 0.7529 & 0.7927 & 0.7577 & 0.6736 & 0.5794  \\ \hline
            40 & 0.7992 & 0.8194 & 0.7695 & 0.6813 & 0.5828  \\ \hline
            50 & 0.8269 & 0.8349 & 0.7834 & 0.6830 & 0.5801  \\ \hline
            60 & 0.8419 & 0.8450 & 0.7984 & 0.7056 & 0.6006  \\ \hline
            70 & 0.8557 & 0.8466 & 0.7977 & 0.7036 & 0.6002  \\ \hline
            80 & 0.8614 & 0.8511 & 0.7990 & 0.7032 & 0.5957  \\ \hline
            90 & 0.8606 & 0.8522 & 0.8039 & 0.7088 & 0.6038  \\ \hline
            100 & 0.8606 & 0.8527 & 0.8022 & 0.7089 & 0.6021  \\ \hline
            110 & 0.8686 & 0.8553 & 0.8074 & 0.7123 & 0.6065  \\ \hline
            120 & 0.8693 & 0.8579 & 0.8097 & 0.7174 & 0.6085  \\ \hline
            130 & 0.8725 & 0.8588 & 0.8160 & 0.7264 & 0.6206  \\ \hline
            140 & 0.8740 & 0.8597 & 0.8157 & 0.7248 & 0.6241  \\ \hline
            150 & 0.8763 & 0.8620 & 0.8171 & 0.7263 & 0.6200  \\ \hline

        \end{tabular}
    \end{table}





%результаты на cutted_0.0 вариации Lambda delta
%0.01 0.001 0.3889
%0.01 0.01 0.3842
%0.01 0.1 0.3924
%0.01 1 0.3900
%0.01 10 0.3895
%0.1 0.001 0.4895
%0.1 0.01 0.4875
%0.1 0.1 0.4886
%0.1 1 0.4850
%0.1 10 0.4847
%100 0.001 0.8294
%100 0.01 0.8318
%100 0.1 0.8283
%100 1 0.8240
%100 10 0.8243
%10 0.001 0.8477
%10 0.01 0.8440
%10 0.1 0.8496
%10 1 0.8454
%10 10 0.8495
%1 0.001 0.8227
%1 0.01 0.8256
%1 0.1 0.8242
%1 1 0.8225
%1 10 0.8212
%
%10 0.06 0.8504
%10 0.08 0.8455
%10 0.1 0.8416
%10 0.12 0.8453
%10 0.14 0.8408
%12 0.06 0.8435
%12 0.08 0.8398
%12 0.1 0.8472
%12 0.12 0.8376
%12 0.14 0.8415
%14 0.06 0.8462
%14 0.08 0.8456
%14 0.1 0.8387
%14 0.12 0.8398
%14 0.14 0.8377
%6 0.06 0.8589
%6 0.08 0.8524
%6 0.1 0.8511
%6 0.12 0.8580
%6 0.14 0.8493
%8 0.06 0.8483
%8 0.08 0.8528
%8 0.1 0.8539
%8 0.12 0.8439
%8 0.14 0.8498



% результаты на cutted_0.5 вариации на тему lambda delta
%10 0.06 0.6434
%10 0.08 0.6408
%10 0.1 0.6542
%10 0.12 0.6662
%10 0.14 0.6335
%12 0.06 0.6381
%12 0.08 0.6523
%12 0.1 0.6424
%12 0.12 0.6312
%12 0.14 0.6625
%14 0.06 0.6392
%14 0.08 0.6417
%14 0.1 0.6504
%14 0.12 0.6606
%14 0.14 0.6294
%6 0.06 0.6523
%6 0.08 0.6447
%6 0.1 0.6560
%6 0.12 0.6511
%6 0.14 0.6430
%8 0.06 0.6427
%8 0.08 0.6455
%8 0.1 0.6457
%8 0.12 0.6378
%8 0.14 0.6703




    \textcolor{red}{Оптимизация параметров ещё не завершена, существующая и очень, очень грубая оценка приведена ниже}

    Оптимизация параметров модели для метода WTMF-G будет производится на наборе данных auto\_cleared, используя метрику МRR.
    Модель WTMF зависит от четырёх параметров: $K$, $I$, $\delta$, $w_m$.
    Параметры $K$ и $I$ влияют на время построения модели, а параметры $\lambda$ и $w_m$ не влияют на время построения модели.

    В качестве начального приближения параметров взяты оптимальные параметры для метода WTMF, а именно $K=90$, $I=1$, $w_m=1.95$.
    В качестве начального приближения параметра $\delta$ вы берем значение 0.1

    Оптимизируется параметр $\delta$. Для этого фиксируются остальные параметры: $K=90$, $I=1$, $w_m=1.95$.
    Для начала находится оптимальный порядок значений начального приближения. Результаты занесены в таблицу~\ref{tabular:wtmfg_test1}.
    \begin{table}[ht!]
        %\small
        \caption{Качество работы алгоритма WMTF-G для различных значений $\delta$ при фиксированных значениях $K=90$, $I=1$, $w_m=1.95$. \bigskip}
        \centering

        \label{tabular:wtmfg_test1}
        \begin{tabular}{|c|c|} \hline
            $\delta$ & \bf{Значение метрики RR} \\ \hline
            \bf{0.001} & 0.5508 \\ \hline
            \bf{0.01} & 0.5307 \\ \hline
            \bf{0.1} & 0.5695 \\ \hline
            \bf{1} & 0.5311 \\ \hline
            \bf{10} & 0.5303 \\ \hline
            \bf{100} & 0.5203 \\ \hline
        \end{tabular}
    \end{table}
    Как видно из таблицы~\ref{tabular:wtmfg_test1} максимальное значение метрики получено при $\delta=0.1$.
    Для уточнения значения коэффициента $\delta$, производится исследование качества работы алгоритма в окрестностях максимального значения метрики.
    Результаты приведены в таблице~\ref{tabular:wtmfg_test2}.
    \begin{table}[ht!]
        %\small
        \caption{Качество работы алгоритма WMTF-G для различных значений $\delta$ при фиксированных значениях $K=90$, $I=1$, $w_m=1.95$. \bigskip}
        \centering

        \label{tabular:wtmfg_test2}
        \begin{tabular}{|c|c|} \hline
            $\delta$ & \bf{Значение метрики RR} \\ \hline
            \bf{0.05} & 0.5340 \\ \hline
            \bf{0.1} & 0.5695 \\ \hline
            \bf{0.15} & 0.5380 \\ \hline
            \bf{0.25} & 0.5533 \\ \hline
            \bf{0.3} & 0.5195 \\ \hline
            \bf{0.35} & 0.5329 \\ \hline
        \end{tabular}
    \end{table}
    Из таблицы~\ref{tabular:wtmfg_test2} получаем оптимальные значения коэффициента $\delta=0.1$.

    В итоге оптимизации качества рекомендаций на основе алгоритма WMTF-G были получены оптимальные параметры:
    $K=90$, $I=1$, $\delta=0.1$, $w_m=1.95$.