\subsection{Оптимизация качества WTMF-G, путём варьирования параметров}
    Оптимизация параметров модели для метода WTMF-G будет производится на наборе данных cutted, используя метрику МRR.
    Модель WTMF-G зависит от пяти параметров: $K$, $I$, $\lambda$, $\delta$, $w_m$.
    Параметры $K$ и $I$ влияют на время построения модели, а параметры $\lambda$ и $w_m$ не влияют на время построения модели.

    В качестве начального приближения параметров взяты оптимальные параметры для метода WTMF, а именно $K=90$, $I=1$, $w_m=1.95$,$\lambda=0.95$. В качестве начального приближения параметра $\delta$ берем значение 0.1.

    Сначала оптимизируем параметры, влияющие на регуляризующий член: $\lambda$ и $\delta$,
    Для этого фиксируем остальные параметры: $I=1$, $K=90$, $w_m=1.95$.
    Сначала найдём оптимальный порядок значений начального приближения. Результаты занесены в таблицу~\ref{tabular:wtmfg_test1}.

    \begin{table}[ht!]
        %\small
        \caption{Качество работы алгоритма WMTF-G для различных значений $\lambda$ и $\delta$ при фиксированных значениях $I=1$, $K=90$, $w_m=1.95$. \bigskip}
        \centering

        \label{tabular:wtmfg_test1}
        \begin{tabular}{|c|c|c|c|c|c|} \hline
            $\lambda \backslash \delta$ & \bf{0.001} & \bf{0.01} & \bf{0.1} & \bf{1} & \bf{10}  \\ \hline
            \bf{0.01} & 0.3889 & 0.3842 & 0.3924 & 0.3900 & 0.3895 \\ \hline
            \bf{0.1}  & 0.4895 & 0.4875 & 0.4886 & 0.4850 & 0.4847  \\ \hline
            \bf{1}    & 0.8227 & 0.8256 & 0.8242 & 0.8225 & 0.8212 \\ \hline
            \bf{10}   & 0.8477 & 0.8440 & 0.8496 & 0.8454 & 0.8495 \\ \hline
            \bf{100}  & 0.8294 & 0.8318 & 0.8283 & 0.8240  & 0.8243 \\ \hline
        \end{tabular}
    \end{table}

    Как видно из таблицы~\ref{tabular:wtmfg_test1} порядок параметра $\delta$ оказывает влияние на качество, но достаточно слабое, порядок параметра $\lambda$, напротив очень сильно влияет на получаемое качество.
    Максимальное значение метрики достигнуто при $\lambda=10$ и $\delta=0.1$.
    Для уточнения значения коэффициентов, производится исследование качества работы алгоритма в окрестностях максимального значения метрики.
    Результаты приведены в таблице~\ref{tabular:wtmfg_test2}.

    \begin{table}[ht!]
        %\small
        \caption{Качество работы алгоритма WMTF-G для различных значений $\lambda$ и $\delta$ при фиксированных значениях $I=1$, $K=90$, $w_m=1.95$. \bigskip}
        \centering

        \label{tabular:wtmfg_test2}
        \begin{tabular}{|c|c|c|c|c|c|} \hline
            $\lambda \backslash \delta$ & \bf{0.06} & \bf{0.08} & \bf{0.1} & \bf{0.12} & \bf{0.14}  \\ \hline
            \bf{4}  & 0.8501 & 0.8512 & 0.8489 & 0.8530 & 0.8476  \\ \hline
            \bf{6}  & 0.8589 & 0.8524 & 0.8511 & 0.8580 & 0.8493  \\ \hline
            \bf{8}  & 0.8483 & 0.8528 & 0.8539 & 0.8439 & 0.8498  \\ \hline
            \bf{10} & 0.8504 & 0.8455 & 0.8416 & 0.8453 & 0.8408  \\ \hline
            \bf{12} & 0.8453 & 0.8398 & 0.8472 & 0.8376 & 0.8415  \\ \hline
            \bf{14} & 0.8462 & 0.8456 & 0.8387 & 0.8398 & 0.8377  \\ \hline
        \end{tabular}
    \end{table}

    В таблице~\ref{tabular:wtmfg_test2} получена достаточно однородная картина. Возьмём в качестве оптимального значения коэффициент полученную точку максимум: $\lambda=6$, $\delta=0.06$.

    Рассмотрим влияние параметра $w_m$ и найдём его оптимальное значение. Сначала рассмотрим качество алгоритма для различных порядков $w_m$.
    Результаты занесены в таблицу~\ref{tabular:wtmfg_test3}

    \begin{table}[ht!]
        %\small
        \caption{Качество работы алгоритма WMTF-G для различных значений $w_m$ при фиксированных значениях $I=1$, $K=90$, $\lambda=6$, $\delta=0.6$. \bigskip}
        \centering

        \label{tabular:wtmfg_test3}
        \begin{tabular}{|c|c|c|c|c|c|c|c|c|c|} \hline
            $w_m$ & 0.01 & 0.05 & 0.1 & 0.5 & 1 & 5 & 10 & 50 & 100 \\ \hline
            \bf{MRR} & 0.8283 & 0.8296 & 0.8285 & 0.8359 & 0.8442 & 0.8639 & 0.8391 & 0.6094 & 0.5035 \\ \hline

        \end{tabular}
    \end{table}

    Как видно из таблицы~\ref{tabular:wtmfg_test3} порядок параметра $w_m$ оказывает заметное влияние на качество. При значительном увеличении до $10^2$ качество начинает резко падать.
    Максимальное значение метрики достигнуто при $w_m=5$, уточним полученное значение.
    Для уточнения значения коэффициента $w_m$, производится исследование качества работы алгоритма в окрестностях максимального значения метрики.
    Результаты приведены в таблице~\ref{tabular:wtmfg_test4}.

    \begin{table}[ht!]
        %\small
        \caption{Качество работы алгоритма WMTF-G для различных значений $w_m$ при фиксированных значениях $I=1$, $K=90$, $\lambda=6$, $\delta=0.6$. \bigskip}
        \centering

        \label{tabular:wtmfg_test4}
        \begin{tabular}{|c|c|c|c|c|c|c|c|c|c|} \hline
            $w_m$ & 1.5 & 2.0 & 2.5 & 3.0 & 3.5 & 4.0 & 4.5 & 5.0 & 5.5 & 6.0 & 6.5 & 7.5 \\ \hline
            \bf{MRR} & 0.8474 & 0.8507 & 0.8563 & 0.8592 & 0.8585 & 0.8594 & 0.8603 & 0.8641 & 0.8591 & 0.8586 & 0.8574 & 0.8536 \\ \hline

        \end{tabular}
    \end{table}

    Из таблицы~\ref{tabular:wtmfg_test4} получаем оптимальное значение параметра $w_m=5$.
    Теперь рассмотрим оставшиеся два параметра $K$ и $I$. Качество работы алгоритма WTMF-G для различных значений $K$ и $I$ приведено в таблице~\ref{tabular:wtmfg_test5}.

    \begin{table}[ht!]
        %\small
        \caption{Качество работы алгоритма WMTF-G для различных значений $K$ и $I$ при фиксированных значениях $w\_m=5$, $\lambda=6$, $\delta=0.06$. \bigskip}
        \centering
        \label{tabular:wtmfg_test5}
        \begin{tabular}{|c|c|c|c|c|c|} \hline
            $K \backslash I$ & 1 & 2 & 3 & 4 & 5  \\ \hline
            30 & 0.7529 & 0.7927 & 0.7577 & 0.6736 & 0.5794  \\ \hline
            40 & 0.7992 & 0.8194 & 0.7695 & 0.6813 & 0.5828  \\ \hline
            50 & 0.8269 & 0.8349 & 0.7834 & 0.6830 & 0.5801  \\ \hline
            60 & 0.8419 & 0.8450 & 0.7984 & 0.7056 & 0.6006  \\ \hline
            70 & 0.8557 & 0.8466 & 0.7977 & 0.7036 & 0.6002  \\ \hline
            80 & 0.8614 & 0.8511 & 0.7990 & 0.7032 & 0.5957  \\ \hline
            90 & 0.8606 & 0.8522 & 0.8039 & 0.7088 & 0.6038  \\ \hline
            100 & 0.8606 & 0.8527 & 0.8022 & 0.7089 & 0.6021  \\ \hline
            110 & 0.8686 & 0.8553 & 0.8074 & 0.7123 & 0.6065  \\ \hline
            120 & 0.8693 & 0.8579 & 0.8097 & 0.7174 & 0.6085  \\ \hline
            130 & 0.8725 & 0.8588 & 0.8160 & 0.7264 & 0.6206  \\ \hline
            140 & 0.8740 & 0.8597 & 0.8157 & 0.7248 & 0.6241  \\ \hline
            150 & 0.8763 & 0.8620 & 0.8171 & 0.7263 & 0.6200  \\ \hline
            160 & 0.8740 & 0.8596 & - & - & - \\ \hline
            170 & 0.8768 & 0.8606 & - & - & -  \\ \hline
            180 & 0.8785 & 0.8613 & - & - & -  \\ \hline
            190 & 0.8767 & 0.8616 & - & - & -  \\ \hline
            200 & 0.8769 & 0.8613 & - & - & -  \\ \hline
            210 & 0.8786 & 0.8613 & - & - & -  \\ \hline
            220 & 0.8816 & 0.8632 & - & - & -  \\ \hline
            230 & 0.8814 & 0.8646 & - & - & -  \\ \hline
            240 & 0.8758 & 0.8632 & - & - & -  \\ \hline
        \end{tabular}
    \end{table}

    Как показано в таблице~\ref{tabular:wtmfg_test5}, WTMF-G показывает аналогично методу WTMF поведение при изменении параметров $K$ и $I$, а именно, в среднем с увеличением количества итераций, качество работы алгоритма уменьшается.
    Максимум был достигнут при $K=220$, $I=1$.

    В итоге оптимизации качества рекомендаций на основе алгоритма WMTF-G были получены оптимальные параметры:
    $K=220$, $I=1$, $\delta=0.6$, $\lambda=6$, $w_m=5$.