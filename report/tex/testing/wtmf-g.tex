\subsection{Оптимизация качества WTMF-G, путём варьирования параметров}
    \textcolor{red}{Оптимизация параметров ещё не завершена, существующая и очень, очень грубая оценка приведена ниже}

    Оптимизация параметров модели для метода WTMF-G будет производится на наборе данных auto\_cleared, используя метрику МRR.
    Модель WTMF зависит от четырёх параметров: $K$, $I$, $\delta$, $w_m$.
    Параметры $K$ и $I$ влияют на время построения модели, а параметры $\lambda$ и $w_m$ не влияют на время построения модели.

    В качестве начального приближения параметров взяты оптимальные параметры для метода WTMF, а именно $K=90$, $I=1$, $w_m=1.95$.
    В качестве начального приближения параметра $\delta$ вы берем значение 0.1

    Оптимизируется параметр $\delta$. Для этого фиксируются остальные параметры: $K=90$, $I=1$, $w_m=1.95$.
    Для начала находится оптимальный порядок значений начального приближения. Результаты занесены в таблицу~\ref{tabular:wtmfg_test1}.
    \begin{table}[ht!]
        %\small
        \caption{Качество работы алгоритма WMTF-G для различных значений $\delta$ при фиксированных значениях $K=90$, $I=1$, $w_m=1.95$. \bigskip}
        \centering

        \label{tabular:wtmfg_test1}
        \begin{tabular}{|c|c|} \hline
            $\delta$ & \bf{Значение метрики RR} \\ \hline
            \bf{0.001} & 0.5508 \\ \hline
            \bf{0.01} & 0.5307 \\ \hline
            \bf{0.1} & 0.5695 \\ \hline
            \bf{1} & 0.5311 \\ \hline
            \bf{10} & 0.5303 \\ \hline
            \bf{100} & 0.5203 \\ \hline
        \end{tabular}
    \end{table}
    Как видно из таблицы~\ref{tabular:wtmfg_test1} максимальное значение метрики получено при $\delta=0.1$.
    Для уточнения значения коэффициента $\delta$, производится исследование качества работы алгоритма в окрестностях максимального значения метрики.
    Результаты приведены в таблице~\ref{tabular:wtmfg_test2}.
    \begin{table}[ht!]
        %\small
        \caption{Качество работы алгоритма WMTF-G для различных значений $\delta$ при фиксированных значениях $K=90$, $I=1$, $w_m=1.95$. \bigskip}
        \centering

        \label{tabular:wtmfg_test2}
        \begin{tabular}{|c|c|} \hline
            $\delta$ & \bf{Значение метрики RR} \\ \hline
            \bf{0.05} & 0.5340 \\ \hline
            \bf{0.1} & 0.5695 \\ \hline
            \bf{0.15} & 0.5380 \\ \hline
            \bf{0.25} & 0.5533 \\ \hline
            \bf{0.3} & 0.5195 \\ \hline
            \bf{0.35} & 0.5329 \\ \hline
        \end{tabular}
    \end{table}
    Из таблицы~\ref{tabular:wtmfg_test2} получаем оптимальные значения коэффициента $\delta=0.1$.

    В итоге оптимизации качества рекомендаций на основе алгоритма WMTF-G были получены оптимальные параметры:
    $K=90$, $I=1$, $\delta=0.1$, $w_m=1.95$.