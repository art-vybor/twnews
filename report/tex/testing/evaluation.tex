\subsection{Методы оценки качества}
    Для оценки качества рассматриваются метрики применимые для решения задач информационного поиска~\cite{steck_recommender}.
    Твит рассматривается как запрос, а список новостей как ответ. 
    Для каждого твита, получаемый список новостей ранжирован по мере убывания их схожести.
    В работе использованы две метрики: $MRR$ и $TOP_I$, их описание дано ниже.

    \subsubsection{Метрика качества $MRR$}
    \label{subsubsec:MRR}
        $MRR$~(от англ. Mean reciprocal rank)~---~статистическая метрика, используемая для измерения качества алгоритмов информационного поиска.
        Пусть $rank_i$~---~позиция первого правильного ответа в $i$-м запросе, $n$~---~общее количество запросов.
        Тогда значение $MRR$ можно получить по формуле:
        \begin{equation}
            MRR = \dfrac{1}{n} \sum_{i=1}^n \dfrac{1}{rank_i}.
        \end{equation}

%        Пример расчёта
    \subsubsection{Метрика качества $TOP_I$}
        $TOP_I$~---~группа метрик, используемых для оценки качества алгоритмов информационного поиска. Значение метрики $TOP_I$
        численно равно проценту запросов с правильным ответом, входящим в первые $I$ ответов.
        Пусть $n$~---~общее количество запросов, $Q_I(i)$~---~равно 1, если правильный ответ на $i$-й запрос входит в первые $I$ предложенных ответов, 0~---~в противном случае.
        Тогда значение $TOP_I$ можно получить по формуле:
        \begin{equation}
            TOP_I=\dfrac{1} {n} \sum_{i=1}^n Q_I(i).
        \end{equation}
        В дальшейшем будут рассматриваться две метрики из группы метрик $TOP_I$: $TOP_1$, $TOP_3$.

%        Пример расчёта

%    \subsubsection{ATOP}
%        описание метрики и обоснование почему она не очень
%
%        Используем метрику $ATOP$ (метрика подробно описана в \cite{steck_recommender}).
%        Рассмотрим что означает эта метрика в применении к нашей задаче (я немного модифицировал метрику, для более простого описания, полученная метрика полностью совпадает с описанной метрикой).
%        Пусть $T$ - это множество твитов, $N \in \mathbb{N}$ - размер рассматриваемого топа новостей для твита (могут быть все новости вообще), $k < N \in \mathbb{N}$.
%        $TOPK_t(k) = 1$, если твит $t \in T$ соответствует хотя бы одной новости в top-k результатов, иначе $TOPK_t(k) = 0$
%        $$TOPK(k) = \dfrac {\sum_{t \in T} TOPK_t(k)} {|T|},$$
%        $$ATOP = \dfrac{\sum_{k=1}^N TOPK_t(k)}{N} = \dfrac{1}{|T| * N} \sum_{k=\overline{1,N}, ~t \in T} TOPK_t(k).$$
%
%        Значения метрики $ATOP$ лежат на отрезке $[0,1]$. Чем ближе $ATOP$ к $1$ тем лучше.