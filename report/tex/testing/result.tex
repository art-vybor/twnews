\subsection{Сравнительные результаты}
    Для выявления влияния добавления связей текст-текст на результаты работы метода WMTF-G производится сравнительное тестирование алгоритма WTMF и WTMF-G. Тестирование производится для различных наборов данных.
    Результаты тестирования приведены в таблице~\ref{tabular:wtmf_wmtfg}.

    \begin{table}[h!]
    %\small
    \caption{Cравнительное тестирование алгоритмов WTMF и WTMF-G. \bigskip}
    \centering

    \label{tabular:wtmf_wmtfg}
        \begin{tabular}{|c|c|c|c|c|c|c|}
            \hline
            \bf{\multirow{2}{*}{\specialcell{Набор данных}}} &
            \multicolumn{2}{|c|}{\bf{Метрика MRR}} &
            \multicolumn{2}{|c|}{\bf{Метрика $TOP_1$}} &
            \multicolumn{2}{|c|}{\bf{Метрика $TOP_3$}} \\ \cline{2-7}
            & \bf{WTMF} & \bf{WTMF-G} & \bf{WTMF} & \bf{WTMF-G} & \bf{WTMF} & \bf{WTMF-G} \\ \hline
            manual & 0.7293 & 0. & 0. & 0. & 0. & 0. \\ \hline
            auto & 0.8640 & 0. & 0. & 0. & 0. & 0. \\ \hline
            total & 0.8196 & 0. & 0. & 0. & 0. & 0. \\ \hline
            cutted & 0.8630 & 0.8816 & 0. & 0. & 0. & 0. \\ \hline
            manual\_nt & 0.6194 & 0. & 0. & 0. & 0. & 0. \\ \hline
            auto\_nt & 0.5297 & 0.5695 & 0. & 0. & 0. & 0. \\ \hline
            total\_nt & 0.5729 & 0. & 0. & 0. & 0. & 0. \\ \hline
            cutted\_nt & 0.6495 & 0. & 0. & 0. & 0. & 0. \\ \hline
        \end{tabular}
    \end{table}

    Как видно из таблицы~\ref{tabular:wtmf_wmtfg} алгоритм WMTF-G показывает стабильно более высокий результат чем алгоритм WMTF, из этого можно сделать вывод, что добавление связей текст-текст позволяет построить более точные рекомендации.

    Сравним два метода рекомендаций: TF-IDF и WTMF-G. Сначала посмотрим на результаты полученные на 
    базовых эталонных наборах данных, то есть тех, которые наряду с нетривиальными содержат большое количество тривиальных связей.
    Результаты тестирования приведены в таблице~\ref{tabular:tfidf_wmtfg}.

    \begin{table}[ht!]
    %\small
    \caption{Cравнительное тестирование алгоритмов TF-IDF и WTMF-G на базовых эталонных наборах данных. \bigskip}
    \centering

    \label{tabular:tfidf_wmtfg}
        \begin{tabular}{|c|c|c|c|c|c|c|}
            \hline
            \bf{\multirow{2}{*}{\specialcell{Набор данных}}} &
            \multicolumn{2}{|c|}{\bf{Метрика MRR}} &
            \multicolumn{2}{|c|}{\bf{Метрика $TOP_1$}} &
            \multicolumn{2}{|c|}{\bf{Метрика $TOP_3$}} \\ \cline{2-7}
            & \bf{TF-IDF} & \bf{WTMF-G} & \bf{TF-IDF} & \bf{WTMF-G} & \bf{TF-IDF} & \bf{WTMF-G} \\ \hline
            manual & 0.8336 & 0. & 0. & 0. & 0. & 0. \\ \hline
            auto & 0.8817 & 0. & 0. & 0. & 0. & 0. \\ \hline
            total & 0.8610 & 0. & 0. & 0. & 0. & 0. \\ \hline
            cutted & 0.9075 & 0.8816 & 0. & 0. & 0. & 0. \\ \hline
        \end{tabular}
    \end{table}

    Как видно из таблицы~\ref{tabular:tfidf_wmtfg} метод TF-IDF показывает заметно более лучший результат на всех наборах данных.
    \textcolor{red}{В cutted  результаты намного ближе, так как метод WTMF-G намного лучше работает при равном числе новостей и твитов}. 

    В целом для метода TF-IDF получены неожиданно высокие результаты, качество полученное для метода TF-IDF авторами метода WTMF-G при связывании твитов и новостей почти в два раза меньше~(качество полученное авторами метода WTMF-G приведено в разделе \ref{subsubsec:wtmfg_review}).
    Настолько высокие результаты метода TF-IDF получены по следующим причинам: во-первых, в русском твиттере очень много тривиальных связей твит-новость, во-вторых, ввиду специфики языка, в заголовках новостей и твитов их описывающий оказалось большое количество общих слов.

    С целью нивелирования влияния тривиальных связей было проведено тестирование на наборах данных, которые содержат исключительно нетривиальные связи. Результаты экспериментов приведены в таблице~\ref{tabular:tfidf_wmtfg_nt}.


    \begin{table}[ht!]
    %\small
    \caption{Cравнительное тестирование алгоритмов TF-IDF и WTMF-G на наборах данных с нетривиальными твитами. \bigskip}
    \centering

    \label{tabular:tfidf_wmtfg_nt}
        \begin{tabular}{|c|c|c|c|c|c|c|}
            \hline
            \bf{\multirow{2}{*}{\specialcell{Набор данных}}} &
            \multicolumn{2}{|c|}{\bf{Метрика MRR}} &
            \multicolumn{2}{|c|}{\bf{Метрика $TOP_1$}} &
            \multicolumn{2}{|c|}{\bf{Метрика $TOP_3$}} \\ \cline{2-7}
            & \bf{TF-IDF} & \bf{WTMF-G} & \bf{TF-IDF} & \bf{WTMF-G} & \bf{TF-IDF} & \bf{WTMF-G} \\ \hline
            manual\_nt & 0.7565 & 0. & 0. & 0. & 0. & 0. \\ \hline
            auto\_nt & 0.6048 & 0.5695 & 0. & 0. & 0. & 0. \\ \hline
            total\_nt & 0.6914 & 0. & 0. & 0. & 0. & 0. \\ \hline
            cutted\_nt & 0.7485 & 0. & 0. & 0. & 0. & 0. \\ \hline
        \end{tabular}
    \end{table}
    Как видно из таблицы~\ref{tabular:tfidf_wmtfg_nt} \textcolor{red}{...}

        объяснение влияния различных датасетов, и сравнение с результатами статьи.

