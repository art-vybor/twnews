\section*{Заключение}
\addcontentsline{toc}{section}{Заключение}
    В рамках дипломной работы было проведено исследование методов установления связей между твитами и новостными статьями.
    Было собрано несколько наборов данных и проанализированы различные подходы к их разметке.
    Реализован программный комплекс, позволяющий устанавлить связи между твитами и новостными статьями в формате рекомендаций на основе различных подходов: методов WTMF, WTMF-G и метода, основанного на частотности слов~(TF-IDF).
    На основе написанного программного комплекса произведено сравнение различных подходов. Как результат сравнения получено, что в для русского сегмента твиттера наиболее оптимальным подходом является метод, основанный на частотности слов~(TF-IDF). \textcolor{red}{предложение про экономическую часть}

    К сожалению, исследование показало, что \textcolor{red}{простейший метод показал хороший результат, он не учитывает кучу моментов, также слишком маленький датасет, нужно посадить толпу индусов разметить побольше}

    \textcolor{red}{нужно проводить доп исследования в таких-то направлениях, это потенциально поможет повысить качество рекомендаций}

    