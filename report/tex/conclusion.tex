\section*{Заключение}
\addcontentsline{toc}{section}{Заключение}
    В рамках дипломной работы было проведено исследование методов установления связей между твитами и новостными статьями.
    Было собрано несколько наборов данных и проанализированы различные подходы к их разметке.
    Реализован программный комплекс, позволяющий устанавить связи между твитами и новостными статьями в формате рекомендаций на основе различных подходов: методов WTMF, WTMF-G и метода, основанного на частотности слов~(TF-IDF).
    На основе написанного программного комплекса произведено сравнение различных подходов.
    Как результат сравнения получено, что в для русского сегмента твиттера наиболее оптимальным подходом является метод, основанный на частотности слов~(TF-IDF).
    %\textcolor{red}{предложение про экономическую часть}

    К сожалению, исследование показало, что наиболее простой метод, основанный на частотности слов, проявил наилучшее качество.
    Высокое качество метода TF-IDF является серьёзной проблемой, так как из этого следует, что рассматриваемые твиты,
    оказались <<большими>> текстами, очень похожими на заголовки новостей, то есть рассматриваемые твиты не репрезентативны.
    Основными причинами вызвавшими это являются как специфика собранных наборов данных, так и особенности русскоязычного сегмента твиттер.

    Построение набора данных, состоящего из большего числа твитов, сильно отличающихся от заголовков новостей, позволит
    получить более реалистичную оценку качества. Что в свою очередь приведёт к возможности оценить преимущество методов,
    использующих дополнительные признаки, предоставляемые предметной областью.
    