\subsection{Пакет twnews\_consumer}
    Исходный код приложения twnews\_consumer расположен в папке \lstinline{consumer} в корне репозитория.
    Приложение позволяет выкачивать и сохранять твиты и новости в формате, удобном для дальнейшей работы пакета twnews.

    \subsubsection{Конфигурирование}
        Конфигурирование пакета twnews\_consumer производится в файле \lstinline{twnews_consumer/defaults.py}.
        После изменения параметров, необходимо переустановить пакет.
        Описание задаваемых параметров находится в таблице~\ref{tabular:consumer_config}.

        \begin{table}[h!]
            \small
            \caption{Описание конфигурации пакета twnews\_consumer \bigskip}
            \center

            %\begin{sideways}
            \label{tabular:consumer_config}
            %\begin{tabular}{|m{0.18cm}|m{0.18cm}|m{0.18cm}|@{}m{0pt}@{}}
            \begin{tabular}{|c|c|m{5cm}|}
                \hline
                \bf{Имя параметра} & \bf{Пример значения} & \bf{Описание} \\ \hline

                LOG\_FILE & \begin{lstlisting}[basicstyle=\small]
'/var/log/twnews_consumer.log'
                \end{lstlisting} & Путь до файла с логом \\ \hline

                LOG\_LEVEL & \begin{lstlisting}[basicstyle=\small]
logging.INFO
                \end{lstlisting} & Уровень подробности лога \\ \hline

                TWNEWS\_DATA\_PATH & \begin{lstlisting}[basicstyle=\small]
'/home/user/twnews_data/'
                \end{lstlisting} & Путь до директории, в которую будут сохранены данные \\ \hline

                RSS\_FEEDS &
                \begin{lstlisting}[basicstyle=\small]
{
'lenta': {'rss_url':
'http://lenta.ru/rss'},
'rt': {'rss_url':
'https://russian.rt.com/rss'},
}
                \end{lstlisting} & Новостные источники, которые требуется выкачать \\ \hline

                TWEETS\_LANGUAGES & \begin{lstlisting}[basicstyle=\small]
['ru']
                \end{lstlisting} & Список языков, твиты с использованием которых выкачиваются из твиттера \\ \hline
            \end{tabular}
            %\end{sideways}
        \end{table}

    \subsubsection{Установка}
        Для установки приложения, необходимо зайти в папку \lstinline{consumer} с исходным кодом пакет twnews\_consumer, находящуюся в корне репозитория, и выполнить команду:

        \begin{lstlisting}
$ make install
        \end{lstlisting}
        Во время установки, с целью распаковки секретного ключа, необходимого для работы с API твиттера, требуется ввести секретный пароль.

    \subsubsection{Использование}
        Результатом работы пакета является множество новостей и твитов, выкаченных за время работы программы.
        Для новостей сохраняются заголовок, краткое описание, ссылка на новость, время публикации и имя ресурса, на котором новость была опубликована.
        Для твитов сохраняются текст, численный уникальный идентификатор ,время публикации, информацию о хэштегах и ссылках и является ли он ретвитом.

        Приложение обладает интерфейсом командной строки.
        Для старта скачивания новостей, необходимо запустить команду:
        \begin{lstlisting}
$ twnews_consumer download --news
        \end{lstlisting}
        Для старта скачивания сообщения твиттера, необходимо запустить команду:
        \begin{lstlisting}
$ twnews_consumer download --tweets
        \end{lstlisting}
        Для завершения скачивания, необходимо использовать сочетание клавиш <<Ctrl-C>>.
        Приложение обработает прерывание и корректно завершится.

        Приложение записывает вспомогательную информацию о своём статусе и обработанных ошибках в файл лога.
        Поэтому из файла лога можно узнать актуальную информацию о работе приложения. Пример:
        \begin{lstlisting}
$ tail -f  /var/log/twnews_consumer.log
2016-04-05 11:37:14: RSS> Start consume rss feeds
2016-04-05 11:37:17: TWITTER> Starting write to /mnt/yandex.disk/twnews_data/logs/tweets.shelve
2016-04-05 11:37:17: TWITTER> Starting to consume twitter
2016-04-05 12:33:32: TWITTER> ('Connection broken: IncompleteRead(0 bytes read, 512 more expected)', IncompleteRead(0 bytes read, 512 more expected))
        \end{lstlisting}

        Приложение twnews\_consumer устойчиво к ошибкам, возникающим при получении новостей и твитов,
        востановление работы заключается в перезапуске скачивания информации.
        Ввиду этого приложение реализует скачивание данных согласно семантики \textit{at-most-once}~---~
        гарантируется отсутствие дублей, но допускается потеря данных.