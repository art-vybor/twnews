\subsection{Пакет twnews}
    Пакет twnews располагается в папке \lstinline{core} в корне репозитория.
    Приложение позволяет обрабатывать данные полученные с помощью консьюмера с целью автоматического установления связей
    между твитами и новостными статьями и оценки качества полученного решения.

    \subsubsection{Конфигурирование}
        Конфигурирование пакета twnews производится в файле \lstinline{twnews/defaults.py}.
        После изменения параметров, необходимо переустановить пакет.
        Описание задаваемых параметров находится в таблице~\ref{tabular:core_config}.
        \begin{table}[h!]
            \small
            \caption{Описание конфигурации пакета twnews \bigskip}
            \center

            \label{tabular:core_config}
            \begin{tabular}{|c|c|m{5cm}|}
                \hline
                \bf{Имя параметра} & \bf{Пример значения} & \bf{Описание} \\ \hline

                LOG\_FILE & \begin{lstlisting}[basicstyle=\small]
'/var/log/twnews.log'
                \end{lstlisting} & Путь до файла с логом \\ \hline

                LOG\_LEVEL & \begin{lstlisting}[basicstyle=\small]
logging.INFO
                \end{lstlisting} & Уровень подробности лога \\ \hline

                TWNEWS\_DATA\_PATH & \begin{lstlisting}[basicstyle=\small]
'/home/user/twnews_data/'
                \end{lstlisting} & Путь до директории, в которую были скачены данные приложением twnews\_consumer \\ \hline

                DATASET\_FRACTION & \begin{lstlisting}[basicstyle=\small]
1.0
\end{lstlisting} & Часть множества скаченных твитов на основе которых происходит построение набора данных \\ \hline

                TMP\_FILE\_DIRECTORY & \begin{lstlisting}[basicstyle=\small]
'/tmp/twnews/'
                \end{lstlisting} & Путь до директории в которую будут сохранены временные данные \\ \hline

                 DEFAULT\_WTMF\_OPTIONS & \begin{lstlisting}[basicstyle=\small]
{
    'DIM': 90,
    'WM': 0.95,
    'ITERATIONS': 1,
    'LAMBDA': 1.95
}
                \end{lstlisting} & Настройки метода WTMF \\ \hline

                 DEFAULT\_WTMFG\_OPTIONS & \begin{lstlisting}[basicstyle=\small]
{
    'DIM': 220,
    'WM': 5,
    'ITERATIONS': 1,
    'DELTA': 0.06,
    'LAMBDA': 6
}
                \end{lstlisting} & Настройки метода WTMF-G \\ \hline

            \end{tabular}
        \end{table}

    \subsubsection{Установка}
        Для установки приложения, необходимо зайти в папку \lstinline{core} с исходным кодом пакет twnews, находящуюся в корне репозитория, и выполнить команду:
        \begin{lstlisting}
$ make install
        \end{lstlisting}
        Для повышения производительности рекомендуется вручную собрать пакет numpy с использованием низкоуровневой математической библиотеки OpenBLAS~\cite{blas_installation}.

    \subsubsection{Использование}
        Финальным результатом работы приложения является получение рекомендаций для произвольных твитов.
        Построение рекомендаций происходит в несколько стадий~(каждая стадия представляет собой отдельный вызов приложения),
        количество стадий различается для различных методов рекомендаций.

        Приложение обладает интерфейсом командной строки.
        Для удобства использования функционал приложения разбивается на набор независимых друг от друга точек входа,
        каждая из которых представляет интерфейс для выполнения одной из стадий работы построения рекомендаций.
        Список всех точек входа представлен в таблице~\ref{tabular:entry_point}.

         \begin{table}[h!]
            \small
            \caption{Описание точек входа пакета twnews \bigskip}
            \center

            \label{tabular:entry_point}
            \begin{tabular}{|c|c|m{7cm}|}
                \hline
                \bf{Название точки входа} & \bf{Пример команды запуска} & \bf{Описание} \\ \hline
                tweets\_sample & twnews tweets\_sample & Получение случайного набора твитов из собранных данных \\ \hline
                resolver & twnews resolver & Расшифровка сокращённых ссылок \\ \hline
                build\_dataset & twnews build\_dataset & автоматическое построение набора данных \\ \hline
                train & twnews train & Построение модели для методов WTMF и WTMF-G \\ \hline
                apply & twnews apply & Применение модели для методов WTMG и WTMF-G \\ \hline
                tfidf & twnews tfidf & Применение метода TF-IDF \\ \hline
                build\_recommendation & twnews build\_recommendation & Построение рекомендаций \\ \hline
                recommendation & twnews recommendation & Обработка полученных рекомендаций \\ \hline
            \end{tabular}
        \end{table}
        В дальнейшем приводится более детальное описание каждой точки входа. Для каждой точки входа в качестве примера использования
        приводится две команды командной строки: вызов и результат вызова автоматически порождаемой приложением справочной информации,
        которая описывает параметры передаваемые в точку входа,
        и один из вариантов вызова точки входа.

        %
        % tweets_sample
        %

        Точка входа tweets\_sample позволяет получить случайный набор твитов из собранных данных; параметризуется двумя необязательными параметрами
        length~---~задаёт количество твитов в результате, и output\_dir~---~определяет директорию в которой будет сохранён файл с твитами;
        на выход в результирующей директории порождается файл с твитами, его имя печается на экран. Пример использования:
        \begin{lstlisting}
$ twnews tweets_sample -h
usage: twnews tweets_sample [-h] [--length LENGTH] [--output_dir OUTPUT_DIR]

optional arguments:
  -h, --help            show this help message and exit
  --length LENGTH       num of tweets in sample (default: 10)
  --output_dir OUTPUT_DIR
                        output directory (default: /home/avybornov/tmp)
$ twnews tweets_sample --length 100
        \end{lstlisting}

        %
        % resolver
        %

        Точка входа resolver позволяет расшифровать все сокращённые ссылки, используемые в скаченном приложением twnews\_consumer множестве данных~(параметр resolve).
        Отображение коротких ссылок в расшифрованные хранится отдельным файлом и в дальнешем используется при построении набора данных.
        Также точка входа resolver позволяет получить статистику по все расшифрованным ссылкам~(параметр analyze).
        Пример использования:
        \begin{lstlisting}
$ twnews resolver -h
usage: twnews resolver [-h] (--resolve | --analyze)

optional arguments:
  -h, --help  show this help message and exit
  --resolve   resolve urls from all tweets (default: False)
  --analyze   print stats of resolved urls (default: False)

$ twnews resolver resolve
        \end{lstlisting}

        %
        % build_dataset
        %

        Точка входа build\_dataset позволяет автоматически построить набор данных; параметризуется двумя параметрами
        unique\_words~---~задаёт процент уникальных слов в твите для пар твит-новость, и output\_dir~---~определяет директорию в которой будет сохранён файл с набором данных;
        Имя построенного набора данных печается на экран. Пример использования:
        \begin{lstlisting}
$ twnews build_dataset -h
usage: twnews build_dataset [-h] [--unique_words UNIQUE_WORDS]
                            [--output_dir OUTPUT_DIR]

optional arguments:
  -h, --help            show this help message and exit
  --unique_words UNIQUE_WORDS
                        percent of unique words in tweet by corresponding news
                        (default: 0.0)
  --output_dir OUTPUT_DIR
                        output directory (default: /home/avybornov/tmp)

$ twnews build_dataset
        \end{lstlisting}

        %
        % train
        %

        Точка входа train позволяет построить модели для методов WTMF и WTMF-G; каждый метод параметризуется тремя параметрами
        dataset~---~название используемого набора данных, input\_dir~---~директория где находится используемый набор данных и
        output\_dir~---~директория в которую будет сохранён файл с модель и файл содержащий новости и твиты из набора данных с векторами для сравнения;
        Имя построенной модели печается на экран. Пример использования:
        \begin{lstlisting}
$ twnews train -h
usage: twnews train [-h] (--wtmf | --wtmf_g) --dataset DATASET
                    [--input_dir INPUT_DIR] [--output_dir OUTPUT_DIR]

optional arguments:
  -h, --help            show this help message and exit
  --wtmf                wtmf method (default: False)
  --wtmf_g              wtmf_g method (default: False)
  --dataset DATASET     dataset name (default: None)
  --input_dir INPUT_DIR
                        input directory (default: /home/avybornov/tmp)
  --output_dir OUTPUT_DIR
                        output directory (default: /home/avybornov/tmp)

$ twnews train --wtmf --dataset dataset_auto
        \end{lstlisting}

        %
        % apply
        %

        Точка входа apply позволяет применить ранее построенные модели для методов WTMF и WTMF-G на набор твитов;
        каждый метод параметризуется четырьмя параметрами: model~---~название используемого набора данных,
        tweets~---~название файла с набором твитов, который был построен с использованием точки входа tweets\_sample,
        input\_dir~---~директория где находится используемая модель и набор твитов;
        output\_dir~---~директория в которую будет сохранён файл содержащий новости из набора данных (использованного на этапе обучения)
        и твиты из переданного файла с векторами для сравнения;
        Имя полученного файла печается на экран. Пример использования:
        \begin{lstlisting}
$ twnews apply -h
usage: twnews apply [-h] (--wtmf | --wtmf_g) --model MODEL --tweets TWEETS
                    [--input_dir INPUT_DIR] [--output_dir OUTPUT_DIR]

optional arguments:
  -h, --help            show this help message and exit
  --wtmf                wtmf method (default: False)
  --wtmf_g              wtmf_g method (default: False)
  --model MODEL         model name (default: None)
  --tweets TWEETS       name of file with tweets (default: None)
  --input_dir INPUT_DIR
                        input directory (default: /home/avybornov/tmp)
  --output_dir OUTPUT_DIR
                        output directory (default: /home/avybornov/tmp)
$ twnews apply --wtmf --model WTMF_model --tweets tweets_file
        \end{lstlisting}

        %
        % tfidf
        %

        Точка входа tfidf позволяет применить метод TFIDF для нахождения векторов сравнения для новостей из набора данных и твитов из
        переданного файла~(если файл не указан используются твиты из набора данных);
        параметризуется четырьмя параметрами
        dataset~---~название используемого набора данных,
        tweets~---~название файла с набором твитов, который был построен с использованием точки входа tweets\_sample,
        input\_dir~---~директория где находится используемый датасет и набор твитов;
        output\_dir~---~директория в которую будет сохранён файл содержащий новости и твиты с векторами для сравнения;
        Имя порождённого файла, который содержит новости и твиты с векторами для сравнения, печается на экран. Пример использования:
        \begin{lstlisting}
$ twnews tfidf -h
usage: twnews tfidf [-h] --dataset DATASET --tweets TWEETS
                    [--input_dir INPUT_DIR] [--output_dir OUTPUT_DIR]

optional arguments:
  -h, --help            show this help message and exit
  --dataset DATASET     dataset name (default: None)
  --tweets TWEETS       name of file with tweets (default: None)
  --input_dir INPUT_DIR
                        input directory (default: /home/avybornov/tmp)
  --output_dir OUTPUT_DIR
                        output directory (default: /home/avybornov/tmp)

$ twnews tfidf --dataset dataset_manual --tweets tweets_file
        \end{lstlisting}


        %
        % build_recommendation
        %

        Точка входа build\_recomendation позволяет построить рекомендации по набору новостей и твитов из набора данных с векторами для сравнения
        ~(если указан файл содержащий твиты с векторами для сравнения, то твиты берутся из него);
        параметризуется четырьмя параметрами
        dataset\_applied~---~название используемого набора данных с векторами для сравнения,
        tweets\_applied~---~название файла с набором твитов c векторами для сравнения,
        input\_dir~---~директория где находится используемый датасет и набор твитов;
        output\_dir~---~директория в которую будет сохранён файл с рекомендациями;
        Имя порождённого файла, который содержит рекомендации, печается на экран. Пример использования:
        \begin{lstlisting}
$ twnews build_recommendation -h
usage: twnews build_recommendation [-h] --dataset_applied DATASET_APPLIED
                                   --tweets_applied TWEETS_APPLIED
                                   [--input_dir INPUT_DIR]
                                   [--output_dir OUTPUT_DIR]

optional arguments:
  -h, --help            show this help message and exit
  --dataset_applied DATASET_APPLIED
                        dataset_applied name (default: None)
  --tweets_applied TWEETS_APPLIED
                        tweets_applied name (default: None)
  --input_dir INPUT_DIR
                        input directory (default: /home/avybornov/tmp)
  --output_dir OUTPUT_DIR
                        output directory (default: /home/avybornov/tmp)

$ twnews build_recommendation --dataset_applied dataset_auto_applied
        \end{lstlisting}

        %
        % recommendation
        %

        Точка входа recommendation позволяет вывести на экран~(параметр eval) или в текстовый файл~(параметр print) построенныe ранее рекомендациям или снять с них метрики~(параметр eval),
        дополнительно параметризуется параметризуется тремя параметрами
        recomended~---~название файла с рекомендациями,
        input\_dir~---~директория где находятся файл с рекомендациями;
        output\_dir~---~директория в которую будет сохранён текстовый файл с рекомендациями;
        Имя порождённого файла, который рекомендации в человекочитаемом формате, печается на экран. Пример использования:
        \begin{lstlisting}
$ twnews recommendation -h
usage: twnews recommendation [-h] (--eval | --dump | --print) --recommended
                             RECOMMENDED [--input_dir INPUT_DIR]
                             [--output_dir OUTPUT_DIR]

optional arguments:
  -h, --help            show this help message and exit
  --eval                eval recomendation results (default: False)
  --dump                dump recomendation result to file (default: False)
  --print               print recomendation result to stdout (default: False)
  --recommended RECOMMENDED
                        tweets_applied name (default: None)
  --input_dir INPUT_DIR
                        input directory (default: /home/avybornov/tmp)
  --output_dir OUTPUT_DIR
                        output directory (default: /home/avybornov/tmp)


$ twnews recommended --eval --recommended dataset_auto_applied
        \end{lstlisting}


        Приложение записывает вспомогательную информацию о своём статусе и обработанных ошибках в файл лога.
        Поэтому из файла лога можно узнать актуальную информацию о работе приложения. Пример:
        \begin{lstlisting}
$ tail -f /var/log/twnews.log
INFO:root:2016-04-08 10:20:08.256411: News successfully loaded
INFO:root:2016-04-08 10:20:23.006948: Function iteration started with time measure
INFO:root:2016-04-08 10:33:32.930520: Function iteration finished in 13m9.9234058857s
INFO:root:2016-04-08 10:33:32.940666: Function find_topk_sim_news_to_tweets started with time measure
INFO:root:2016-04-08 10:34:42.360326: Function find_topk_sim_news_to_tweets finished in 1m9.41950583458s
INFO:root:2016-04-08 10:34:42.587674: Function iteration started with time measure
INFO:root:2016-04-08 10:48:04.453983: Function iteration finished in 13m21.8661620617s
INFO:root:2016-04-08 10:48:04.466846: Function find_topk_sim_news_to_tweets started with time measure
INFO:root:2016-04-08 10:49:18.958096: Function find_topk_sim_news_to_tweets finished in 1m14.4910538197s
INFO:root:2016-04-08 10:49:19.171160: Function iteration started with time measure
        \end{lstlisting}