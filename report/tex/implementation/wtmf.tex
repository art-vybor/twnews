\subsection{Реализация WTMF}
    \subsubsection{Модель данных}
        какие параметры

        от чего зависит построение модели

        какой результат

    \subsubsection{Обучение}

    \subsubsection{Apply}

    \subsubsection{Оптимизация}
        Как установить numpy и scipy
        https://hunseblog.wordpress.com/2014/09/15/installing-numpy-and-openblas/


        Ускорение формулы 4 из статьи WTMF, примерное время за 100 итераций:
        В одну итерацию входит перемножение нескольких больших матриц и взятие обратной матрицы.

        \begin{verbatim}
        ||что сделали||текущее время||во сколько раз стало быстрее||
        |наивная реализация|205с|1|
        |перемножение с помощью OpenBlass|55с|3.73|
        |вынесение общих множителей|15.15с|3.63|
        |переход к работе с sparse матрицами|0.75с|20.2|
        |удаление ненужного приведения матрицы к python list|0.63c|1.21|
        \end{verbatim}
        итого пришли к коду, который работает в 325 раз быстрее наивной реализации

        идеи по до оптимизации: CUDA
        почему быстрее на ноуте? на котором не работает распараллеливание?