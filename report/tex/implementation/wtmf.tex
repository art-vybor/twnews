\subsection{Метод WTMF}
    Модель для метода WTMF построена на основе мнзаранее подготовленного набора данных.
    В контексте работы набор данных состоит из множества новостей и твитов, из которых в процессе работы извлекается набор текстов
    ~(для твита~---~текст твита, для новости~---~конкатенация заголовка и краткого изложения статьи).

    По множеству текстов, которые получены из набора данных, построена модель, пригодная для сериализации, состоящая из матрицы $P$
    ~(здесь и далее используются обозначения введённые в главе \ref{subsubsec:wtmf}).
    Построение модели зависит от четырёх констант:
    \begin{enumerate}
        \item $K$~---~размерность вектора, по которому производится сравнение~
        (если TF-IDF матрица $X$ была размера $M \times N$, то по завершении работы алгоритма будут получены две матрицы $P$ размера $K \times M$ и $Q$ размера $K \times N$);
        \item $I$~---~число итераций алгоритма построения модели;
        \item $w_M$~---~коэффициент, задающий вес негативного сигнала при построении матрицы весов $W$;
        \item $\lambda$~---~регуляризирующий член.
    \end{enumerate}

    Применение полученной модели на множество твитов представляет собой следующий процесс:
    сначала строится TF-IDF матрица $X$ для новостей из набора данных и множества твитов, затем на основе новой матрицы $X$ строится весовая матрица $W$,
    и наконец на основе построенных матриц $X$ и $W$ и посчитанной на этапе обучения матрицы $P$ выполняется половина итерации алгоритма обучения,
    а именно получение матрицы $Q$ по матрице $P$:
    $$Q_{\cdot, j} = (P W'_j P^T + \lambda I)^{-1} P W'_j X_{j,\cdot}.$$
    В результате получаем вектора для сравнения твитов из заданного множества.