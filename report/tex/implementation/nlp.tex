\subsection{Обработка естественного языка}
\label{subsubsec:lemma}
    Работа посвящена поиску семантической близости текстов, поэтому в ней имеет место использование решений таких задач обработки естественного языка, как:
    \begin{enumerate}
        \item токенизация~---~разбиение предложения на слова;
        \item лемматизация~---~процесс приведения словоформы к лемме;
        \item извлечение именованных сущностей.
    \end{enumerate}
    Описанные выше задачи решены с использованием набора сторонних библиотек для языка Python, а именно:
    ntlk~---~платформа, для написания приложений на языке Python, обрабатывающих естественный язык;
    pymorphy2~---~морфологический анализатор~\cite{pymorphy};
    polyglot~---~библиотека, позволяющая извлекать именованные сущности из текстов на разных языках~\cite{polyglot}.

    Для решения задачи токенезации используется стандартный токенизатор, реализованный в ntlk.
    Задача лемматизации решается в случае русского языка с помощью морфологического анализатора pymorphy2,
    в случае английского языка с помощью морфологического анализатора на базе WordNet~\cite{wordnet}, реализованного в ntlk.

    Извлечение именованных сущностей происходит с помощью библиотеки polyglot.
    В используемой библиотеке реализуется выявление именованных сущностей на основе заранее сформированного и размеченного корпуса именованных сущностей.
    Корпус формируется на основе данных из Википедии.

    В работе используется понятие нормальной формы текста, под ним подразумевается текст преобразованный по следующему алгоритму:
    \begin{enumerate}
        \item текст токенизирован;
        \item в полученном множестве токенов, оставлены только слова;
        \item каждое полученное слово отображено в свою лемму~(лемматизировано);
        \item все слова входящие в список стоп-слов, удалены;
        \item оставшиеся слова объеденены в одну строку через пробел.
    \end{enumerate}
    Полученная в результате работы алгоритма строка и является нормальной формой текста.
%
%
%
%
%    \subsubsection{Лемматизация}
%    \label{subsubsec:lemma}
%        Лемматизация~---~процесс приведения словоформы к лемме.
%
%        как используется
%
%        %https://pythonprogramming.net/stop-words-nltk-tutorial/?completed=/tokenizing-words-sentences-nltk-tutorial/
%    \subsubsection{Извлечение имён собственных}
%        что это
%
%        Обзор подходов
%
%        объяснение используемого
