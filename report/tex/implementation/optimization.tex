
\subsection{Эффективная работа с матрицами}
    Построение и применение моделей WTMF и WTMF-G требует большого количества операций над матрицами, что на практике занимает продолжительное время.
    Поэтому задача по повышению эффективности работы с матрицами актуальна.

    Для эффективной работы с матрицами используются программные библиотеки для языка Python numpy и
    scipy~(базируется на библиотеке numpy и расширяет её функционал).

    Оптимизируется формула получения строк матрицы $P$, используемая при построении моделей WTMF и WTMF-G.
    На каждой итерации построения модели происходит многократное выполнение формулы (количество выполнений порядка $10^4$, зависит от размера корпуса):
    $$P_{i, \cdot} = (Q W'_i Q^T + \lambda I)^{-1} Q W'_i X_{i,\cdot}^T.$$

    В начале была написана наивная реализация алгоритма, которая показала производительность, не приемлемую в рамках решения задачи.
    Затем наивная реализация оптимизировалась следующим образом:
    \begin{enumerate}
        \item переход к перемножению матриц с использованием высокопроизводительной библиотеки для языка С OpenBlass~(в библиотеке numpy существует возможность перейти к использованию для работы с матрицами некоторых библиотек, написанных на языке С~\cite{blas_installation});
        \item сохранение в отдельной переменной переиспользуемых результатов вычислений над матрицами;
        \item переписывание кода для работы с разреженными матрицами;
        \item удаление лишних приведений матриц к формату python list и обратно.
    \end{enumerate}
    Результаты оптимизации приведены в таблице~\ref{tabular:matrix_optimization}.

    \begin{table}[ht!]
        %\small
        \caption{Оптимизация работы с матрицами\bigskip}
        \centering

        \label{tabular:matrix_optimization}
        \begin{tabular}{|p{5cm}|c|c|}
            \hline
            \bf{\specialcell{Добавленная \\ оптимизация}} &
            \bf{\specialcell{Время за \\ 100 итераций~(c)}} &
            \bf{\specialcell{Прирост \\ производительности \\ (раз) }} \\ \hline

            Наивная реализация & 205 & 1 \\ \hline
            Перемножение с помощью OpenBlass & 55 & 3.73 \\ \hline
            Переиспользование результатов & 15.15 & 3.63 \\ \hline
            Работа с разреженными матрицами & 0.75 & 20.2 \\ \hline
            Сокращение количества приведений типов & 0.63 & 1.21 \\ \hline
        \end{tabular}
    \end{table}
    Получили, что оптимизированное решение работает в 325 раз быстрее наивной реализации.
    Дальнейшая оптимизация не производилась, так как получено решение работающее за приемлемое время.
%    Дополнительную оптимизацию можно произвести с помощью использования библиотек для работы с матрицами, использующих видеокарт
%
%
%
%    идеи по дальнейшей оптимизации: CUDA