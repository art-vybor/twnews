\subsection{Получение данных}
    Для получения данных~(твитов и новостей) необходимо:
    \begin{enumerate}
        \item реализовать скачивание твитов и новостей из разных новостных источников;
        \item определить формат хранения скачиваемых данных;
        \item скачать твиты за определённый промежуток времени;
        \item расшифровать сокращённые URL, широко употребляемые в твиттере;
        \item на основе статистики употребляемых в твиттере ссылок получить список наиболее популярных новостных источников;
        \item в течение длительного времени собрать данные как с твиттера, так и с новостных источников.
    \end{enumerate}

    \subsubsection{Cкачивание твитов}
        Для скачивания данных твиттера используется Twitter Streaming API~---~сервис,
        предоставляющий разработчикам возможность в реальном времени получить поток данных твиттера~\cite{twitter_streaming}.
        С помощью Twitter Streaming API можно бесплатно получить 1\% от всех опубликованных твитов.

        Для работы с Twitter Streaming API необходимо на сайте~https://apps.twitter.com/ зарегистрировать новое приложение и получить набор секретных ключей,
        которые требуются для авторизации.
        Для упрощения работы с Twitter Streaming API используется библиотека tweepy~\cite{tweepy}, предоставляющая удобный интерфейс на языке Python.

        \textcolor{red}{Получаемый формат данных}
        \textcolor{red}{Что в нём нас интересует}

        \textcolor{red}{Оставляем только русские}

    \subsubsection{Скачивание новостных статей}
        Новостные статьи скачиваются в формате
        \textit{RSS}~---~ семейство XML-форматов, предназначенных для описания лент новостей, анонсов статей, изменений в блогах и т.п.
        Формат RSS выбран ввиду его поддержки всеми популярными новостными источниками.

        Для работы с потоками RSS используется библиотека feedparser~\cite{feedparser}, позволяющая скачивать и анализировать данные в формате RSS.

        \textcolor{red}{Что в нём нас интересует}

    \subsubsection{Расшифровка сокращённых URL}
        \textit{Сокращение URL}~---~это сервис, предоставляемый разными компаниями, заключающийся в создании дополнительного, в общем случае более короткого URL, введущего на искомый адрес.
        Обычно применяется с целью экономии длины сообщения или для предотвращения непреднамеренно искажения URL.
        В общем случае механизм сокращений реализовывается путём переадресации короткого URL на искомый.

        В твиттере все ссылки автоматически сокращается с помощью сервиса \textit{t.co}. Также многие ссылки добавляются в твиттер уже сокращёнными через сторонние сервисы.
        Для автоматического выявления связей между твитами и новостями с целью построения тестового набора данных необходимо уметь по сокращённому URL получить исходный.

        \textit{Расшифровка сокращённых URL}~---~процесс получения по сокращённому URL исходного адреса.
        На практике часто встречается применение сокращения URL каскадом: сокращение уже сокращённого URL,
        в таком случае расшифровка заключается в получении исходного URL, который не является сокращённой ссылкой.
        Можно трактовать задачу расшифровки следующим образом: необходимо получить URL адрес на котором завершится процесс переадресации.

        Расматриваемая задача требует обработки большого количества твитов и следовательно большого количества расшифровок сокращённых URL~(
        в главе~\ref{sec:impl_dataset} получено, что количество ссылок, требуемых для анализа превышает $10^5$).
        Поэтому возникает требование к повышенной эффективности решения.

        В качестве базового решения используется стандартный API языка Python, позволяющий получить содержимое веб-страницы по URL,
        а следовательно адрес целевой страницы на которую вела сокращённая ссылка. Случаи в которых исходный URL не был получен, будем называть ошибочными.
        Базовое решение было оптимизировано следующим образом:
        \begin{enumerate}
            \item Работа только с заголовками ответа. Это позволило снизить количество данных пересылаемых по сети.
            Работа с заголовками требует логики для принятия решения об остановке~---~то есть выявления искомого URL.
            \item Использование многопоточности.
            Так как большую часть времени код, получающий заголовок страницы ждёт ответа сервера, то асинхронность позволит значительно увеличить быстродействие.
            \item Использование "воронки" данных. При увеличении количества потоков стало появляться большее количество ошибок,
            ввиду того, что загруженность интернет-канала повышает время ответа http-запросов.
            Для их снижения было выбран подход "воронки" данных с последующей коррекцией ошибок. Данный подход на первом этапе обрабатывает все ссылки в $N$ потоков,
            на втором этапе все ошибочные ссылки полученные на первом обрабатываются в $N/10$ потоков и так далее, вплоть до 1 потока на итерацию.
        \end{enumerate}
        Замеры времени работы в рамках оптимизации приведены в таблице~\ref{}.
        Также для обоснования необходимости самостоятельной реализации подобного алгоритма в таблице приводится сравнение с популярным сервисом \
        для расшифровки ссылок \textit{api.unshorten.it}~(работа с сервисом производилась через публичный api).

        \textcolor{red}{Таблица времени работы}

        Пример результата

    \subsubsection{Выявление источников новостей}
        Задача выявления источников новостей требует статистического исследования ссылок, которые встречаются в твитах.
        Для определения ссылок ведущих на новостные источники из всех URL извлекалось полное доменное имя~(в дальнейшем доменное имя).
        Также стоит отметить, что новостные агрегаторы~(к примеру Яндекс-новости, Рамблер-новости) не рассматривались ввиду того, что они агрегируют
        очень большое количество новостных статей с множества разнородных источников. То есть очень сложно собрать и в дальнейшем обрабатывать эталонный набор новостей.

        Для грубой оценки использовалась выборка 1, содержащая 35704 твитов, 13670 ссылок, 12510 уникальных ссылок.
        Статистика по 20 наиболее часто встречаемым доменным именам в выборке 1 представленна в таблице~\ref{tabular:domain_1}.
        \begin{table}[ht!]
            %\small
            \caption{20 наиболее часто вречаемых доменных имён в выборке 1 (всего 12510 уникальных ссылок)\bigskip}
            \centering

            \label{tabular:domain_1}
            \begin{tabular}{|c|c|c|c|}
                \hline
                \bf{\specialcell{Доменное имя}} &
                \bf{\specialcell{Количество \\ ссылок}} &
                \bf{\specialcell{Процент от \\ общего числа\\ ссылок}} &
                \bf{\specialcell{Новостной источник}} \\ \hline
                twitter.com & 3521 & 25.76 & нет \\ \hline
                www.facebook.com & 1418 & 10.37 & нет \\ \hline
                t.co & 405 & 2.96 & нет \\ \hline
                www.youtube.com & 315 & 2.30 & нет \\ \hline
                news.yandex.ru & 239 & 1.75 & нет \\ \hline
                su.epeak.in & 214 & 1.57 & нет \\ \hline
                www.instagram.com & 198 & 1.45 & нет \\ \hline
                www.periscope.tv & 191 & 1.40 & нет \\ \hline
                l.ask.fm & 121 & 0.89 & нет \\ \hline
                lifenews.ru & 109 & 0.80 & да \\ \hline
                ria.ru & 108 & 0.79 & да \\ \hline
                vk.com & 93 & 0.68 & нет \\ \hline
                news.7crime.com & 82 & 0.60 & нет \\ \hline
                lenta.ru & 74 & 0.54 & да \\ \hline
                russian.rt.com & 61 & 0.45 & да \\ \hline
                linkis.com & 57 & 0.42 & нет \\ \hline
                www.gazeta.ru & 53 & 0.39 & да \\ \hline
                tass.ru & 43 & 0.31 & да \\ \hline
                www.swarmapp.com & 42 & 0.31 & нет \\ \hline
                pi2.17bullets.com & 36 & 0.26 & нет \\ \hline
            \end{tabular}
        \end{table}

        Как видно из таблицы~\ref{tabular:domain_1} популярные новостные агенства составляют лишь малую долю от общего количества используемых ссылок (3.3\%).
        Для получения более точной количественной информации за неделю собрана выборка 2, содержащая 341863 твитов, 134945 ссылок, 115940 уникальных ссылок.
        Статистика по 20 наиболее часто используемым доменным именам в выборке 2 представленна в таблице~\ref{tabular:domain_2}.
        \begin{table}[ht!]
            %\small
            \caption{20 наиболее часто вречаемых доменных имён в выборке 2 (всего 115940 уникальных ссылок)\bigskip}
            \centering

            \label{tabular:domain_2}
            \begin{tabular}{|c|c|c|c|}
                \hline
                \bf{\specialcell{Доменное имя}} &
                \bf{\specialcell{Количество \\ ссылок}} &
                \bf{\specialcell{Процент от \\ общего числа\\ ссылок}} &
                \bf{\specialcell{Новостной источник}} \\ \hline
                twitter.com & 36807 & 31.75 & нет \\ \hline
                apps.facebook.com & 6234 & 5.38 & нет \\ \hline
                www.youtube.com & 3659 & 3.16 & нет \\ \hline
                m.vk.com & 2400 & 2.07 & нет \\ \hline
                www.periscope.tv & 2215 & 1.91 & нет \\ \hline
                news.yandex.ru & 2041 & 1.76 & нет \\ \hline
                www.instagram.com & 1798 & 1.55 & нет \\ \hline
                su.epeak.in & 1624 & 1.4 & нет \\ \hline
                www.facebook.com & 1406 & 1.21 & нет \\ \hline
                lifenews.ru & 888 & 0.77 & да \\ \hline
                ria.ru & 863 & 0.74 & да \\ \hline
                l.ask.fm & 803 & 0.69 & нет \\ \hline
                vk.com & 696 & 0.6 & нет \\ \hline
                lenta.ru & 647 & 0.56 & да \\ \hline
                pi2.17bullets.com & 577 & 0.5 & нет \\ \hline
                news.7crime.com & 567 & 0.49 & нет \\ \hline
                russian.rt.com & 564 & 0.49 & да \\ \hline
                www.gazeta.ru & 523 & 0.45 & да \\ \hline
                linkis.com & 485 & 0.42 & нет \\ \hline
                ask.fm & 430 & 0.37 & нет \\ \hline
            \end{tabular}
        \end{table}

        Как видно из таблицы~\ref{tabular:domain_2} среди твитов, собранных на довольно большом промежутке времени~(неделя), популярные новостные источники
        составляют лишь малую долю от общего числа употребляемых ссылок (3\%).

        Было принято решение одновременно использовать 5 самых популярных новостных источников, а именно: \url{ria.ru},
        \url{lifenews.ru}, \url{lenta.ru}, \url{russian.rt.com}, \url{www.gazeta.ru}.

        %~(во время работы над дипломом новостная слубжба lifenews.ru сменила доменное имя на life.ru, поэтому в последующих выборках будет упоминаться именно life.ru)

        \clearpage
%
%    \subsubsection{Обработка данных}
%        Для твиттера и новостей
%
%        Известные проблемы:
%        * Ретвиты обрезаются.
%
%        Берём поля:
%        * retweet - \lstinline{parsed_tweet['retweeted'] or text.startswith('RT @')}
%        * text.
%        * created\_at - время по гринвичу.
%        * timestamp\_ms - timestamp по местному времени (в нашем случае +0300).
%        * lang.
%
%        Ретвиты скипаем, ввиду описанной выше проблемы. Надо разобраться как брать для ретвита исходный твит.
%
%        Формируем tsv по lang:
%        timestamp\_ms, text
%
%        Пример строения типичного сообщения твиттера, получаемого с помощью Twitter Stream API.



%\begin{lstlisting}
%{
%    "contributors": null,
%    "coordinates": null,
%    "created_at": "Wed Mar 09 00:24:55 +0000 2016",
%    "entities": {
%        "hashtags": [
%            {
%                "indices": [
%                    61,
%                    74
%                ],
%                "text": "OneDirection"
%            },
%            {
%                "indices": [
%                    75,
%                    94
%                ],
%                "text": "YouKnowYouLoveThem"
%            }
%        ],
%        "symbols": [],
%        "urls": [],
%        "user_mentions": [
%            {
%                "id": 4387486337,
%                "id_str": "4387486337",
%                "indices": [
%                    3,
%                    16
%                ],
%                "name": "HELP 1D",
%                "screen_name": "HELPONEDVOTE"
%            },
%            {
%                "id": 77504008,
%                "id_str": "77504008",
%                "indices": [
%                    95,
%                    107
%                ],
%                "name": "RADIO DISNEY",
%                "screen_name": "radiodisney"
%            }
%        ]
%    },
%    "favorite_count": 0,
%    "favorited": false,
%    "filter_level": "low",
%    "geo": null,
%    "id": 707361488508469248,
%    "id_str": "707361488508469248",
%    "in_reply_to_screen_name": null,
%    "in_reply_to_status_id": null,
%    "in_reply_to_status_id_str": null,
%    "in_reply_to_user_id": null,
%    "in_reply_to_user_id_str": null,
%    "is_quote_status": false,
%    "lang": "pt",
%    "place": null,
%    "retweet_count": 0,
%    "retweeted": false,
%    "retweeted_status": {
%        "contributors": null,
%        "coordinates": null,
%        "created_at": "Tue Mar 08 21:38:42 +0000 2016",
%        "entities": {
%            "hashtags": [
%                {
%                    "indices": [
%                        43,
%                        56
%                    ],
%                    "text": "OneDirection"
%                },
%                {
%                    "indices": [
%                        57,
%                        76
%                    ],
%                    "text": "YouKnowYouLoveThem"
%                }
%            ],
%            "symbols": [],
%            "urls": [],
%            "user_mentions": [
%                {
%                    "id": 77504008,
%                    "id_str": "77504008",
%                    "indices": [
%                        77,
%                        89
%                    ],
%                    "name": "RADIO DISNEY",
%                    "screen_name": "radiodisney"
%                }
%            ]
%        },
%        "favorite_count": 13,
%        "favorited": false,
%        "filter_level": "low",
%        "geo": null,
%        "id": 707319658517549057,
%        "id_str": "707319658517549057",
%        "in_reply_to_screen_name": null,
%        "in_reply_to_status_id": null,
%        "in_reply_to_status_id_str": null,
%        "in_reply_to_user_id": null,
%        "in_reply_to_user_id_str": null,
%        "is_quote_status": false,
%        "lang": "pt",
%        "place": {
%            "attributes": {},
%            "bounding_box": {
%                "coordinates": [
%                    [
%                        [
%                            -44.062789,
%                            -20.059816
%                        ],
%                        [
%                            -44.062789,
%                            -19.777568
%                        ],
%                        [
%                            -43.856856,
%                            -19.777568
%                        ],
%                        [
%                            -43.856856,
%                            -20.059816
%                        ]
%                    ]
%                ],
%                "type": "Polygon"
%            },
%            "country": "Brasil",
%            "country_code": "BR",
%            "full_name": "Belo Horizonte, Brasil",
%            "id": "d9d978b087a92583",
%            "name": "Belo Horizonte",
%            "place_type": "city",
%            "url": "https://api.twitter.com/1.1/geo/id/d9d978b087a92583.json"
%        },
%        "retweet_count": 82,
%        "retweeted": false,
%        "source": "<a href=\"http://twitter.com\" rel=\"nofollow\">Twitter Web Client</a>",
%        "text": "Eeh a tag n\u00e3o subiu em\nFAMILY ONED\n- Maria\n#OneDirection #YouKnowYouLoveThem @radiodisney",
%        "truncated": false,
%        "user": {
%            "contributors_enabled": false,
%            "created_at": "Sat Dec 05 22:28:59 +0000 2015",
%            "default_profile": false,
%            "default_profile_image": false,
%            "description": "Projeto feito na inten\u00e7\u00e3o de ajudar os meninos nas vota\u00e7\u00f5es. Ative as notifica\u00e7\u00f5es e participe de mutir\u00f5es. Adms:Anny, Cah, Maria, Mary, Kaah, Biiah, Mari.",
%            "favourites_count": 4013,
%            "follow_request_sent": null,
%            "followers_count": 5901,
%            "following": null,
%            "friends_count": 5866,
%            "geo_enabled": true,
%            "id": 4387486337,
%            "id_str": "4387486337",
%            "is_translator": false,
%            "lang": "pt",
%            "listed_count": 3,
%            "location": "SNAP : PROJETOHELP",
%            "name": "HELP 1D",
%            "notifications": null,
%            "profile_background_color": "000000",
%            "profile_background_image_url": "http://abs.twimg.com/images/themes/theme1/bg.png",
%            "profile_background_image_url_https": "https://abs.twimg.com/images/themes/theme1/bg.png",
%            "profile_background_tile": false,
%            "profile_banner_url": "https://pbs.twimg.com/profile_banners/4387486337/1457296538",
%            "profile_image_url": "http://pbs.twimg.com/profile_images/706651250323025923/Csjoq0NA_normal.jpg",
%            "profile_image_url_https": "https://pbs.twimg.com/profile_images/706651250323025923/Csjoq0NA_normal.jpg",
%            "profile_link_color": "FF691F",
%            "profile_sidebar_border_color": "000000",
%            "profile_sidebar_fill_color": "000000",
%            "profile_text_color": "000000",
%            "profile_use_background_image": false,
%            "protected": false,
%            "screen_name": "HELPONEDVOTE",
%            "statuses_count": 8533,
%            "time_zone": null,
%            "url": null,
%            "utc_offset": null,
%            "verified": false
%        }
%    },
%    "source": "<a href=\"http://twitter.com/download/android\" rel=\"nofollow\">Twitter for Android</a>",
%    "text": "RT @HELPONEDVOTE: Eeh a tag n\u00e3o subiu em\nFAMILY ONED\n- Maria\n#OneDirection #YouKnowYouLoveThem @radiodisney",
%    "timestamp_ms": "1457483095658",
%    "truncated": false,
%    "user": {
%        "contributors_enabled": false,
%        "created_at": "Tue Feb 02 18:00:32 +0000 2016",
%        "default_profile": true,
%        "default_profile_image": false,
%        "description": "ACESSE NOT\u00cdCIA FOTOS E V\u00cdDEOS SEBRE ONE DIRECTION NO BRASIL",
%        "favourites_count": 117,
%        "follow_request_sent": null,
%        "followers_count": 30,
%        "following": null,
%        "friends_count": 35,
%        "geo_enabled": false,
%        "id": 4872198435,
%        "id_str": "4872198435",
%        "is_translator": false,
%        "lang": "pt",
%        "listed_count": 0,
%        "location": "Brasil",
%        "name": "ACESSO 1D",
%        "notifications": null,
%        "profile_background_color": "F5F8FA",
%        "profile_background_image_url": "",
%        "profile_background_image_url_https": "",
%        "profile_background_tile": false,
%        "profile_banner_url": "https://pbs.twimg.com/profile_banners/4872198435/1454436907",
%        "profile_image_url": "http://pbs.twimg.com/profile_images/694584374961004545/G-Oh7i6P_normal.jpg",
%        "profile_image_url_https": "https://pbs.twimg.com/profile_images/694584374961004545/G-Oh7i6P_normal.jpg",
%        "profile_link_color": "2B7BB9",
%        "profile_sidebar_border_color": "C0DEED",
%        "profile_sidebar_fill_color": "DDEEF6",
%        "profile_text_color": "333333",
%        "profile_use_background_image": true,
%        "protected": false,
%        "screen_name": "acesso1DcomBR",
%        "statuses_count": 1050,
%        "time_zone": null,
%        "url": null,
%        "utc_offset": null,
%        "verified": false
%    }
%}
%
%
%\end{lstlisting}