\subsection{Метод WTMF-G}
    Построение модели для метода WTMF-G основывается на построение модели метода WTMF.
    Набор данных состоит из множества новостей и твитов и связей вида текст-текст, из которых, в процессе работы извлекается набор текстов.
    ~(для твита~---~текст твита, для новости~---~конкатенация заголовка и краткого изложения статьи).

    По множеству текстов, которые получены из набора данных, построена пригодная для сериализации модель, представляющая собой матрицу $P$.
    Построение модели зависит от четырёх констант:
    \begin{enumerate}
        \item $K$~---~размерность вектора, по которому производится сравнение~
        (если TF-IDF матрица $X$ была размера $M \times N$, то по завершении работы алгоритма будут получены две матрицы $P$ размера $K \times M$ и $Q$ размера $K \times N$);
        \item $I$~---~число итераций алгоритма построения модели;
        \item $w_M$~---~коэффициент, задающий вес негативного сигнала при построении матрицы весов $W$;
        \item $\delta$~---~коэффициент, задающий степень влияния связей вида текст-текст.
    \end{enumerate}

    Применение полученной модели на множество твитов производится аналогично применению модели для метода WTMF за исключением двух моментов:
    во-первых, необходимо на основе новостей из набора данных и множества твитов перестроить связи текст-текст, во-вторых получение матрицы $Q$ происходит по следующей формуле:
    $$Q_{\cdot, j} = (P W'_j P^T + \lambda I + \delta  L_j^2 Q_{\cdot,n(j)} diag(L^2_{n(j)})Q_{\cdot,n(j)}^T)^{-1}   (P W'_j X_{j,\cdot} + \delta  L_j Q_{\cdot,n(j)} L_{n(j)}).$$
    В результате получаем вектора для сравнения твитов из заданного множества.