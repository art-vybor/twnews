\section*{Введение}
\addcontentsline{toc}{section}{Введение}
    В современном мире всё больший вес приобретают социальные медиа~(преимущественно социальные сети).
    Их главное отличие от традиционных медиа (газеты, тв) заключается в том, что контент порождается тысячами и миллионами людей.
    Социальные медиа не заменяют традиционные новостные источники, а дополняют их.
    Они могут служить полезным социальным датчиком того, насколько популярна история~(тема) и насколько долго.
    Часто, обсуждения в социальных медиа основаны на событиях из новостей и, наоборот, социальные медиа влияют на новостные события.

    Одной из самых популярных социальных сетей является Twitter~(Твиттер)~---~социальная сеть для публичного обмена сообщениями.
    Главной особенностью Твиттера является малый размер сообщений~(140 символов), называемых твитами.
    Часто твиты являют собой описание происходящего прямо сейчас события, отклик на него.

    Происходящие в мире события описываются статьями в новостных изданиях.
    Новостные статьи и твиты пользователей твиттера не редко описывают одно и то же событие.
    Существует актуальная проблема установления связей между твитами и новостными статьями, которые описывают одно и тоже событие.
    Выявление связи между сообщениями твиттера и новостями позволит как расширить информативность твитов, так и обогатить новости.

    Среди преимуществ расширения новости с помощью твитов можно выделить такие, как определение отношения аудитории к новости,
    дополнительные признаки для тематической классификации новостей, дополнительная информация для аннотирования новостей.

    Современные методы обработки естественного языка хорошо работают, используя большой массив текста в качестве входных данных, однако, они становятся неэффективными,
    когда применяются на коротких текстах, таких как твиты.
    Существенным преимуществом расширения твита с помощью новости является появляющаяся возможность использования большого количества методов обработки естественного языка.

    Данная работа ставит целью исследование и разработку методов автоматического установления связей между сообщениями твиттера и новостными статьями.

    \clearpage