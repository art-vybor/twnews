\section*{Введение}
\addcontentsline{toc}{section}{Введение}
    В современном мире всё больший вес приобретают социальные медиа~(преимущественно социальные сети). Их главное отличие от традиционных медиа (газеты, тв) заключается в том, что контент порождается тысячами и миллионами людей. Социальные медиа не заменяют традиционные новостные источники, а дополняют их. Они могут служить полезным социальным датчиком того, насколько популярна история (тема) и как долго. Часто обсуждения в социальных медиа основаны на событиях из новостей и, наоборот, социальные медиа влияют на новостные события.

    Одной из самых популярных социальных сетей является Twitter - социальная сеть для публичного обмена сообщениями. Главной особенностью Twitter является малый размер сообщений (140 символов), называемых твитами. Часто твиты являют собой описание, происходящего прямо сейчас события, отклик на него.

    ...

    Выявление связи между сообщениями твиттера (твитов) и новостями позволит как расширить информативность твитов, так и обогатить новости.

    ...

    Преимущества расширения новости с помощью твитов: определение отношения аудитории к новости, дополнительные признаки для тематической классификации новостей, дополнительная информация для аннотирования новостей.

    Современные методы обработки естественного языка хорошо работают, используя большой массив текста в качестве входных данных, однако, они становятся неэффективными, когда применяются на коротких текстах, таких как твиты. Существенным преимуществом расширения твита с помощью новости является появляющаяся возможность использования большого количества методов обработки естественного языка (Natural Language Processing).

    ...


    Данная работа ставит целью исследование и разработку методов автоматического установления связей между сообщениями твиттера и новостными статьями.

    {\color{red} Не существует стандартных решений. и есть считанное количество статей. На основе этих статей будет сделана попытка построить pipeline для получения подобной взаимосвязи.}

    \clearpage