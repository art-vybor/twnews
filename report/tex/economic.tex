\section{Технико-экономическое обоснование}
    Разработка программного обеспечения~---~достаточно трудоемкий и длительный процесс, требующий выполнения большого числа разнообразных операций.
    Организация и планирование процесса разработки программного продукта или программного комплекса при традиционном методе планирования предусматривает выполнение следующих работ~\cite{economic_smirnov}:
    \begin{itemize}
        \item формирование состава выполняемых работ и группировка их по стадиям разработки;
        \item расчет трудоемкости выполнения работ;
        \item установление профессионального состава и расчет количества исполнителей;
        \item определение продолжительности выполнения отдельных этапов разработки;
        \item построение календарного графика выполнения разработки;
        \item контроль выполнения календарного графика.
    \end{itemize}

    Далее приведен перечень и состав работ при разработке программного средства для автоматического установления связей между сообщениями твиттера и новостными статьями.
    Отметим, что процесс разработки программного продукта характеризуется совместной работой разработчиков постановки задач и разработчиков программного обеспечения.

    Укрупненный состав работ по стадиям разработки программного продукта~\cite{gost_34601}~\cite{economic_sajin}:
    \begin{enumerate}
        \item Техническое задание:
            \begin{itemize}
                \item Постановка задач, выбор критериев эффективности,
                \item Разработка технико-экономического обоснования разработки,
                \item Определение состава пакета прикладных программ, состава и структуры информационной базы,
                \item Выбор языков программирования,
                \item Предварительный выбор методов выполнения работы,
                \item Разработка календарного плана выполнения работ;
            \end{itemize}
        \item Эскизный проект:
            \begin{itemize}
                \item Предварительная разработка структуры входных и выходных данных,
                \item Разработка общего описания алгоритмов реализации решения задач,
                \item Разработка пояснительной записки,
                \item Консультации разработчиков постановки задач,
                \item Согласование и утверждение эскизного проекта;
            \end{itemize}
        \item Технический проект:
            \begin{itemize}
                \item Разработка алгоритмов решения задач,
                \item Разработка пояснительной записки,
                \item Согласование и утверждение технического проекта,
                \item Разработка структуры программы,
                \item Разработка программной документации и передача ее для включения в технический проект,
                \item Уточнение структуры, анализ и определение формы представления входных и выходных данных,
                \item Выбор конфигурации технических средств;
            \end{itemize}
        \item Рабочий проект:
            \begin{itemize}
                \item Комплексная отладка задач и сдача в опытную эксплуатацию,
                \item Разработка проектной документации,
                \item Программирование и отладка программ,
                \item Описание контрольного примера,
                \item Разработка программной документации,
                \item Разработка, согласование программы и методики испытаний,
                \item Предварительное проведение всех видов испытаний;
            \end{itemize}
        \item Внедрение:
            \begin{itemize}
                \item Подготовка и передача программной документации для сопровождения с оформлением соответствующего Акта,
                \item Передача программной продукции в фонд алгоритмов и программ,
                \item Проверка алгоритмов и программ решения задач, корректировка документации после опытной эксплуатации программного продукта;
            \end{itemize}
    \end{enumerate}

    Трудоемкость разработки программной продукции зависит от ряда факторов, основными из которых являются следующие: степень новизны разрабатываемого программного комплекса, сложность алгоритма его функционирования, объем используемой информации, вид ее представления и способ обработки, а также уровень используемого алгоритмического языка программирования.
    Чем выше уровень языка, тем трудоемкость меньше.

    По степени новизны разрабатываемый проект относится к \textit{группе новизны A} – разработка программных комплексов, требующих использования принципиально новых методов их создания, проведения НИР и т.п.

    По степени сложности алгоритма функционирования проект относится к \textit{2 группе сложности} - программная продукция, реализующая учетно-статистические алгоритмы.

    По виду представления исходной информации и способа ее контроля программный продукт относится к \textit{группе 12} - исходная информация представлена в форме документов, имеющих различный формат и структуру и \textit{группе 22} - требуется печать документов одинаковой формы и содержания, вывод массивов данных на машинные носители.

    \subsection{Трудоемкость разработки программной продукции}
    \label{subsec:trud}
        Трудоемкость разработки программной продукции~($\tau_{PP}$) может быть определена как сумма величин трудоемкости выполнения отдельных стадий разработки программного продукта из выражения:
        \begin{equation}
            \tau_{PP} = \tau_{TZ} + \tau_{EP} + \tau_{TP} + \tau_{RP} + \tau_{V},
        \end{equation}
        где $\tau_{TZ}$~---~трудоемкость разработки технического задания на создание программного продукта;
        $\tau_{EP}$~---~трудоемкость разработки эскизного проекта программного продукта;
        $\tau_{TP}$~---~трудоемкость разработки технического проекта программного продукта;
        $\tau_{RP}$~---~трудоемкость разработки рабочего проекта программного продукта;
        $\tau_{V}$~---~трудоемкость внедрения разработанного программного продукта.

        \subsubsection{Трудоемкость разработки технического задания}
            Расчёт трудоёмкости разработки технического задания~($\tau_{PP}$)~[чел.-дни] производится по формуле:
            \begin{equation}
                \tau_{TZ} = T^Z_{RZ} + T^Z_{RP},
            \end{equation}
            где $T^Z_{RZ}$~---~затраты времени разработчика постановки задачи на разработку ТЗ,~[чел.-дни];
            $T^Z_{RP}$~---~затраты времени разработчика программного обеспечения на разработку ТЗ,~[чел.-дни].
            Их значения рассчитываются по формулам:
            \begin{equation}
                T^Z_{RZ} = t_Z  \cdot  K^Z_{RZ},
            \end{equation}
            \begin{equation}
                T^Z_{RP} = t_Z  \cdot  K^Z_{RP},
            \end{equation}
            где $t_Z$~--~норма времени на разработку ТЗ на программный продукт~(зависит от функционального назначения и степени новизны разрабатываемого программного продукта),~[чел.-дни].
            В нашем случае по таблице получаем значение~(группа новизны – А, функциональное назначение – технико-экономическое планирование):
            \begin{equation*}
                t_Z = 79.
            \end{equation*}
            $K^Z_{RZ}$~---~коэффициент, учитывающий удельный вес трудоемкости работ, выполняемых разработчиком постановки задачи на стадии ТЗ.
            В нашем случае~(совместная разработка с разработчиком ПО):
            \begin{equation*}
                K^Z_{RZ} = 0.65.
            \end{equation*}
            $K^Z_{RP}$~---~коэффициент, учитывающий удельный вес трудоемкости работ, выполняемых разработчиком программного обеспечения на стадии ТЗ.
            В нашем случае~(совместная разработка с разработчиком постановки задач):
            \begin{equation*}
                K^Z_{RP} = 0.35.
            \end{equation*}
            Тогда:
            \begin{equation*}
                \tau_{TZ} = 79  \cdot (0.35 + 0.65) = 79.
            \end{equation*}

        \subsubsection{Трудоемкость разработки эскизного проекта}
            Расчёт трудоёмкости разработки эскизного проекта~($\tau_{EP}$)~[чел.-дни] производится по формуле:
            \begin{equation}
                \tau_{EP} = T^E_{RZ} + T^E_{RP},
            \end{equation}
            где $T^E_{RZ}$~---~затраты времени разработчика постановки задачи на разработку эскизного проекта~(ЭП),~[чел.-дни];
            $T^E_{RP}$~---~затраты времени разработчика программного обеспечения на разработку ЭП,~[чел.-дни].
            Их значения рассчитываются по формулам:
            \begin{equation}
                T^E_{RZ} = t_E  \cdot  K^E_{RZ},
            \end{equation}
            \begin{equation}
                T^E_{RP} = t_E  \cdot  K^E_{RP},
            \end{equation}
            где $t_E$~--~норма времени на разработку ЭП на программный продукт~(зависит от функционального назначения и степени новизны разрабатываемого программного продукта),~[чел.-дни].
            В нашем случае по таблице получаем значение~(группа новизны – А, функциональное назначение – технико-экономическое планирование):
            \begin{equation*}
                t_E = 175.
            \end{equation*}
            $K^E_{RZ}$~---~коэффициент, учитывающий удельный вес трудоемкости работ, выполняемых разработчиком постановки задачи на стадии ЭП.
            В нашем случае~(совместная разработка с разработчиком ПО):
            \begin{equation*}
                K^E_{RZ} = 0.7.
            \end{equation*}
            $K^E_{RP}$~---~коэффициент, учитывающий удельный вес трудоемкости работ, выполняемых разработчиком программного обеспечения на стадии ТЗ.
            В нашем случае~(совместная разработка с разработчиком постановки задач):
            \begin{equation*}
                K^E_{RP} = 0.3.
            \end{equation*}
            Тогда:
            \begin{equation*}
                \tau_{EP} = 175  \cdot (0.3 + 0.7) = 175.
            \end{equation*}

        \subsubsection{Трудоемкость разработки технического проекта}
            Трудоёмкость разработки технического проекта~($\tau_{TP}$)~[чел.-дни] зависит от функционального назначения программного продукта, количества разновидностей форм входной и выходной информации и определяется по формуле:
            \begin{equation}
                \tau_{TP} = (t^T_{RZ} + t^T_{RP}) \cdot K_V \cdot K_R,
            \end{equation}
            где $t^T_{RZ}$~---~норма времени, затрачиваемого на разработку технического проекта~(ТП) разработчиком постановки задач,~[чел.-дни];
            $t^T_{RP}$~---~норма времени, затрачиваемого на разработку ТП разработчиком ПО,~[чел.-дни].
            По таблице принимаем~(функциональное назначение~---~технико-экономическое планирование,
            количество разновидностей форм входной информации~---~2~(твиты, новости),
            количество разновидностей форм выходной информации~---~2~(набор связей твит-новости, оценка работы рекомендательной системы)):
            \begin{equation*}
                t^T_{RZ} = 52,
            \end{equation*}
            \begin{equation*}
                t^T_{RP} = 14.
            \end{equation*}
            $K_R$~---~коэффициент учета режима обработки информации. По таблице принимаем~(группа новизны~---~А, режим обработки информации~---~реальный масштаб времени):
            \begin{equation*}
                K_R = 1.67.
            \end{equation*}
            $K_V$~---~коэффициент учета вида используемой информации, определяется по формуле:
            \begin{equation}
                K_V = \dfrac {K_P \cdot n_P + K_{NS} \cdot n_{NS} + K_B \cdot n_B} {n_P + n_{NS} + n_B },
            \end{equation}
            где $K_P$~---~коэффициент учета вида используемой информации для переменной информации;
            $K_{NS}$~---~коэффициент учета вида используемой информации для нормативно-справочной информации;
            $K_B$~---~коэффициент учета вида используемой информации для баз данных;
            $n_P$~---~количество наборов данных переменной информации;
            $n_{NS}$~---~количество наборов данных нормативно-справочной информации;
            $n_B$~---~количество баз данных.
            Коэффициенты находим по таблице~(группа новизны - А):
            \begin{equation*}
                K_P=1.70,
            \end{equation*}
            \begin{equation*}
                K_{NS}=1.45,
            \end{equation*}
            \begin{equation*}
                K_B=4.37.
            \end{equation*}
            Количество наборов данных, используемых в рамках задачи:
            \begin{equation*}
                n_P=3,
            \end{equation*}
            \begin{equation*}
                n_{NS}=0,
            \end{equation*}
            \begin{equation*}
                n_B=1.
            \end{equation*}
            Находим значение $K_V$:
            \begin{equation*}
                K_V = \dfrac{1.70 \cdot 3+1.45 \cdot 0+4.37 \cdot 1}{3+0+1}=2.3675.
            \end{equation*}
            Тогда:
            \begin{equation*}
                \tau_{TP} = (52+14) \cdot 2.3675 \cdot 1.67 = 261.
            \end{equation*}

        \subsubsection{Трудоемкость разработки рабочего проекта}
            Трудоёмкость разработки рабочего проекта~($\tau_{RP}$)~[чел.-дни] зависит от функционального назначения программного продукта, количества разновидностей форм входной и выходной информации, сложности алгоритма функционирования, сложности контроля информации, степени использования готовых программных модулей, уровня алгоритмического языка программирования и определяется по формуле:
            \begin{equation}
                \tau_{RP} = (t^R_{RZ} + t^R_{RP}) \cdot K_K \cdot K_R \cdot K_Y \cdot K_Z \cdot K_{IA},
            \end{equation}
            где $t^R_{RZ}$~---~норма времени, затраченного на разработку рабочего проекта на алгоритмическом языке высокого уровня разработчиком постановки задач,~[чел.-дни].
            $t^R_{RP}$~---~норма времени, затраченного на разработку рабочего проекта на алгоритмическом языке высокого уровня разработчиком ПО,~[чел.-дни].
            По таблице принимаем~(функциональное назначение~---~технико-экономическое планирование,
            количество разновидностей форм входной информации~---~2~(твиты, новости),
            количество разновидностей форм выходной информации~---~2~(набор связей твит-новости, оценка работы рекомендательной системы)):
            \begin{equation*}
                t^R_{RZ} = 15,
            \end{equation*}
            \begin{equation*}
                t^R_{RP} = 91.
            \end{equation*}
            $K_K$~---~коэффициент учета сложности контроля информации.
            По таблице принимаем~(степень сложности контроля входной информации~---~12, степень сложности контроля выходной информации~---~22):
            \begin{equation*}
                K_K = 1.00.
            \end{equation*}
            $K_R$~---~коэффициент учета режима обработки информации.
            По таблице принимаем~(группа новизны~---~А, режим обработки информации~---~реальный масштаб времени):
            \begin{equation*}
                K_R = 1.75.
            \end{equation*}
            $K_Y$~---~коэффициент учета уровня используемого алгоритмического языка программирования. По таблице принимаем значение~(интерпретаторы, языковые описатели):
            \begin{equation*}
                K_Y = 0.8.
            \end{equation*}
            $K_Z$~---~коэффициент учета степени использования готовых программных модулей. По таблице принимаем~(использование готовых программных модулей составляет около 30%%):
            \begin{equation*}
                K_Z = 0.7.
            \end{equation*}
            $K_{IA}$~---~коэффициент учета вида используемой информации и сложности алгоритма программного продукта, его значение определяется по формуле:
            \begin{equation}
                K_IA = \dfrac {K'_P \cdot n_P + K'_{NS} \cdot n_{NS} + K'_B \cdot n_B} {n_P + n_{NS} + n_B },
            \end{equation}
            где $K'_P$~---~коэффициент учета сложности алгоритма ПП и вида используемой информации для переменной информации;
            $K'_{NS}$~---~коэффициент учета сложности алгоритма ПП и вида используемой информации для нормативно-справочной информации;
            $K'_B$~---~коэффициент учета сложности алгоритма ПП и вида используемой информации для баз данных.
            $n_P$~---~количество наборов данных переменной информации;
            $n_{NS}$~---~количество наборов данных нормативно-справочной информации;
            $n_B$~---~количество баз данных.
            Коэффициенты находим по таблице~(группа новизны - А):
            \begin{equation*}
                K'_P=2.02,
            \end{equation*}
            \begin{equation*}
                K'_{NS}=1.21,
            \end{equation*}
            \begin{equation*}
                K'_B=1.05.
            \end{equation*}
            Количество наборов данных, используемых в рамках задачи:
            \begin{equation*}
                n_P=3,
            \end{equation*}
            \begin{equation*}
                n_{NS}=0,
            \end{equation*}
            \begin{equation*}
                n_B=1.
            \end{equation*}
            Находим значение $K_{IA}$:
            \begin{equation*}
                K_{IA} = \dfrac{2.02 \cdot 3+1.21 \cdot 0+1.05 \cdot 1}{3+0+1}=1.7775.
            \end{equation*}
            Тогда:
            \begin{equation*}
                \tau_{RP} = (15+91)\cdot 1.00 \cdot 1.75 \cdot 0.8 \cdot 0.7 \cdot 1.7775 = 185.
            \end{equation*}

        \subsubsection{Трудоемкость выполнения стадии <<Внедрение>>}
            Расчёт трудоёмкости разработки технического проекта~($\tau_{V}$)~[чел.-дни] производится по формуле:
            \begin{equation}
                \tau_{V} = (t^V_{RZ} + t^V_{RP}) \cdot K_K \cdot K_R \cdot K_Z,
            \end{equation}
            где $t^V_{RZ}$~---~норма времени, затрачиваемого разработчиком постановки задач на выполнение процедур внедрения программного продукта,~[чел.-дни];
            $t^V_{RP}$~---~норма времени, затрачиваемого разработчиком программного обеспечения на выполнение процедур внедрения программного продукта,~[чел.-дни].
            По таблице принимаем~(функциональное назначение~---~технико-экономическое планирование,
            количество разновидностей форм входной информации~---~2~(твиты, новости),
            количество разновидностей форм выходной информации~---~2~(набор связей твит-новости, оценка работы рекомендательной системы)):
            \begin{equation*}
                t^V_{RZ} = 17,
            \end{equation*}
            \begin{equation*}
                t^V_{RP} = 19.
            \end{equation*}
            Коэффициент $K_K$ и $K_Z$ были найдены выше:
            \begin{equation*}
                K_K=1.00,
            \end{equation*}
            \begin{equation*}
                K_Z=0.7.
            \end{equation*}
            $K_R$~---~коэффициент учета режима обработки информации. По таблице принимаем~(группа новизны~---~А, режим обработки информации~---~реальный масштаб времени):
            \begin{equation*}
                K_R = 1.60.
            \end{equation*}
            Тогда:
            \begin{equation*}
                \tau_{V} = (17+19) \cdot 1.00 \cdot 1.60 \cdot 0.7= 40.
            \end{equation*}

        Общая трудоёмкость разработки ПП:
        \begin{equation*}
            \tau_{PP} = 79+175+261+185+40= 740.
        \end{equation*}

    \subsection{Расчет количества исполнителей}
    \label{subsec:slaves}
        Средняя численность исполнителей при реализации проекта разработки и внедрения ПО определяется соотношением:
        \begin{equation}
            N=\dfrac {t} {F},
        \end{equation}
        где $t$~---~затраты труда на выполнение проекта (разработка и внедрение ПО); $F$~---~фонд рабочего времени.
        Разработка велась 5 месяцев с 1 января 2016 по 31 мая 2016.
        Количество рабочих дней по месяцам приведено в таблице~\ref{tabular:work_day}. Из таблицы получаем, что фонд рабочего времени $$F=96.$$
        \begin{table}[h!]
            \caption{Количество рабочих дней по месяцам\bigskip}
            \centering
            \label{tabular:work_day}
            \begin{tabular}{|c|c|c|}
                \hline
                \bf{Номер месяца} & \bf{Интервал дней}& \bf{Количество рабочих дней} \\ \hline
                1 & 01.01.2016~-~31.01.2016 & 15 \\ \hline
                3 & 01.02.2016~-~29.02.2016 & 20 \\ \hline
                4 & 01.03.2016~-~31.03.2016 & 21 \\ \hline
                5 & 01.04.2016~-~30.04.2016 & 21 \\ \hline
                6 & 01.05.2016~-~31.05.2016 & 19 \\ \hline
                \multicolumn{2}{|c|}{Итого} & 96 \\ \hline
            \end{tabular}
        \end{table}

        Получаем число исполнителей проекта:
        \begin{equation*}
            N=\dfrac{740}{96}=8
        \end{equation*}

        Для реализации проекта потребуются 3 старших инженеров и 5 простых инженеров.

    \clearpage
    \subsection{Ленточный график выполнения работ}
        На основе рассчитанных в главах \ref{subsec:trud}, \ref{subsec:slaves} трудоёмкости и фонда рабочего времени найдём количество рабочих дней, требуемых для выполнения каждого этапа разработка.
        Результаты приведены в таблице~\ref{tabular:tau_PP}.
        \begin{table}[ht!]
            %\small
            \caption{Трудоёмкость выполнения работы над проектом \bigskip}
            \centering

            \label{tabular:tau_PP}
            \begin{tabular}{|c|c|c|c|c|}
                \hline
                \bf{\specialcell{Номер \\ стадии}} & \bf{Название стадии} & \bf{\specialcell{Трудоёмкость \\ $[$чел.-дни$]$}} & \bf{\specialcell{Удельный вес \\ $[$\%$]$}} & \bf{\specialcell{Количество\\ рабочих дней}} \\ \hline
                1 &  Техническое задание    & 79  & 11 & 10 \\ \hline
                2 & Эскизный проект         & 175 & 24 & 23 \\ \hline
                3 & Технический проект      & 261 & 35 & 34 \\ \hline
                4 & Рабочий проект          & 185 & 25 & 24 \\ \hline
                5 & Внедрение               & 40  & 5  & 5 \\ \hline
                \multicolumn{2}{|c|}{Итого} & 740 & 100 & 96 \\ \hline
            \end{tabular}
        \end{table}

        Планирование и контроль хода выполнения разработки проводится по ленточному графику выполнения работ.
        По данным в таблице~\ref{tabular:tau_PP} в ленточный график (таблица ~\ref{tabular:lenta}), в ячейки столбца “продолжительности рабочих дней” заносятся времена, которые требуются на выполнение соответствующего этапа.
        Все исполнители работают одновременно.
        \begin{table}[ht!]
            \caption{Ленточный график выполнения работ \bigskip}
            \centering
            \label{tabular:lenta}
            \begin{tabular}{|c|c|c|c|c|c|c|c|c|c|c|c|c|c|c|c|c|c|c|c|c|c|c|c|c|}
                \hline

                & & \multicolumn{23}{|c|}{Календарные дни} \\ \cline{3-25}

                \parbox[t]{3mm}{\multirow{4}{*}[2em]{\rotatebox[origin=c]{90}{Номер стадии}}} &
                \parbox[t]{3.6mm}{\multirow{4}{*}[5.8em]{\rotatebox[origin=c]{90}{Продолжительность [раб.-дни]}}} &
                \rotatebox[origin=c]{90}{~01.01.2016~-~03.01.2016~} &
                \rotatebox[origin=c]{90}{~04.01.2016~-~10.01.2016~} &
                \rotatebox[origin=c]{90}{~11.01.2016~-~17.01.2016~} &
                \rotatebox[origin=c]{90}{~18.01.2016~-~24.01.2016~} &
                \rotatebox[origin=c]{90}{~25.01.2016~-~31.01.2016~} &
                \rotatebox[origin=c]{90}{~01.02.2016~-~07.02.2016~} &
                \rotatebox[origin=c]{90}{~08.02.2016~-~14.02.2016~} &
                \rotatebox[origin=c]{90}{~15.02.2016~-~21.02.2016~} &
                \rotatebox[origin=c]{90}{~22.02.2016~-~28.02.2016~} &
                \rotatebox[origin=c]{90}{~29.02.2016~-~06.03.2016~} &
                \rotatebox[origin=c]{90}{~07.03.2016~-~13.03.2016~} &
                \rotatebox[origin=c]{90}{~14.03.2016~-~20.03.2016~} &
                \rotatebox[origin=c]{90}{~21.03.2016~-~27.03.2016~} &
                \rotatebox[origin=c]{90}{~28.03.2016~-~03.04.2016~} &
                \rotatebox[origin=c]{90}{~04.04.2016~-~10.04.2016~} &
                \rotatebox[origin=c]{90}{~11.04.2016~-~17.04.2016~} &
                \rotatebox[origin=c]{90}{~18.04.2016~-~24.04.2016~} &
                \rotatebox[origin=c]{90}{~25.04.2016~-~01.05.2016~} &
                \rotatebox[origin=c]{90}{~02.05.2016~-~08.05.2016~} &
                \rotatebox[origin=c]{90}{~08.05.2016~-~15.05.2016~} &
                \rotatebox[origin=c]{90}{~16.05.2016~-~22.05.2016~} &
                \rotatebox[origin=c]{90}{~23.05.2016~-~29.05.2016~} &
                \rotatebox[origin=c]{90}{~30.05.2016~-~31.05.2016~}
                \\ \cline{3-25}

                & & \multicolumn{23}{|c|}{Количество рабочих дней} \\ \cline{3-25}

                  &    & 0 & 0 & 5 & 5 & 5 & 5 & 5 & 6 & 3 & 5 & 3 & 5 & 5 & 5 & 5 & 5 & 5 & 5 & 3 & 4 & 5 & 5 & 2 \\ \hline
                1 & 10 &   &   & 5 & 5 &   &   &   &   &   &   &   &   &   &   &   &   &   &   &   &   &   &   &   \\ \hline
                2 & 23 &   &   &   &   & 5 & 5 & 5 & 6 & 2 &   &   &   &   &   &   &   &   &   &   &   &   &   &   \\ \hline
                3 & 34 &   &   &   &   &   &   &   &   & 1 & 5 & 3 & 5 & 5 & 5 & 5 & 5 &   &   &   &   &   &   &   \\ \hline
                4 & 24 &   &   &   &   &   &   &   &   &   &   &   &   &   &   &   &   & 5 & 5 & 3 & 4 & 5 & 2 &   \\ \hline
                5 & 5  &   &   &   &   &   &   &   &   &   &   &   &   &   &   &   &   &   &   &   &   &   & 3 & 2 \\ \hline
            \end{tabular}
        %\end{sidewaystable}
        \end{table}

    \subsection{Определение себестоимости программной продукции}
        Затраты, образующие себестоимость продукции (работ, услуг), состоят из затрат на заработную плату исполнителям, затрат на закупку или аренду оборудования, затрат на организацию рабочих мест, и затрат на накладные расходы.

        В таблице~\ref{tabular:zarplata} приведены затраты на заработную плату и отчисления на социальное страхование в пенсионный фонд, фонд занятости и фонд обязательного медицинского страхования (30.5 \%).
        Для старшего инженера предполагается оклад в размере 120000 рублей в месяц, для инженера предполагается оклад в размере 100000  рублей в месяц.
        \begin{table}[ht!]
            %\small
            \caption{Затраты на зарплату и отчисления на социальное страхование \bigskip}
            \centering

            \label{tabular:zarplata}
            \begin{tabular}{|c|c|c|c|c|}
                \hline
                \bf{Должность} &
                \bf{\specialcell{Зарплата \\ в месяц}} &
                %\bf{\specialcell{Кол-во \\ работников}} &
                \bf{\specialcell{Рабочих \\ месяцев}} &
                \bf{\specialcell{Суммарная \\ зарплата}} &
                \bf{\specialcell{Затраты на \\ социальные нужды}} \\ \hline

                Старший инженер & 120000 & 5 & 600000 & 183000 \\ \hline
                Старший инженер & 120000 & 5 & 600000 & 183000 \\ \hline
                Старший инженер & 120000 & 5 & 600000 & 183000 \\ \hline
                Инженер & 100000 & 5 & 500000 & 152500 \\ \hline
                Инженер & 100000 & 5 & 500000 & 152500 \\ \hline
                Инженер & 100000 & 5 & 500000 & 152500 \\ \hline
                Инженер & 100000 & 5 & 500000 & 152500 \\ \hline
                Инженер & 100000 & 5 & 500000 & 152500 \\ \hline
                \multicolumn{3}{|c|}{Суммарные затраты} & \multicolumn{2}{|c|}{5611500} \\ \hline
            \end{tabular}
        \end{table}

        Расходы на материалы, необходимые для разработки программной продукции, указаны в таблице~\ref{tabular:material}.

        \begin{table}[ht!]
            %\small
            \caption{Затраты на материалы \bigskip}
            \centering

            \label{tabular:material}
            \begin{tabular}{|c|c|c|c|c|}
                \hline
                \bf{\specialcell{Наименование \\ материала}} &
                \bf{\specialcell{Единица \\ измерения}} &
                \bf{\specialcell{Кол-во}} &
                \bf{\specialcell{Цена за \\ единицу, руб.}} &
                \bf{\specialcell{Сумма, руб.}} \\ \hline

                Бумага А4 & Пачка 400 л. & 2 & 200 & 400 \\ \hline
                \specialcell{Картридж для \\ принтера HP P10025} & Шт. & 3 & 450 & 1350 \\ \hline
                \multicolumn{4}{|c|}{Суммарные затраты} & \multicolumn{1}{|c|}{1750} \\ \hline
            \end{tabular}
        \end{table}

        В работе над проектом используется специальное оборудование~---~персональные электронно-вычислительные машины (ПЭВМ) в количестве 8 шт.
        Стоимость одной ПЭВМ составляет 90000 рублей.
        Месячная норма амортизации K = 2,7\%.
        Тогда за 4 месяцев работы расходы на амортизацию составят $P = 90000  \cdot  8  \cdot   0.027  \cdot  4 = 77760$~рублей.

        Общие затраты на разработку программного продукта (ПП) составят

        {\centering{$5611500 +  1750 + 77760  = 5691010$ рублей.}

        }


    \subsection{Определение стоимости программной продукции}
        Для определения стоимости работ необходимо на основании плановых сроков выполнения работ и численности исполнителей рассчитать общую сумму затрат на разработку программного продукта.
        Если ПП рассматривается и создается как продукция производственно-технического назначения,
        допускающая многократное тиражирование и отчуждение от непосредственных разработчиков, то ее цена~$P$ определяется по формуле:
        \begin{equation}
            P = K \cdot C+Pr,
        \end{equation}
        где $C$~---~затраты на разработку ПП (сметная себестоимость);
        $K$~---~коэффициент учёта затрат на изготовление опытного образца ПП как продукции производственно-технического назначения~($K=1.1$);
        $Pr$~---~нормативная прибыль, рассчитываемая по формуле:
        \begin{equation}
            Pr= \frac {C  \cdot  \rho_N} {100},
        \end{equation}
        где $\rho_N$~---~норматив рентабельности, $\rho_N=30\%$;

        Получаем стоимость программного продукта:

        {\centering$P=1.1 \cdot 5691010 + 5691010 \cdot 0.3=7967414$~рублей.

        }
    \subsection{Расчет экономической эффективности}
        Основными показателями экономической эффективности является чистый дисконтированный доход~(NPV) и срок окупаемости вложенных средств.
        Чистый дисконтированный доход определяется по формуле:
        \begin{equation}
            NPV=\sum_{t=0}^T (R_t-Z_t) \cdot \dfrac{1}{(1+E)^t},
        \end{equation}
        где $T$~---~горизонт расчета по месяцам;
        $t$~---~период расчета;
        $R_t$~---~результат, достигнутый на $t$ шаге (стоимость);
        $Z_t$~---~текущие затраты (на шаге $t$);
        $E$~---~приемлемая для инвестора норма прибыли на вложенный капитал.

        На момент начала 2016 года, ставка рефинансирования 11\% годовых~(ЦБ РФ), что эквивалентно 0.87\% в месяц. В виду особенности разрабатываемого продукта он может быть продан лишь однократно.
        Отсюда получаем $$E=0.0087.$$

        В таблице~\ref{tabular:npv} находится расчёт чистого дисконтированного дохода. График его изменения приведён на рисунке~\ref{pic:npv}.

        \begin{table}[ht!]
            %\small
            \caption{Расчёт чистого дисконтированного дохода \bigskip}
            \centering

            \label{tabular:npv}
            \begin{tabular}{|c|c|c|c|c|}
                \hline
                \bf{\specialcell{Месяц}} &
                \bf{\specialcell{Текущие затраты,\\ руб.}} &
                %\bf{\specialcell{Кол-во \\ работников}} &
                \bf{\specialcell{Затраты \\ с начала года, \\ руб.}} &
                \bf{\specialcell{Текущий доход, \\ руб.}} &
                \bf{\specialcell{ЧДД, руб.}} \\ \hline

                Январь  & 1201810 & 1201810 & 0       & -1201810 \\ \hline
                Февраль & 1122300 & 2324110 & 0       & -2314430 \\ \hline
                Март    & 1122300 & 3446410 & 0       & -3417454 \\ \hline
                Апрель  & 1122300 & 4568710 & 0       & -4510964 \\ \hline
                Мая     & 1122300 & 5700730 & 7967414 &  2101032 \\ \hline

            \end{tabular}
        \end{table}

        \begin{figure}[h!]
            \centering
            \begin{tikzpicture}[scale=1]
                \begin{axis}[ylabel=ЧДД (руб.), xlabel=Количество месяцев с начала проекта,
                ] %\tiny
                    \addplot coordinates {
                        (1, -1201810)
                        (2, -2314430)
                        (3, -3417454)
                        (4, -4510964)
                        (5, 2101032)
                    };
                \end{axis}
            \end{tikzpicture}
            \caption{График изменения чистого дисконтированного дохода}
            \label{pic:npv}
        \end{figure}

        Согласно проведенным расчетам, проект является рентабельным.
        Разрабатываемый проект позволит превысить показатели качества существующих систем и сможет их заменить.
        Итоговый ЧДД составил: 2101032 рублей.

    \subsection{Результаты}
        В рамках организационно-экономической части был спланирован календарный график проведения работ по созданию подсистемы поддержки проведения диагностики промышленных, а также были проведены расчеты по трудозатратам.
        Были исследованы и рассчитаны следующие статьи затрат: материальные затраты; заработная плата исполнителей; отчисления на социальное страхование; накладные расходы.

        В результате расчетов было получено общее время выполнения проекта, которое составило 96 рабочих дней,
        получены данные по суммарным затратам на создание системы для автоматического сопоставления твитов и новостных статей, которые составили 5700730 рублей.
        Согласно проведенным расчетам, проект является рентабельным.
        Цена данного программного проекта составила 7967414 рублей, итоговый ЧДД составил 2101032 рублей.

%        Затраты, образующие себестоимость продукции (работ, услуг), группируются в соответствии с их экономическим содержанием по следующим элементам:
%        \begin{enumerate}
%            \item нематериальные активы и затраты на оборудование (за вычетом стоимости возвратных отходов);
%            \item затраты на оплату труда;
%            \item отчисления на социальные нужды;
%            \item амортизация основных фондов;
%            \item прочие затраты.
%        \end{enumerate}
%
%        \subsubsection{Расчет нематериальных активов и затрат на оборудование}
%            В данной статье учитываются суммарные затраты на приобретение оборудования и нематериальных активов, требуемых для разработки данного программного продукта.
%            $$C_{oo}=\sum_i \dfrac{P_{Bi} \cdot  \alpha_i \cdot t_i}{100 \cdot F_D},$$
%            где $P_{Bi}$~---~балансовая цена i-ого вида оборудования, руб.;
%            $\alpha_i$~---~норма годовых амортизационных отчислений для оборудования i-го вида, \%;
%            $F_D$~---~действительный годовой фонд времени, ч;
%            $t_i$~---~время использования i-ого вида оборудования при выполнении данной разработки, ч.
%
%            В таблице~\ref{tabular:equipment} приведена информация об используемом оборудовании.
%
%            \begin{table}[ht!]
%                \caption{Информация об используемом оборудовании и нематериальных активах \bigskip}
%                \centering
%
%                \label{tabular:equipment}
%                \begin{tabular}{|c|c|c|c|c|}
%                    \hline
%                    \bf{Название} & \bf{\specialcell{Единица \\ измерения}} & \bf{Количество} & \bf{\specialcell{Цена за  \\ единицу, руб.}} & \bf{Сумма, руб.} \\ \hline
%                    ПЭВМ & Шт. & 9  & 90000 & 810000 \\ \hline
%                    Python 2.7 & Шт. & 9 & 0 & 0 \\ \hline
%                    Ubuntu 15.04 & Шт. & 9  & 0 & 0 \\ \hline
%                \end{tabular}
%            \end{table}
%
%            Годовой фонд рабочего времени на ПЭВМ (5-ти дневная неделя, 8-и часовой рабочий день) – 2080 ч.
%            Суммарные затраты на оборудование и нематериальные активы составят:
%
%            {\centering{$C_{oo}= \dfrac{810000 \cdot 12 \cdot 740 \cdot 8}{100 \cdot 2080} = 276646$~руб.}
%
%            }
%
%            Затраты, связанные с использованием вычислительной техники определяют по формуле:
%            $$C_{evm}=t^{evm}  \cdot  K_i^{evm}  \cdot  P_{evm}  \cdot  K_{BD}^{evm}  \cdot  K_e^{evm},$$
%            где $t^{evm}$~---~время использования ЭВМ для разработки данного ПП, [час.].
%            По таблице находим значение~(количество разновидностей форм входной информации~---~1, количество разновидностей форм выходной информации~---~2):
%            $$t^{evm}=31.$$
%            $K_i^{evm}$~---~ поправочный коэффициент учета времени использования ЭВМ.
%            Находим по таблице~(для языка высокого уровня, сложность алгоритма ПП~---~2, группа новизны~---~A):
%            $$K_i^{evm}=1.3.$$
%            $P_{evm}$~---~цена 1-го часа работы ЭВМ, [руб./час]. Находим по таблице~(тип ЭВМ~---~PC/AT):
%            $$P_{evm}=15.$$
%            $K_{BD}^{evm}$~---~коэффициент учета степени использования СУБД. Выбираем~(СУБД используется):
%            $$K_{BD}^{evm}=1.1.$$
%            $K_e^{evm}$~---~коэффициент учета быстродействия ЭВМ. Выбираем~(менее $20 \cdot 10^{30}$ опер./с.):
%            $$K_e^{evm}=1.$$
%            Тогда: $C_{evm}=31 \cdot 1.3 \cdot 15 \cdot 1,1 \cdot 1=664,95$~руб.
%        \subsubsection{Расчет основной заработной платы}
%            %\subsubsection{Расчет дополнительной заработной платы}
%        \subsubsection{Отчисления на социальные нужды}
%        \subsubsection{Расчет амортизационных отчислений}
%        \subsubsection{Накладные расходы}
