\section{Руководство пользователя}
\label{sec:documentation}
    Программный комплекс состоит из двух приложений, каждое из которых устанавливается и используется в отдельности.
    \begin{itemize}
        \item twnews\_consumer~---~приложение, предназначенное для того, чтобы получать твиты с твиттера и новости с rss каналов.
        \item twnews~---~приложение, позволяющее по твитам и новостям, произвести все необходимые для автоматического построения рекомендций
        преобразования данных и на основе полученных признаков произвести обучение и оценку модели.
    \end{itemize}

    Оба пакета ориентированы на работу в операционных системах семейства linux.
    Для начала работы необходимо получить содержимое git-репозитория: https://github.com/art-vybor/twnews.git, в котором расположены исходный код обоих приложений.
    Если установлен пакет git, то получение git-репозитория происходит с помощью следующей команды:
    \begin{lstlisting}
$ git clone https://github.com/art-vybor/twnews.git
    \end{lstlisting}
    Для сборки приложений необходимо иметь установленный менеджер пакетов языка Python~---~pip. С помощью него необходимо установить Python библиотеку setuptools:
    \begin{lstlisting}
$ pip install setuptools
    \end{lstlisting}

    Отдельно стоит отметить, что для корректной работы пакетов twnews и twnews\_consumer, все директории, указываемые в конфигурации пакетов должны быть созданы заранее.

    \subsection{Пакет twnews\_consumer}
    Исходный код приложения twnews\_consumer расположен в папке \lstinline{consumer} в корне репозитория.
    Приложение позволяет выкачивать и сохранять твиты и новости в формате, удобном для дальнейшей работы пакета twnews.

    \subsubsection{Конфигурирование}
        Конфигурирование пакета twnews\_consumer производится в файле \lstinline{twnews_consumer/defaults.py}.
        После изменения параметров, необходимо переустановить пакет.
        Описание задаваемых параметров находится в таблице~\ref{tabular:consumer_config}.

        \begin{table}[h!]
            \small
            \caption{Описание конфигурации пакета twnews\_consumer \bigskip}
            \center

            %\begin{sideways}
            \label{tabular:consumer_config}
            %\begin{tabular}{|m{0.18cm}|m{0.18cm}|m{0.18cm}|@{}m{0pt}@{}}
            \begin{tabular}{|c|c|m{5cm}|}
                \hline
                \bf{Имя параметра} & \bf{Пример значения} & \bf{Описание} \\ \hline

                LOG\_FILE & \begin{lstlisting}[basicstyle=\small]
'/var/log/twnews_consumer.log'
                \end{lstlisting} & Путь до файла с логом \\ \hline

                LOG\_LEVEL & \begin{lstlisting}[basicstyle=\small]
logging.INFO
                \end{lstlisting} & Уровень подробности лога \\ \hline

                TWNEWS\_DATA\_PATH & \begin{lstlisting}[basicstyle=\small]
'/home/user/twnews_data/'
                \end{lstlisting} & Путь до директории, в которую будут сохранены данные \\ \hline

                RSS\_FEEDS &
                \begin{lstlisting}[basicstyle=\small]
{
'lenta': {'rss_url':
'http://lenta.ru/rss'},
'rt': {'rss_url':
'https://russian.rt.com/rss'},
}
                \end{lstlisting} & Новостные источники, которые требуется выкачать \\ \hline

                TWEETS\_LANGUAGES & \begin{lstlisting}[basicstyle=\small]
['ru']
                \end{lstlisting} & Список языков, твиты с использованием которых выкачиваются из твиттера \\ \hline
            \end{tabular}
            %\end{sideways}
        \end{table}

    \subsubsection{Установка}
        Для установки приложения, необходимо зайти в папку \lstinline{consumer} с исходным кодом пакет twnews\_consumer, находящуюся в корне репозитория, и выполнить команду:

        \begin{lstlisting}
$ make install
        \end{lstlisting}
        Во время установки, с целью распаковки секретного ключа, необходимого для работы с API твиттера, требуется ввести секретный пароль.

    \subsubsection{Использование}
        Результатом работы пакета является множество новостей и твитов, выкаченных за время работы программы.
        Для новостей сохраняются заголовок, краткое описание, ссылка на новость, время публикации и имя ресурса, на котором новость была опубликована.
        Для твитов сохраняются текст, численный уникальный идентификатор ,время публикации, информацию о хэштегах и ссылках и является ли он ретвитом.

        Приложение обладает интерфейсом командной строки.
        Для старта скачивания новостей, необходимо запустить команду:
        \begin{lstlisting}
$ twnews_consumer download --news
        \end{lstlisting}
        Для старта скачивания сообщения твиттера, необходимо запустить команду:
        \begin{lstlisting}
$ twnews_consumer download --tweets
        \end{lstlisting}
        Для завершения скачивания, необходимо использовать сочетание клавиш <<Ctrl-C>>.
        Приложение обработает прерывание и корректно завершится.

        Приложение записывает вспомогательную информацию о своём статусе и обработанных ошибках в файл лога.
        Поэтому из файла лога можно узнать актуальную информацию о работе приложения. Пример:
        \begin{lstlisting}
$ tail -f  /var/log/twnews_consumer.log
2016-04-05 11:37:14: RSS> Start consume rss feeds
2016-04-05 11:37:17: TWITTER> Starting write to /mnt/yandex.disk/twnews_data/logs/tweets.shelve
2016-04-05 11:37:17: TWITTER> Starting to consume twitter
2016-04-05 12:33:32: TWITTER> ('Connection broken: IncompleteRead(0 bytes read, 512 more expected)', IncompleteRead(0 bytes read, 512 more expected))
        \end{lstlisting}

        Приложение twnews\_consumer устойчиво к ошибкам, возникающим при получении новостей и твитов,
        востановление работы заключается в перезапуске скачивания информации.
        Ввиду этого приложение реализует скачивание данных согласно семантики \textit{at-most-once}~---~
        гарантируется отсутствие дублей, но допускается потеря данных.
    \subsection{Пакет twnews}
    Пакет twnews располагается в папке \lstinline{core} в корне репозитория.
    Приложение позволяет обрабатывать данные полученные с помощью консьюмера с целью автоматического установления связей
    между твитами и новостными статьями и оценки качества полученного решения.

    \subsubsection{Конфигурирование}
        Конфигурирование пакета twnews производится в файле \lstinline{twnews/defaults.py}.
        После изменения параметров, необходимо переустановить пакет.
        Описание задаваемых параметров находится в таблице~\ref{tabular:core_config}.
        \begin{table}[h!]
            \small
            \caption{Описание конфигурации пакета twnews \bigskip}
            \center

            \label{tabular:core_config}
            \begin{tabular}{|c|c|m{5cm}|}
                \hline
                \bf{Имя параметра} & \bf{Пример значения} & \bf{Описание} \\ \hline

                LOG\_FILE & \begin{lstlisting}[basicstyle=\small]
'/var/log/twnews.log'
                \end{lstlisting} & Путь до файла с логом \\ \hline

                LOG\_LEVEL & \begin{lstlisting}[basicstyle=\small]
logging.INFO
                \end{lstlisting} & Уровень подробности лога \\ \hline

                TWNEWS\_DATA\_PATH & \begin{lstlisting}[basicstyle=\small]
'/home/user/twnews_data/'
                \end{lstlisting} & Путь до директории, в которую были скачены данные приложением twnews\_consumer \\ \hline

                DATASET\_FRACTION & \begin{lstlisting}[basicstyle=\small]
1.0
\end{lstlisting} & Часть множества скаченных твитов на основе которых происходит построение набора данных \\ \hline

                TMP\_FILE\_DIRECTORY & \begin{lstlisting}[basicstyle=\small]
'/tmp/twnews/'
                \end{lstlisting} & Путь до директории в которую будут сохранены временные данные \\ \hline

                 DEFAULT\_WTMF\_OPTIONS & \begin{lstlisting}[basicstyle=\small]
{
    'DIM': 90,
    'WM': 0.95,
    'ITERATIONS': 1,
    'LAMBDA': 1.95
}
                \end{lstlisting} & Настройки метода WTMF \\ \hline

                 DEFAULT\_WTMFG\_OPTIONS & \begin{lstlisting}[basicstyle=\small]
{
    'DIM': 220,
    'WM': 5,
    'ITERATIONS': 1,
    'DELTA': 0.06,
    'LAMBDA': 6
}
                \end{lstlisting} & Настройки метода WTMF-G \\ \hline

            \end{tabular}
        \end{table}

    \subsubsection{Установка}
        Для установки приложения, необходимо зайти в папку \lstinline{core} с исходным кодом пакет twnews, находящуюся в корне репозитория, и выполнить команду:
        \begin{lstlisting}
$ make install
        \end{lstlisting}
        Для повышения производительности рекомендуется вручную собрать пакет numpy с использованием низкоуровневой математической библиотеки OpenBLAS~\cite{blas_installation}.

    \subsubsection{Использование}
        Финальным результатом работы приложения является получение рекомендаций для произвольных твитов.
        Построение рекомендаций происходит в несколько стадий~(каждая стадия представляет собой отдельный вызов приложения),
        количество стадий различается для различных методов рекомендаций.

        Приложение обладает интерфейсом командной строки.
        Для удобства использования функционал приложения разбивается на набор независимых друг от друга точек входа,
        каждая из которых представляет интерфейс для выполнения одной из стадий работы построения рекомендаций.
        Список всех точек входа представлен в таблице~\ref{tabular:entry_point}.

         \begin{table}[h!]
            \small
            \caption{Описание точек входа пакета twnews \bigskip}
            \center

            \label{tabular:entry_point}
            \begin{tabular}{|c|c|m{7cm}|}
                \hline
                \bf{Название точки входа} & \bf{Пример команды запуска} & \bf{Описание} \\ \hline
                tweets\_sample & twnews tweets\_sample & Получение случайного набора твитов из собранных данных \\ \hline
                resolver & twnews resolver & Расшифровка сокращённых ссылок \\ \hline
                build\_dataset & twnews build\_dataset & Автоматическое построение набора данных \\ \hline
                train & twnews train & Построение модели для методов WTMF и WTMF-G \\ \hline
                apply & twnews apply & Применение модели для методов WTMF и WTMF-G \\ \hline
                tfidf\_dataset & twnews tfidf\_dataset & Применение метода TF-IDF на набор данных \\ \hline
                tfidf\_tweets & twnews tfidf\_tweets & Применение метода TF-IDF на произвольные твиты\\ \hline
                recommend\_dataset & twnews recommend\_dataset & Построение и оценка качества рекомендаций для набора данных \\ \hline
                recommend\_tweets & twnews recommend\_tweets & Построение рекомендаций для произвольных твитов \\ \hline
            \end{tabular}
        \end{table}
        Далее приводится более детальное описание каждой точки входа. Для каждой точки входа в качестве примера использования
        приводится две команды командной строки: результат вызова автоматически порождаемой приложением справочной информации и пример вызова точки входа.

        %
        % tweets_sample
        %

        Точка входа tweets\_sample позволяет получить случайный набор твитов из собранных данных. Параметризуется двумя необязательными параметрами
        length~---~задаёт количество твитов в результате, и tweets~---~определяет полное имя создаваемого файла, который содержит список твитов.
        Пример использования:
        \begin{lstlisting}
$ twnews tweets_sample -h
usage: twnews tweets_sample [-h] [--length LENGTH] [--tweets TWEETS]

optional arguments:
  -h, --help       show this help message and exit
  --length LENGTH  num of tweets in sample (default: 1000)
  --tweets TWEETS  tweets sample filepath (default:
                   /home/avybornov/tmp/tweets_sample)

$ twnews tweets_sample
get sample of random tweets
sample generated and saved at /home/avybornov/tmp/tweets_sample
        \end{lstlisting}

        %
        % build_dataset
        %

        Точка входа build\_dataset позволяет автоматически построить набор данных. Параметризуется двумя необязательными параметрами
        unique\_words~---~задаёт процент уникальных слов в твите для пар твит-новость, и dataset~---~задаёт полное имя создаваемого файла, который содержит набор данных;
        Пример использования:
        \begin{lstlisting}
$ twnews build_dataset -h
usage: twnews build_dataset [-h] [--unique_words UNIQUE_WORDS]
                            [--dataset DATASET]

optional arguments:
  -h, --help            show this help message and exit
  --unique_words UNIQUE_WORDS
                        percent of unique words in tweet by corresponding news
                        (default: 0.0)
  --dataset DATASET     dataset filepath (default:
                        /home/avybornov/tmp/dataset)

$ twnews build_dataset
building automatic dataset
dataset builded and saved at /home/avybornov/tmp/dataset
        \end{lstlisting}

        %
        % resolver
        %

        Точка входа resolver позволяет расшифровать все сокращённые ссылки, используемые в скаченном приложением twnews\_consumer множестве данных~(параметр resolve).
        Отображение коротких ссылок в расшифрованные хранится отдельным файлом и в дальнешем используется при построении набора данных.
        Также точка входа resolver позволяет получить статистику по все расшифрованным ссылкам~(параметр analyze).
        Пример использования:
        \begin{lstlisting}
$ twnews resolver -h
usage: twnews resolver [-h] (--resolve | --analyze)

optional arguments:
  -h, --help  show this help message and exit
  --resolve   resolve urls from all tweets (default: False)
  --analyze   print stats of resolved urls (default: False)

$ twnews resolver resolve
        \end{lstlisting}

        %
        % train
        %

        Точка входа train позволяет построить модели для методов WTMF и WTMF-G. Каждый метод параметризуется тремя необязательными параметрами:
        dataset~---~полное имя файла, содержащего набора данных, model\_dir~---~директория в которую будет сохранён файл модели,
        dataset\_applied~---~полное имя файла, в который будут записаны новости и твиты из набора данных с векторами для сравнения.
        Пример использования:
        \begin{lstlisting}
$ twnews train -h
usage: twnews train [-h] (--wtmf | --wtmf_g) [--dataset DATASET]
                    [--model_dir MODEL_DIR]
                    [--dataset_applied DATASET_APPLIED]

optional arguments:
  -h, --help            show this help message and exit
  --wtmf                wtmf method (default: False)
  --wtmf_g              wtmf_g method (default: False)
  --dataset DATASET     dataset filepath (default:
                        /home/avybornov/tmp/dataset)
  --model_dir MODEL_DIR
                        name of directory to save model (default:
                        /home/avybornov/tmp)
  --dataset_applied DATASET_APPLIED
                        dataset_applied filepath (default:
                        /home/avybornov/tmp/dataset_applied)

$ twnews train --wtmf
train model
apply model to dataset
model dumped to /home/avybornov/tmp/WTMF_(dataset_auto_0.0)_90_1_1.95_0.95
applied dataset saved: /home/avybornov/tmp/dataset_applied


        \end{lstlisting}

        %
        % apply
        %

        Точка входа apply позволяет применить ранее построенные модели для методов WTMF и WTMF-G на набор твитов.
        Каждый метод параметризуется тремя параметрами: model~---~полное имя файла, содержащего модель,
        tweets~---~полное имя файла с множеством твитов, который был построен с использованием точки входа tweets\_sample,
        tweets\_applied~---~полное имя файла, в который будут записаны полученнные твиты с векторами для сравнения.
        Пример использования:
        \begin{lstlisting}
$ twnews apply -h
usage: twnews apply [-h] (--wtmf | --wtmf_g) [--model MODEL] [--tweets TWEETS]
                    [--tweets_applied TWEETS_APPLIED]

optional arguments:
  -h, --help            show this help message and exit
  --wtmf                wtmf method (default: False)
  --wtmf_g              wtmf_g method (default: False)
  --model MODEL         model filepath (default: /home/avybornov/tmp/WTMF_(dat
                        aset_auto_0.0)_90_1_1.95_0.95)
  --tweets TWEETS       tweets sample filepath (default:
                        /home/avybornov/tmp/tweets_sample)
  --tweets_applied TWEETS_APPLIED
                        tweets_applied filepath (default:
                        /home/avybornov/tmp/tweets_applied)

$ twnews apply --wtmf --model /home/avybornov/tmp/WTMF_\(dataset_auto_0.0\)_90_1_1.95_0.95
apply model
tweets applied and stored at /home/avybornov/tmp/tweets_applied

        \end{lstlisting}

        %
        % tfidf_dataset
        %

        Точка входа tfidf\_dataset позволяет применить метод TFIDF для нахождения векторов сравнения для новостей и твитов из набора данных.
        Параметризуется двумя параметрами
        dataset~---~полное имя файла, содержащего набор данных,
        dataset\_applied~---~полное имя файла, в который будут записаны новости и твиты из набора данных с векторами для сравнения.
        Пример использования:
        \begin{lstlisting}
twnews tfidf_dataset -h
usage: twnews tfidf_dataset [-h] [--dataset DATASET]
                            [--dataset_applied DATASET_APPLIED]

optional arguments:
  -h, --help            show this help message and exit
  --dataset DATASET     dataset filepath (default:
                        /home/avybornov/tmp/dataset)
  --dataset_applied DATASET_APPLIED
                        dataset_applied filepath (default:
                        /home/avybornov/tmp/dataset_applied)

$ twnews tfidf_dataset
apply tfidf to dataset
dataset applied and stored at /home/avybornov/tmp/dataset_applied
        \end{lstlisting}

        %
        % tfidf_tweets
        %

        Точка входа tfidf\_tweets позволяет применить метод TFIDF для нахождения векторов сравнения для набора данных~(как в случае с точкой входа tfidf\_dataset).
        Параметризуется четырьмы параметрами
        dataset~---~полное имя файла, содержащего набор данных,
        dataset\_applied~---~полное имя файла, в который будут записаны новости и твиты из набора данных с векторами для сравнения,
        tweets~---~полное имя файла с множеством твитов, который был построен с использованием точки входа tweets\_sample,
        tweets\_applied~---~полное имя файла, в который будут записаны полученнные твиты с векторами для сравнения.
        Пример использования:
        \begin{lstlisting}
$ twnews tfidf_tweets -h
usage: twnews tfidf_tweets [-h] [--dataset DATASET]
                           [--dataset_applied DATASET_APPLIED]
                           [--tweets TWEETS] [--tweets_applied TWEETS_APPLIED]

optional arguments:
  -h, --help            show this help message and exit
  --dataset DATASET     dataset filepath (default:
                        /home/avybornov/tmp/dataset)
  --dataset_applied DATASET_APPLIED
                        dataset_applied filepath (default:
                        /home/avybornov/tmp/dataset_applied)
  --tweets TWEETS       tweets sample filepath (default:
                        /home/avybornov/tmp/tweets_sample)
  --tweets_applied TWEETS_APPLIED
                        tweets_applied filepath (default:
                        /home/avybornov/tmp/tweets_applied)

$ twnews tfidf_tweets
apply tfidf to dataset
dataset applied and stored at /home/avybornov/tmp/dataset_applied
apply tfidf to tweets
tweets applied and stored at /home/avybornov/tmp/tweets_applied
        \end{lstlisting}


        %
        % recommend_dataset
        %

        Точка входа recommend\_dataset позволяет построить рекомендации по набору новостей и твитов из набора данных с векторами для сравнения
        и получить метрики качества построенных рекомендаций.
        Параметризуется двумя параметрами:
        dataset\_applied~---~название используемого набора данных с векторами для сравнения;
        dump~---~имя файла в который будут сохранены полученные рекомендации.
        Пример использования:
        \begin{lstlisting}
$ twnews recommend_dataset -h
usage: twnews recommend_dataset [-h] [--dataset_applied DATASET_APPLIED]
                                [--dump DUMP]

optional arguments:
  -h, --help            show this help message and exit
  --dataset_applied DATASET_APPLIED
                        dataset_applied filepath (default:
                        /home/avybornov/tmp/dataset_applied)
  --dump DUMP           recommendation dump filepath (default:
                        /home/avybornov/tmp/reccommendation_dump)
$ twnews recommend_dataset
build recommendation
recommendation result evaluation
RR = 0.908454481596
TOP1 = 0.869796484736
TOP3 = 0.940564292322
recommendation dumped to /home/avybornov/tmp/reccommendation_dump
        \end{lstlisting}

        %
        % recommend_tweets
        %

        Точка входа recommend\_tweets позволяет построить рекомендации для произвольных твитов с векторами для сравнения.
        Параметризуется тремя параметрами
        dataset\_applied~---~название используемого набора данных с векторами для сравнения;
        tweets\_applied~---~название файла с набором твитов c векторами для сравнения;
        dump~---~имя файла в который будут сохранены полученные рекомендации.
        Пример использования:
        \begin{lstlisting}
$ twnews recommend_tweets -h
usage: twnews recommend_tweets [-h] [--dataset_applied DATASET_APPLIED]
                               [--tweets_applied TWEETS_APPLIED] [--dump DUMP]

optional arguments:
  -h, --help            show this help message and exit
  --dataset_applied DATASET_APPLIED
                        dataset_applied filepath (default:
                        /home/avybornov/tmp/dataset_applied)
  --tweets_applied TWEETS_APPLIED
                        tweets_applied filepath (default:
                        /home/avybornov/tmp/tweets_applied)
  --dump DUMP           recommendation dump filepath (default:
                        /home/avybornov/tmp/reccommendation_dump)
$ twnews recommend_tweets
recommendation dumped to /home/avybornov/tmp/reccommendation_dump

        \end{lstlisting}


        Приложение записывает вспомогательную информацию о своём статусе и обработанных ошибках в файл лога.
        Поэтому из файла лога можно узнать актуальную информацию о работе приложения. Пример:
        \begin{lstlisting}
$ tail -f /var/log/twnews.log
INFO:root:2016-04-08 10:20:08.256411: News successfully loaded
INFO:root:2016-04-08 10:20:23.006948: Function iteration started with time measure
INFO:root:2016-04-08 10:33:32.930520: Function iteration finished in 13m9.9234058857s
INFO:root:2016-04-08 10:33:32.940666: Function find_topk_sim_news_to_tweets started with time measure
INFO:root:2016-04-08 10:34:42.360326: Function find_topk_sim_news_to_tweets finished in 1m9.41950583458s
INFO:root:2016-04-08 10:34:42.587674: Function iteration started with time measure
INFO:root:2016-04-08 10:48:04.453983: Function iteration finished in 13m21.8661620617s
        \end{lstlisting}