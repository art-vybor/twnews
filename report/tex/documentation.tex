\section{Руководство пользователя}
\label{sec:documentation}
    \textcolor{red}{Информация писалась 3 месяца назад и уже не актуальна. Необходимо переписать главу под текущие реалии}

    Программный комплекс состоит из двух приложений, каждое из которых устанавливается и используется в отдельности.
    \begin{itemize}
        \item twnews\_consumer~---~консьюмер, который позволяет выкачивать твиты с твиттера и новости с rss каналов.
        \item twnews~---~пакет, позволяющий по твитам и новостям, произвести все необходимые преобразования данных и на основе полученных признаков произвести обучение и оценку модели.
    \end{itemize}

    Оба пакета ориентированы на работу в операционных системах из семейства linux. Для начала работы необходимо иметь установленный менеджер пакетов для языка Python~---~pip, а также установить setuptools:
    \begin{lstlisting}
    $ pip install setuptools
    \end{lstlisting}
    Также необходимо выкачать git-репозиторий: https://github.com/art-vybor/twnews.git.
    Если установлен пакет git, то это можно сделать следующим образом:
    \begin{lstlisting}
    $ git clone https://github.com/art-vybor/twnews.git
    \end{lstlisting}

    Для корректной работы пакетов, все указываемые в конфигурации директории должны быть заранее созданы.

    \subsection{Пакет twnews\_consumer}
        Пакет twnews\_consumer располагается в папке \lstinline{consumer} в корне репозитория.
        Он позволяет выкачивать и сохранять в формате, удобном для дальнейшей работы пакета twnews, твиты и новости.

        Конфигурирование пакета производится в файле \lstinline{twnews_consumer/defaults.py}.
        Описание задаваемых параметров находится в таблице~\ref{tabular:consumer_config}.

        \begin{table}[h!]
            \small
            \caption{Описание конфигурации пакета twnews\_consumer \bigskip}
            \centering

            \begin{sideways}
            \label{tabular:consumer_config}
            \begin{tabular}{|c|c|c|}
                \hline
                \bf{Имя параметра} & \bf{Пример значения} & \bf{Описание} \\ \hline
                LOG\_FILE & '/var/log/twnews\_consumer.log' & Путь до файла с логом \\ \hline
                LOG\_LEVEL & logging.INFO & Уровень подробности лога \\ \hline
                TWNEWS\_DATA\_PATH & '/home/avybornov/twnews\_data/' & \specialcell{Путь до директории, \\ в которую будут сохранены данные} \\ \hline
                RSS\_FEEDS & \specialcell{\{'ria': \\ \{'rss\_url': 'http://ria.ru/export/rss2/index.xml'\},\\ 'lifenews': \\ \{'rss\_url': 'http://lifenews.ru/xml/feed.xml'\}\}} & \specialcell{Новостные источники, \\ которые требуется выкачать}\\ \hline
                TWEETS\_LANGUAGES & ['ru'] & \specialcell{Список языков, \\ твиты с использованием которых \\ выкачиваются из твиттера} \\ \hline
            \end{tabular}
        \end{sideways}
        \end{table}

        Результатом работы пакета является множество новостей и твитов, выкаченных за время работы программы. Для новостей сохраняются заголовок, краткое описание, ссылка на новость, время публикации и имя ресурса, на котором новость была опубликована. Для твитов сохраняются текст, время публикации и информация о том, является ли он ретвитом~(ретвит~---~твит, представляющий собой, ссылку на ранее созданный твит).

        \subsubsection{Установка}
            Для установки, необходимо зайти в папку \lstinline{consumer}, находящуюся в корне репозитория и выполнить команду:

            \begin{lstlisting}
    $ make install
            \end{lstlisting}

            Во время установки, нужно будет ввести пароль, для распаковки секретного ключа, который необходим для работы с API твиттера.

        \subsubsection{Использование}
            Для того, чтобы начать выкачивать новости, необходимо запустить команду:
            \begin{lstlisting}
    $ twnews_consumer download --news
            \end{lstlisting}
            Для того, чтобы начать выкачивать сообщения твиттера, необходимо запустить команду:
            \begin{lstlisting}
    $ twnews_consumer download --tweets
            \end{lstlisting}

            Узнать информацию о работе программы можно из файла лога. Пример:
            \begin{lstlisting}
    $ tail -f  /var/log/twnews_consumer.log
    2016-04-05 11:37:14: RSS> Start consume rss feeds
    2016-04-05 11:37:17: TWITTER> Starting write to /mnt/yandex.disk/twnews_data/logs/tweets.shelve
    2016-04-05 11:37:17: TWITTER> Starting to consume twitter
    2016-04-05 12:33:32: TWITTER> ('Connection broken: IncompleteRead(0 bytes read, 512 more expected)', IncompleteRead(0 bytes read, 512 more expected))
            \end{lstlisting}

    \subsection{Пакет twnews}
        Пакет twnews располагается в папке \lstinline{core} в корне репозитория.
        Он позволяет обрабатывать данные полученные с помощью консьюмера с целью построения и оценки качества модели WTMF.

        Конфигурирование пакета производится в файле \lstinline{twnews/defaults.py}.
        Описание задаваемых параметров находится в таблице~\ref{tabular:core_config}.

        \begin{table}[h!]
            \small
            \caption{Описание конфигурации пакета twnews\bigskip}
            \centering

            \begin{sideways}
            \label{tabular:core_config}
            \begin{tabular}{|c|c|c|}
                \hline
                \bf{Имя параметра} & \bf{Пример значения} & \bf{Описание} \\ \hline
                LOG\_FILE & '/var/log/twnews.log' & Путь до файла с логом \\ \hline
                LOG\_LEVEL & logging.INFO & Уровень подробности лога \\ \hline
                TWNEWS\_DATA\_PATH & '/home/avybornov/twnews\_data/' & \specialcell{Путь до рабочей директории в которой \\ лежат выкаченные с помощью консьюмера данные} \\ \hline
                DATASET\_FRACTION & 1.0 & \specialcell{Часть датасета, которая  \\ будет использована для обучения модели } \\ \hline
                TMP\_FILE\_DIRECTORY & '/tmp/twnews/' & \specialcell{Путь до директории в которую \\ будут сохранены временные данные} \\ \hline
            \end{tabular}
        \end{sideways}
        \end{table}

        Результатом работы пакета является построенная модель WTMF, для которой измерено её качество. Перед построением модели, необходимо выполнить команду, которая разрешает ссылки~(может занять длительное время).

        \subsubsection{Установка}
            Для установки, необходимо зайти в папку \lstinline{core}, находящуюся в корне репозитория и выполнить команду:

            \begin{lstlisting}
    $ make install
            \end{lstlisting}

            Для повышения производительности рекомендуется вручную собрать пакет numpy с использованием математической библиотеки OpenBLAS~\cite{blas_installation}.

        \subsubsection{Использование}
            Для того, чтобы разрешить ссылки, необходимо запустить команду:
            \begin{lstlisting}
    $ twnews_consumer --resolve
            \end{lstlisting}

            Для того, чтобы посмотреть статистику по упомянутым в коллекции твитов ссылкам нужно выполнить команду:
            \begin{lstlisting}
    $ twnews_consumer download --analyse_urls
            \end{lstlisting}

            Для построения модели небходимо запустить команду:
            \begin{lstlisting}
    $ twnews_consumer download --run_pipe
            \end{lstlisting}

            Узнать информацию о работе программы можно из файла лога. Пример:
            \begin{lstlisting}
    $ tail -f /var/log/twnews.log
    INFO:root:2016-04-08 10:20:08.256411: News successfully loaded
    INFO:root:2016-04-08 10:20:23.006948: Function iteration started with time measure
    INFO:root:2016-04-08 10:33:32.930520: Function iteration finished in 13m9.9234058857s
    INFO:root:2016-04-08 10:33:32.940666: Function find_topk_sim_news_to_tweets started with time measure
    INFO:root:2016-04-08 10:34:42.360326: Function find_topk_sim_news_to_tweets finished in 1m9.41950583458s
    INFO:root:2016-04-08 10:34:42.587674: Function iteration started with time measure
    INFO:root:2016-04-08 10:48:04.453983: Function iteration finished in 13m21.8661620617s
    INFO:root:2016-04-08 10:48:04.466846: Function find_topk_sim_news_to_tweets started with time measure
    INFO:root:2016-04-08 10:49:18.958096: Function find_topk_sim_news_to_tweets finished in 1m14.4910538197s
    INFO:root:2016-04-08 10:49:19.171160: Function iteration started with time measure
            \end{lstlisting}