\clearpage
\begin{thebibliography}{0}
\addcontentsline{toc}{section}{Список литературы}
%    \bibitem{linking_base} W. Guo, H. Li, H. Ji, and M. T. Diab. Linking tweets to news: A framework to enrich short text data in social media. - ACL, pages 239–249, 2013.
%    \bibitem{linking_news_media} Manos Tsagkias, Maarten de Rijke, Wouter Weerkamp. Linking Online News and Social Media. - ISLA, University of Amsterdam.
%    \bibitem{bridging} T. Hoang-Vu, A. Bessa, L. Barbosa and J. Freire. Bridging Vocabularies to Link Tweets and News. - International Workshop on the Web and Databases (WebDB 2014), Snowbird, Utah, US, 2014.
%    \bibitem{twitterstand} J. Sankaranarayanan, H. Samet, B. Teitler, M. Lieberman, J. Sperling. TwitterStand: news in tweets. - 17th ACM SIGSPATIAL International Conference on Advances in Geographic Information Systems, 2009, Seattle, Washington.
%
%    \bibitem{long_to_short} Ou Jin, Nathan N. Liu, Kai Zhao, Yong Yu, and Qiang Yang. 2011. Transferring topical knowledge from auxiliary long texts for short text clustering. In Proceedings of the 20th ACM international conference on Information and knowledge management.
%
%    \bibitem{wtmf} Weiwei Guo and Mona Diab. 2012a. Modeling sentences in the latent space. In Proceedings of the 50th Annual Meeting of the Association for Computational Linguistics.
%
%    \bibitem{steck_recommender} Harald Steck. 2010. Training and testing of recommender systems on data missing not at random. In Proceedings of the 16th ACM SIGKDD International Conference on Knowledge Discovery and Data Mining.
%
%    \bibitem{matrix_approximation} Nathan Srebro and Tommi Jaakkola. 2003. Weighted low-rank approximations. In Proceedings of the Twentieth International Conference on Machine Learning.
%
%    \bibitem{blas_installation} Eric Huns. Hunseblog on Wordpress: URL: https://hunseblog.wordpress.com/2014/09/15/installing-numpy-and-openblas/.


    \bibitem{economic_sajin} Арсеньев В.В., Сажин Ю.Б. Методические указания к выполнению организационно-экономической части дипломных проектов по созданию программной продукции. М.: изд. МГТУ им. Баумана, 1994. 52 с. 2.
    \bibitem{economic_smirnov} Под ред. Смирнова С.В. Организационно-экономическая часть дипломных проектов исследовательского профиля. М.: изд. МГТУ им. Баумана, 1995. 100 с.
    \bibitem{gost_34601} ГОСТ 34.601 "АС. Стадии создания".

\end{thebibliography}