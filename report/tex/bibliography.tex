
\begin{thebibliography}{0}
%\addcontentsline{toc}{section}{Список литературы}
    \bibitem{wtmf} W. Guo. Modeling sentences in the latent space / W. Guo, M. Diab // Annual Meeting of the Association for Computational Linguistics: Long Papers.~---~2012.~---~C.~864-872.
    \bibitem{tfidf} A. Aizawa. An information-theoretic perspective of tf—idf measures // Information Processing and Management: an International Journa.~---~2003.~---~С.~45-65.
    \bibitem{linking_news_media} M. Tsagkias. Linking Online News and Social Media / M. Tsagkias, M. de Rijke, W. Weerkamp // Fourth ACM international conference on Web search and data mining.~---~2011.~---~С.~565-574.
    \bibitem{bridging} T. Hoang-Vu. Bridging Vocabularies to Link Tweets and News / T. Hoang-Vu, A. Bessa, L. Barbosa and J. Freire // International Workshop on the Web and Databases.~---~2014.
    \bibitem{matrix_approximation} N. Srebro. Weighted low-rank approximations / N. Srebro, T. Jaakkola // 20th International Conference on Machine Learning.~--~2003.
    \bibitem{steck_recommender} H. Steck. Training and testing of recommender systems on data missing not at random // 16th ACM SIGKDD International Conference on Knowledge Discovery and Data Mining.~---~2010.~---~С.~713-722.
    \bibitem{linking_base} W. Guo. Linking tweets to news: A framework to enrich short text data in social media /  W. Guo, H. Li, H. Ji, M. T. Diab // 51st Annual Meeting of the Association for Computational Linguistics.~---~2013.~---~С.~239–249.
    \bibitem{flowchart_gost} ГОСТ 19.701-90. Схемы алгоритмов, программ, данных и систем. Условные обозначения и правила выполнения.~---~М.: Изд-во стандартов, 1990.
    \bibitem{pymorphy} Korobov M. Morphological Analyzer and Generator for Russian and Ukrainian Languages // Analysis of Images, Social Networks and Texts.~---~2015.~---~C.~320-332.
    \bibitem{polyglot} Y. Chen. Building Sentiment Lexicons for All Major Languages / Y. Chen, S. Skiena // Annual Meeting of the Association for Computational Linguistics: Short Papers.~---~2014.~---~С.~383-389.
    \bibitem{wordnet} G. A. Miller. WordNet: a lexical database for English // Communications of the ACM~---~1995~---~C.~39-41.
    \bibitem{scipy} T. E. Oliphant. Python for Scientific Computing // Computing in Science and Engineering.~---~2007.~---~С.~10-20.
    \bibitem{blas_installation} E. Huns. Installing Numpy and OpenBLAS.~---~Электронный ресурс.~---~Режим доступа: \\\url{https://hunseblog.wordpress.com/2014/09/15/installing-numpy-and-openblas/}
    \bibitem{economic_smirnov} Смирнов С.В. Организационно-экономическая часть дипломных проектов исследовательского профиля / С.В. Смирнов, В.В. Степанов, Лилейкина Г.А. и др.~---~М.: изд. МГТУ им. Баумана, 1995.~---~100 с.
    \bibitem{gost_34601} ГОСТ 34.601-90 Информационная технология. Комплекс стандартов на автоматизированные системы. Автоматизированные системы. Стадии создания.~---~М.: Изд-во стандартов, 1990.
    \bibitem{economic_sajin} Арсеньев, В.В. Методические указания к выполнению организационно-экономической части дипломных проектов по созданию программной продукции / В.В. Арсеньев, Ю.Б Сажин.~---~М.: изд. МГТУ им. Баумана, 1994.~---~52 с.

\end{thebibliography}