\vocabularytitle
    В настоящей работе применяют следующие термины с соответствующими определениями.

    \textit{LDA}~(от англ. Latent Dirichlet allocation~---~латентное размещение Дирихле)~---~методов тематического моделирования, позволяющий объяснять результаты наблюдений с помощью неявных групп.

    \textit{TF-IDF}~(от англ. TF~---~term frequency, IDF~---~inverse document frequency)~---~статистическая мера, используемая для оценки важности слова в контексте документа, являющегося частью коллекции документов или корпуса. Вес некоторого слова пропорционален количеству употребления этого слова в документе, и обратно пропорционален частоте употребления слова в других документах коллекции.

    \textit{TF-IDF матрица}~---~матрица, строки которой соответствуют словам из корпуса, а столбцы текстам. Значение ячейки матрицы $(i,j)$ равно значению метрики tf-idf для слова, соответствующего строчке $i$, и текста, соответствующего столбцу $j$.

    \textit{URL}~(от англ. Uniform Resource Locator~---~единый указатель ресурса)~---~единообразный определитель местонахождения ресурса. URL служит стандартизированным способом записи адреса ресурса в сети Интернет.

    \textit{WTMF}~---~метод машинного обучения, применяемый для анализа схожести между короткими текстами~\cite{wtmf}.

    \textit{Аннотирование текста}~---~краткое представление содержания текста в виде аннотации~(обзорного реферата).

    \textit{Именованная сущность}~---~последовательность слов, являющаяся именем, названием, идентификатором, временным, денежным или процентным выражением.

    \textit{Информационный поиск}~---~процесс поиска неструктурированной документальной информации, удовлетворяющей информационные потребности, и наука об этом поиске.

    \textit{Новость}~---~оперативное информационное сообщение, которое представляет политический, социальный или экономический интерес для аудитории в своей свежести, то есть сообщение о событиях произошедших недавно или происходящих в данный момент.

    \textit{Обработка естественного языка}~(англ. Natural language processing)~---~направление математической лингвистики, которое изучает проблемы компьютерного анализа и синтеза естественных языков.

    \textit{Ретвит}~---~сообщение, целиком состоящее из цитирования сообщения одного пользователя Твиттера другим.

    \textit{Твит}~---~термин сервиса микроблоггинга Твиттер, обозначающий сообщение, публикуемое пользователем в его твиттере. Особенностью твита является его длина, которая не может быть больше 140 знаков.

    \textit{Твиттер}~---~cоциальная сеть для публичного обмена короткими~(до 140 символов) сообщениями при помощи веб-интерфейса, SMS, средств мгновенного обмена сообщениями или сторонних программ-клиентов.

    \textit{Тематическое моделирование}~---~способ построения модели коллекции текстовых документов, которая определяет, к каким темам относится каждый из документов.

    \textit{Хэштег}~---~слово или фраза, которым предшествует символ \#, используется в различных социальных сетях~(Twitter, Facebook, Instagram) для объединения группы сообщений по теме или типу. Например: \#искусство, \#техника, \#смешное, \#анекдоты и т.д.




