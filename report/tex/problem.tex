\section{Постановка задачи}
    В разделе~\ref{sec:review} проведено исследование существующих методов автоматического установления связей между сообщениями твиттера и новостными статьями, выбраны методы, на основе которых необходимо реализовать программный комплекс, позволяющий устанавливать связи между твитами и новостными статьями.

    Для установления связей должен быть собран эталонный набор данных, которой состоит из множества твитов, новостей и связей между ними. Выбранный алгоритм WTMF-G накладывает ограничение на формат эталонного набора: для каждого твита существует связь с единственной новостью.

    Решение задачи установления связей между твитами и новостными статьями в общем случае неоднозначно:
    как твиту может соответствовать несколько новостей, так и новостной статье может соотвествовать несколько твитов.
    Отталкиваясь от существующего ограничения: твит связан с единственной новость, получаем что для оценки качества установления связей хорошо подходят метрики, принятые в информационном поиске.
    Для использования подобных метрик будем рассматривать твит как запрос, в терминологии информационного поиска, а список новостей как ответ нашей системы установления связей. То есть для каждого твита мы получаем список новостей, ранжированный по мере убывания их схожести, в дальнейшем будем называть подобный список рекомендацией, а процесс установления связей построением рекомендаций.

    Целью работы является создание программного комплекса, который реализует такие методы машинного обучения как WTMF, WTMF-G, TF-IDF, позволяет для произвольного твита построить рекомендацию новостей с использованием любого из предложенных методов машинного обучения, а также способен оценить качество используемого метода машинного обучения. 